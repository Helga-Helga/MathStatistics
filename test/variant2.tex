\addcontentsline{toc}{section}{Вариант 12}
\section*{Вариант 2}

\subsubsection*{1}

\textit{Задание.}
Средняя прибыль магазина за день составляет $10000$ грн.,
среднее квадратическое отклонение прибыли равно $2000$ грн.
Найдите наименьшее $n$ такое,
что с вероятностью $0.97$ прибыль магазина за $n$ дней будет не меньше чем $100000$ грн.

\textit{Решение.} Обозначим через $ \xi_n$ прибыль магазина за $n$-й день.
Тогда
$$S =
  \xi_1 + \dotsc + \xi_n$$
является прибылью магазина за $n$ дней.
Чтобы найти такое $n$, что
$$P \left( S \geq 100000 \right) =
  0.97,$$
воспользуемся аппроксимацией нормальным распределением:
$$P \left( S \geq 100000 \right) =
  P \left( \frac{S - MS}{ \sqrt{DS}} \geq \frac{100000 - MS}{ \sqrt{DS}} \right) =
  1 - \Phi_t \left( \frac{MS - 100000}{ \sqrt{DS}} \right),$$
где $ \Phi_t$ --- функция распределения стандартного нормального распределения.
Значит находит $n$ из условия
$$ \Phi_t \left( \frac{MS - 100000}{ \sqrt{DS}} \right) =
  0.03.$$
Воспользовавшись таблицей значений функции $ \Phi_t$, находим, что
$$ \frac{MS - 100000}{ \sqrt{DS}} =
  1.88.$$
По условию
$$MS = n \cdot M \xi_1 = n \cdot 10000, \,
  \sqrt{DS} = \sqrt{n \cdot 2000^2} = \sqrt{n} \cdot 2000$$
и значит $10n - 100 = 2 \cdot 1.88 \sqrt{n}$, откуда
$$ \sqrt{n} =
  \frac{2 \cdot 1.88 + \sqrt{ \left( 2 \cdot 1.88 \right)^2 + 4000}}{2 \cdot 10} =
  \frac{2 \cdot 1.88 + 63.36}{20} =
  3.36,$$
и $n = \left( 3.36 \right)^2 = 11.29$, то есть наименьшее $n = 12$.

\subsubsection*{2}

\textit{Задание.}
Пусть $X_1, \dotsc, X_n$ --- выборка из распределения $F$ и $ \nu_n$ ---
количество элементов выборки, которые попали в полуинтервал $ \left( a, b \right] $, где $a < b$ ---
фиксированные числа.
Докажите, что статистика
$$ \frac{ \nu_n}{n}$$
является несмещённой состоятельной оценкой разности $F \left( b \right) - F \left( a \right) $.

\textit{Решение.} Для того, чтобы доказать, что статистика
$$ \frac{ \nu_n}{n}$$
является несмещённой оценкой разности $F \left( b \right) - F \left( a \right) $,
необходимо проверить выполнение условия
$$M \frac{ \nu_n}{n} =
  F \left( b \right) - F \left( a \right).$$
Находим математическое ожидание статистики
$$M \frac{ \nu_n}{n} =
  M \frac{ \sum \limits_{i = 1}^n \mathbbm{1} \left\{ X_i \in \left( a, b \right] \right\} }{n} =
  \frac{1}{n} \cdot
  M \sum \limits_{i = 1}^n \mathbbm{1} \left\{ X_i \in \left( a, b \right] \right\} =
  M \mathbbm{1} \left\{ X_1 \in \left( a, b \right] \right\}.$$
Разбиваем на разность двух вероятностей
$$M \mathbbm{1} \left\{ X_1 \in \left( a, b \right] \right\} =
  P \left( X_1 \leq b \right) - P \left( X_1 < a \right) =
  F \left( b \right) - F \left( a \right).$$

Таким образом,
$$M \frac{ \nu_n}{n} =
  F \left( b \right) - F \left( a \right),$$
а значит, статистика
$$ \frac{ \nu_n}{n}$$
является несмещённой оценкой разности $F \left( b \right) - F \left( a \right) $.

Чтобы доказать состоятельность, нужно найти предел по вероятности статистики
$$ \frac{ \nu_n}{n}$$
при $n \to \infty $.
Поскольку случайные величины
$ \mathbbm{1} \left\{ X_1 \in \left( a, b \right] \right\}, \dotsc,
  X_n \in \left( a, b \right] $
являются независимыми одинаково распределёнными случайными величинами с конечным первым моментом,
то согласно с законом больших чисел
$$ \frac{1}{n} \cdot
  \sum \limits_{i = 1}^n
    \mathbbm{1} \left\{ X_i \in \left( a, b \right] \right\} \overset{P}{ \rightarrow}
  M \mathbbm{1} \left\{ X_1 \in \left( a, b \right] \right\}, \,
  n \to \infty,$$
а тогда при $n \to \infty $ выполняется
$$ \frac{ \nu_n}{n} \overset{P}{ \rightarrow }
  M \mathbbm{1} \left\{ X_1 \in \left( a, b \right] \right\} =
  F \left( b \right) - F \left( a \right).$$

Значит, статистика
$$ \frac{ \nu_n}{n}$$
является состоятельной оценкой разности $F \left( b \right) - F \left( a \right) $.

\subsubsection*{4}

\textit{Задание.}
Пусть $X_1, \dotsc, X_n$ ---
выборка объёма $n \geq 5$ из распределения Пуассона с параметром $ \lambda $.
Для какого параметра $ \theta \left( \lambda \right) $ статистика
$$ \theta_n^* =
  X_1 X_2 X_3 X_4 X_5$$
является несмещённой оценкой?
Является ли $ \theta_n^*$ состоятельной оценкой того же параметра?

\textit{Решение.}
$$M \theta_n^* =
  \theta \left( \lambda \right) =
  M \left( X_1 X_2 X_3 X_4 X_5 \right) =
  MX_1 \cdot MX_2 \cdot MX_3 \cdot MX_4 \cdot MX_5 =
  \left( MX_1 \right)^5.$$
Так как выборка имеет распределение Пуассона, то
$$ \left( MX_1 \right)^5 =
  \lambda^5.$$

Поскольку $ \theta_n^*$ не зависит от $n$,
оценка $ \theta_n^*$ не является состоятельной для $ \lambda^5$.

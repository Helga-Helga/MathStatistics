\addcontentsline{toc}{chapter}{Занятие 2. Свойства оценок}
\chapter*{Занятие 2. Свойства оценок}

\addcontentsline{toc}{section}{Контрольные вопросы и задания}
\section*{Контрольные вопросы и задания}

\subsubsection*{Что называют оценкой неизвестного параметра?}

Статистику, значение которой заменяет неизвестный параметр, называют оценкой этого параметра.

\subsubsection*{Приведите определение оценки: несмещённой, ассимптотически несмещённой,
                состоятельной, сильно состоятельной, оптимальной.}

Оценка $ \hat{ \theta}$  несмещённая,
если $ \forall \theta \in \Theta: \, M_{ \theta } \hat{ \theta } = \theta $.

Асимптотически несмещенная оценка --- такая оценка,
математическое ожидание которой совпадает с оцениваемым параметром при $n \to \infty $.

Оценка $ \hat{ \theta }$ называется состоятельной,
если стремится к истинному значению $ \theta $ по вероятности
$ \hat{ \theta } \overset{P}{ \rightarrow } \theta, \,
  n \to \infty $.

Оценка $ \hat{ \theta }$ называется сильно состоятельной,
если стремится к истинному значению $ \theta $ почти наверное
$ \hat{ \theta } \overset{a.s.}{ \rightarrow } \theta, \,
  n \to \infty $.

Несмещённая оценка $ \hat{ \theta } \in K$
называется оптимальной в классе квадратично интегрируемых оценок $K$,
если для всякой другой несмещённой оценки
$ \tilde{ \theta } \in \Theta \,
  \forall \theta \in \Theta: \,
  D_{ \theta } \hat{ \theta } \leq D_{ \theta } \tilde{ \theta }$
или же
$ \forall \theta \in \Theta, \,
  M_{ \theta } \left( \hat{ \theta } - \theta \right)^ \leq
  M_{ \theta } \left( \tilde{ \theta } - \theta \right)^2$.

\subsubsection*{Что называется среднеквадратическим отклонением оценки?}

$M_{ \theta } \left( \hat{ \theta } - \theta \right) $ --- среднеквадратическое оклонение.

\subsubsection*{Сформулируйте утверждение про поведение выборочных моментов.}

ыборочный начальный момент $M_k \, k$-го порядка стремится к начальному моменту $ \nu_k$
случайной величины $X$, то есть
$$ \lim \limits_{n \to \infty } P \left( \left| M_k - \nu_k \right| \geq \varepsilon \right) =
  0,$$
для любого сколь угодно малого $ \varepsilon > 0$,
если моменты $ \nu_{2k}$ и $ \nu_k$ случайной величины $X$ существуют и конечны.

\subsubsection*{Какая оценка является несмещённой и состоятельной для математического ожидания
                распределения выборки?}

В качестве оценки для математического
ожидания естественно предложить среднееарифметическое наблюденных значений
$$ \tilde{m} =
  \frac{ \sum \limits_{i = 1}^n X_i}{n}.$$

\subsubsection*{Какая статистика является несмещённой оценкой для дисперсии распределения выборки?}

$$ \frac{1}{n - 1} \sum \limits_{k = 1}^n \left( x_k - \overline{x} \right)^2$$
--- несмещённая оценка для $ \sigma^2 = Dx_1$.

\addcontentsline{toc}{section}{Аудиторные задачи}
\section*{Аудиторные задачи}

\subsubsection*{2.4}

\textit{Задание.}
Для выборки равномерного распределения на отрезке $ \left[ \theta , 1 \right] $
проверьте состоятельность и несмещённость оценки $X_{ \left( 1 \right) }$ параметра $ \theta $.

\textit{Решение.} $ \theta $ --- минимальное наблюдение.
Проверяем,
выполняется ли $X_{ \left( 1 \right) } \overset{P}{ \rightarrow } \theta, \, n \to \infty $.

По определению сходимости по вероятности
$$ \forall \varepsilon > 0 \,
  P \left( \left| X_{ \left( 1 \right) } - \theta \right| > \varepsilon \right) \to 0, \,
  n \to \infty.$$

Раскроем модуль
$$P \left\{ X_{ \left( 1 \right) } > \varepsilon + \theta \right\} =
  P \left( X_1 > \varepsilon + \theta, \dotsc, X_n > \varepsilon + \theta \right) =
  \left[ P \left( X_1 > \varepsilon + \theta \right) \right]^n.$$
Подставим значение вероятности из геометрического эксперимента
$$ \left[ P \left( X_1 > \varepsilon + \theta \right) \right]^n =
  \left( \frac{1 - \theta - \varepsilon }{1 - \theta } \right)^n =
  \left( 1 - \frac{ \varepsilon }{1 - \theta } \right)^n \to
  0, \,
  n \to \infty.$$

Число в скобках строго меньше единицы, так как $0 \leq \theta \leq 1$.

Отсюда следует, что оценка состоятельная.

Проверяем несмещённость оценки.
Проверяем, выполняется ли
$$MX_{ \left( 1 \right) } =
  \theta.$$

Нужно найти плотность
$$MX_{ \left( 1 \right) } =
  \int \limits_{ \mathbb{R}} f_{X_{ \left( 1 \right) }} \left( y \right) ydy.$$

Начинаем с функции распределения
$F_{X_{ \left( 1 \right) }} \left( y \right) =
  P \left( X_{ \left( 1 \right) } \leq y \right) $.
Переходим к противоположному событию
$$P \left( X_{ \left( 1 \right) } \leq y \right) =
  1 - P \left( X_{ \left( 1 \right) } > y \right) =
  1 - \left[ P \left( X_1 > y \right) \right]^n.$$
Переходим к противоположному событию
$1 - \left[ P \left( X_1 > y \right) \right]^n =
  1 - \left[ 1 - F \left( y \right) \right]^n$.

Продифференцируем
$$ \frac{dF_{X_{ \left( 1 \right) }} \left( y \right) }{dy} =
  n \left[ 1 - F \left( y \right) \right]^{n - 1} f \left( y \right).$$
На отрезке $ \left[ \theta, 1 \right] $ имеет равномерное распределение
$$n \left[ 1 - F \left( y \right) \right]^{n - 1} f \left( y \right) =
  n \left[ 1 - \frac{y - \theta }{1 - \theta } \right]^{n - 1} \cdot
  \mathbbm{1} \left\{ y \in \left[ \theta, 1 \right] \right\} \cdot \frac{1}{1 - \theta }.$$
Приведём к общему знаменателю
$$n \left[ 1 - \frac{y - \theta }{1 - \theta } \right]^{n - 1} \cdot
  \mathbbm{1} \left\{ y \in \left[ \theta, 1 \right] \right\} \cdot \frac{1}{1 - \theta } =
  \frac{n}{ \left( 1 - \theta \right)^2} \cdot \left( 1 - y \right)^{n - 1} \cdot
  \mathbbm{1} \left\{ y \in \left[ \theta, 1 \right] \right\}.$$

Нашли плотность $X_{ \left( 1 \right) }$ и теперь можем вычислить интеграл
$$MX_{ \left( 1 \right) } =
  \int \limits_{ \theta }^1
    y \cdot \frac{n}{ \left( 1 - \theta \right)^2} \cdot \left( 1 - y \right)^{n - 1}
  dy.$$
Замена:
$$1 - y = z, \,
  dy = -dz, \,
  y = 1 - z, \,
  y = 1 \Rightarrow \implies z = 0, \,
  y = \theta \implies z = 1 - \theta.$$
Подставляя замену, получаем
$$ \int \limits_{ \theta }^1
    y \cdot \frac{n}{ \left( 1 - \theta \right)^2} \cdot \left( 1 - y \right)^{n - 1}
  dy =
  n \cdot \frac{1}{ \left( 1 - \theta \right)^n}
  \int \limits_0^{1 - \theta } \left( 1 - z \right) z^{n - 1} dz.$$
Вычислим интеграл
\begin{equation*}
  \begin{split}
    n \cdot \frac{1}{ \left( 1 - \theta \right)^n}
    \int \limits_0^{1 - \theta } \left( 1 - z \right) z^{n - 1} dz
    \frac{n}{ \left( 1 - \theta \right)^n}
    \left[
      \frac{ \left( 1 - \theta \right)^n}{n} - \frac{ \left( 1 - \theta \right)^{n+1}}{n + 1}
    \right] = \\
    = n \left( \frac{1}{n} - \frac{1 - \theta }{n + 1} \right) =
    1 - \frac{n}{n + 1} \left( 1 - \theta \right).
  \end{split}
\end{equation*}
Раскроем скобки
$$1 - \frac{n}{n + 1} \left( 1 - \theta \right) =
  1 - \frac{n}{n + 1} - \theta \cdot \frac{n}{n + 1} \neq
  \theta.$$
Отсюда следует, что оценка смещённая, но ассимптотически несмещённая, потому что
$$1 - \frac{n}{n + 1} \to 0, \,
  n \to \infty $$
и
$$ \frac{n}{n + 1} \to 1, \,
  n \to \infty.$$

\subsubsection*{2.5}

\textit{Задание.}
Пусть $X_1, \dotsc, X_n$ --- выборка из распределения Пуассона с параметром $ \lambda > 0$.
Выясните, является ли статистика
$$ \frac{1}{n} \sum \limits_{i = 1}^n X_i^2:$$
\begin{enumerate}[label=\alph*)]
  \item несмещённой оценкой для $ \lambda^2$;
  \item состоятельной оценкой для $ \lambda^2$.
\end{enumerate}

\textit{Решение.}
\begin{enumerate}[label=\alph*)]
  \item Нужно проверить, выполняется ли
  $$M \frac{1}{n} \sum \limits_{i = 1}^n X_i^2 =
    \lambda^2.$$
  Преобразуем левую часть
  $$M \frac{1}{n} \sum \limits_{i = 1}^n X_i^2 =
    MX_1^2 =
    DX_1 + \left( MX_1 \right)^2 =
    \lambda + \lambda^2 \neq
    \lambda^2.$$

  Значит, оценка смещённая;
  \item проверяем, имеет ли место
  $$ \frac{1}{n} \sum \limits_{i = 1}^n X_i^2  \overset{P}{ \rightarrow } \lambda^2, \,
    n \to \infty.$$

  По закону больших чисел
  $$ \frac{1}{n} \sum \limits_{i = 1}^n X_i^2 \to
    MX_1^2 =
    \lambda^2 + \lambda \neq
    \lambda^2,$$
  значит, оценка не состоятельная.
\end{enumerate}

\subsubsection*{2.6}

\textit{Задание.}
Пусть $X_1, \dotsc, X_n$ --- выборка из показательного распределения с параметром $ \alpha > 0$.
Докажите, что статистика $1 / \overline{X}$ является состоятельной оценкой для $ \alpha $.

\textit{Решение.} Нужно показать, что
$$ \frac{1}{ \overline{X}} \overset{P}{ \rightarrow } \alpha, \,
   n \to \infty.$$

Выборочное среднее
$$ \overline{X} =
  \frac{1}{n} \sum \limits_{i = 1}^n X_i.$$

По закону больших чисел
$$ \overline{X} \overset{P}{ \rightarrow }
  MX_1 =
  \frac{1}{ \alpha }.$$
Отсюда следует, что
$$ \frac{1}{ \overline{X}} \overset{P}{ \rightarrow }
  \frac{1}{MX_1} =
  \alpha.$$

\subsubsection*{2.7}

\textit{Задание.}
Пусть $X_1, \dotsc, X_n$ --- выборка из нормального распределения $N \left( a, \sigma^2 \right) $.
Докажите, что статистика
$$S_n =
  \frac{1}{2 \left( n - 1 \right) } \sum \limits_{i = 1}^{n - 1} \left( X_{i + 1} - X_i \right)^2$$
является несмещённой и состоятельной оценкой для $ \sigma^1$.

\textit{Решение.} Нужно проверить условие $MS_n = \sigma^2$.

Разность двух соседних элементов выборки имеет распределение
$$X_{i + 1} - X_i \sim
  N \left( 0, 2 \sigma^2 \right).$$

Найдём математическое ожидание статистики
$$MS_n =
  M \frac{1}{2 \left( n - 1 \right) } \cdot
  \sum \limits_{i = 1}^{n - 1} \left( X_{i + 1} - X_i \right)^2 =
  \frac{1}{2 \left( n - 1 \right) } \sum \limits_{i = 1}^n M \left( X_{i + 1} - X_i \right)^2.$$
Случайные величины одинаково распределены
$$ \frac{1}{2 \left( n - 1 \right) } \sum \limits_{i = 1}^n M \left( X_{i + 1} - X_i \right)^2 =
  \frac{n}{2 \left( n - 1 \right) } \cdot M \left( X_2 - X_1 \right)^2.$$
В данном случае второй момент равен  дисперсии
$$ \frac{n}{2 \left( n - 1 \right) } \cdot M \left( X_2 - X_1 \right)^2 =
  \frac{1}{2} \cdot 2 \sigma^2 =
  \sigma^2.$$
Отсюда следует, что оценка несмещённая.

Проверим состоятельность, то есть $S_n \overset{P}{ \rightarrow } \sigma^2, \, n \to \infty $.

Разобъём $S_n$ на две суммы
$$S_n =
  \frac{1}{2 \left( n - 1 \right) } \cdot
  \left[
    \sum \limits_{even \, i} \left( X_{i + 1} - X_i \right)^2 +
    \sum \limits_{odd \, i} \left( X_{i + 1} - X_i \right)^2
  \right].$$
В каждой из сумм слагаемые независимы
\begin{equation*}
  \begin{split}
    \frac{1}{2 \left( n - 1 \right) } \cdot
    \left[
      \sum \limits_{even \, i} \left( X_{i + 1} - X_i \right)^2 +
      \sum \limits_{odd \, i} \left( X_{i + 1} - X_i \right)^2
    \right] = \\
    = \frac{1}{2 \left( n - 1 \right) } \cdot
    \left[
      \frac{m}{m} \sum \limits_{even \, i} \left( X_{i + 1} - X_i \right)^2 +
      \frac{n - 1 - m}{n - 1 - m} \sum \limits_{odd \, i} \left( X_{i + 1} - X_i \right)^2
    \right].
  \end{split}
\end{equation*}
По закону больших чисел
\begin{equation*}
  \begin{split}
    \frac{1}{2 \left( n - 1 \right) } \cdot
    \left[
      \frac{m}{m} \sum \limits_{even \, i} \left( X_{i + 1} - X_i \right)^2 +
      \frac{n - 1 - m}{n - 1 - m} \sum \limits_{odd \, i} \left( X_{i + 1} - X_i \right)^2
    \right] \overset{P}{ \rightarrow } \\
    \overset{P}{ \rightarrow } \frac{1}{2} \cdot M \left( X_2 - X_1 \right)^2 =
    \frac{1}{2} \cdot D \left( X_2 - X_1 \right) =
    \sigma^2, \,
    n \to \infty.
  \end{split}
\end{equation*}
Отсюда следует, что оценка состоятельная.

\subsubsection*{2.8}

\textit{Задание.}
Пусть $X_1, \dotsc, X_n$ --- выборка из показательного распредения с параметром $ \alpha > 1$.
Для какого параметра $ \theta = \theta \left( \alpha \right) $ статистика
$$ \hat{ \theta_n} =
  e^{ \overline{X}}$$
является состоятельной оценкой?
Является ли $ \hat{ \theta_n}$ сильно состоятельной оценкой того же параметра?
Является ли $ \hat{ \theta_n}$ несмещённой оценкой того же параметра?
Ассимптотически немещённой?

\textit{Решение.}
$$ \overline{X} =
  \frac{1}{n} \sum \limits_{i = 1}^n X_i.$$
По закону больших чисел
$$ \frac{1}{n} \sum \limits_{i = 1}^n X_i \overset{P}{ \to } MX_1, \,
  n \to \infty.$$

Случайные величины в выборке имеют показательное распределение
$$MX_1 =
  \frac{1}{ \alpha },$$
значит,
$$ \overline{X} \overset{P}{ \to } \frac{1}{ \alpha }, \,
  n \to \infty.$$

Применяем непрерывную функцию $e^x$.
Получаем $e^{ \overline{X}} \overset{P}{ \to } e^{ \frac{1}{ \alpha }}, \, n \to \infty $.

Проверяем, является ли оценка $e^{ \overline{X}}$ несмещённой к параметру $e^{ \frac{1}{ \alpha }}$,
то есть выполняется ли $Me^{ \overline{X}} = e^{ \frac{1}{ \alpha }}$.

Вычисляем $Me^{ \overline{X}} = Me^{ \frac{1}{n} \sum \limits_{i = 1}^n X_i}$.
Случайные величины независимы и одинаково распределены,
поэтому $Me^{ \frac{1}{n} \sum \limits_{i = 1}^n X_i} = \left( Me^{ \frac{X_1}{n}} \right)^n$.
По определению характеристической функции $ \varphi_{X_1} = Me^{itX_1}$ получаем
$$ \left( Me^{ \frac{X_1}{n}} \right)^n =
  \left[ \varphi_{X_1} \left( \frac{1}{in} \right) \right]^n.$$
Характеристическая функция показательного распределения
$$ \varphi_{X_1} \left(t \right) =
  Me^{itX_1} =
  \frac{ \alpha }{ \alpha - it}.$$
Подставляем
$$ \left[ \varphi_{X_1} \left( \frac{1}{in} \right) \right]^n =
  \left( \frac{ \alpha }{ \alpha - \frac{1}{n}} \right)^n.$$
Прибавим и отнимем в числителе $1 / n$ и поделим числитель на знаменатель
$$ \left( \frac{ \alpha }{ \alpha - \frac{1}{n}} \right)^n =
  \left( \frac{ \alpha + \frac{1}{n} - \frac{1}{n}}{ \alpha - \frac{1}{n}} \right)^n =
  \left[ 1 + \frac{1}{n \left( \alpha - \frac{1}{n} \right)} \right]^n =
  e^{ \frac{1}{ \alpha }}.$$
Значит, оценка смещённая, но несмещённая ассимптотически.

\subsubsection*{2.9}

\textit{Задание.}
Пусть $X_1, \dotsc, X_n$ --- выборка из геометрического распределения с параметром $p$.
Найдите состоятельные оценки для параметров
$$p, \,
  p^2, \,
  ln p, \,
  p \sin \left( 1 - p \right), \,
  pe^{ \frac{q^2}{2}}.$$

\textit{Решение.} Всё это --- непрерывные функции от $p$.
Применение непрерывной функции не нарушает сходимости по вероятности.

Если параметр каким-то образом связан со средним, но нужно пробовать выборочные моменты.

Сформируем выборочное среднее и применим закон больших чисел
$$ \overline{X} =
  \frac{1}{n} \sum \limits_{i = 1}^n X_i \overset{P}{ \to }
  MX_1.$$
Для геометрического распределения
$$MX_1 =
  \frac{1 - p}{p}.$$

Прибавим единицу слева и справа
$$1 + \overline{X} \overset{P}{ \to }
  1 + \frac{1 - p}{p} =
  1 + \frac{1}{p} - 1 =
  \frac{1}{p}.$$

Функция
$$f \left( x \right) =
  \frac{1}{x}$$
--- это непрерывная функция, значит, можем применить эту функцию слева и справа,
и сходимость сохранится
$$ \frac{1}{1 + \overline{X}} \overset{P}{ \to }
  p.$$

Состоятельной оценкой для параметра $p$ будет
$$ \hat{p} =
  \frac{1}{1 + \overline{X}}.$$

Состоятельной оценкой лдя $p^2$ будет $ \hat{p^2} = \hat{p}^2 \overset{P}{ \to } p^2$,
потому что $f \left( x \right) = x^2$ --- это непрерывная функция.

Логарифм --- это непрерывная функция
$$ \hat{ln p} =
  ln \hat{p} =
  ln \frac{1}{1 + \overline{X}} \overset{P}{ \to }
  ln p,$$
поскольку
$$ \frac{1}{1 + \overline{X}} \overset{P}{ \to }
  p.$$

\subsubsection*{2.12}

\textit{Задание.}
Пусть $X_1, \dotsc, X_n$ --- выборка из распределния Пуассона с параметром $ \lambda > 0$.
Докажите, что не существует несмещённой оценки для параметра $1 / \lambda $.

\textit{Решение.} Будем действовать от противного.

Допустим, что такая оценка существует, то есть существует $ \hat{ \theta }$ такое, что
$$M \hat{ \theta } =
  \frac{1}{ \lambda }.$$
Это условие несмещённости.

$ \hat{ \theta }$ --- функция от выборки,
то есть $ \hat{ \theta } = f \left( X_1, \dotsc, X_n \right) $.

Распределение Пуассона дискретное
\begin{equation*}
  \begin{split}
    M \hat{ \theta } =
    Mf \left( X_1, \dotsc, X_n \right) = \\
    = \sum \limits_{k_1 = 0}^{ \infty } \dotsc \sum \limits_{k_n = 0}^{ \infty }
      f \left( X_1, \dotsc, X_n \right) P \left( X_1 = k_1, X_2 = k_2, \dotsc, X_n = k_n \right).
  \end{split}
\end{equation*}
Воспользуемся независимостью
\begin{equation*}
  \begin{split}
    \sum \limits_{k_1 = 0}^{ \infty } \dotsc \sum \limits_{k_n = 0}^{ \infty }
      f \left( X_1, \dotsc, X_n \right) \cdot
      P \left( X_1 = k_1, X_2 = k_2, \dotsc, X_n = k_n \right) = \\
    = \sum \limits_{k_1 = 0}^{ \infty } \dotsc \sum \limits_{k_n = 0}^{ \infty }
      f \left( X_1, \dotsc, X_n \right) \cdot
      P \left( X_1 = k_1 \right) \cdot \dotsc \cdot P \left( X_n = k_n \right).
  \end{split}
\end{equation*}
Подставим в явном виде вероятности
\begin{equation*}
  \begin{split}
    \sum \limits_{k_1 = 0}^{ \infty } \dotsc \sum \limits_{k_n = 0}^{ \infty }
      f \left( X_1, \dotsc, X_n \right) \cdot
      P \left( X_1 = k_1 \right) \cdot \dotsc \cdot P \left( X_n = k_n \right) = \\
    = \sum \limits_{k_1 = 0}^{ \infty } \dotsc \sum \limits_{k_n = 0}^{ \infty }
      f \left( X_1, \dotsc, X_n \right) \cdot
      \frac{ \lambda^{ \sum \limits_{i = 1}^n k_i}}{ \prod \limits_{i = 1}^n k_i!} \cdot
      e^{- \lambda n}.
  \end{split}
\end{equation*}
Обозначим
$$ \sum k_i =
  k$$
и получим
$$e^{- \lambda n} \cdot
  \sum \limits_{k = 0}^{ \infty } \lambda^k \cdot
  \sum \limits_{k_i: \, k_1 + k_2 + \dotsc + k_n = k}
    \frac{f \left( k_1, \dotsc, k_n \right) }{k_1! \dotsc k_n!} =
  e^{- \lambda n} \sum \limits_{k = 0}^{ \infty } \lambda^k c_k,$$
где $c_k$ --- число.

Допустим, что эта величина равна $1 / \lambda $.

Посмотрим, возможно ли это
$$e^{- \lambda n} \sum \limits_{k = 0}^{ \infty } \lambda^k c_k =
  \frac{1}{ \lambda }.$$

Запишем так, чтобы с одной стороны было $e^{ \lambda n}$, а всё остальное перенесём
$$e^{ \lambda n} =
  \sum \limits_{k = 0}^{ \infty } \lambda^{k + 1} c_k.$$

Для экспоненты существует одно развитие в ряд.
Из этого следует, что последнее равенство невозможно
(развитие в ряд начинается со степени $ \lambda $, равной единице).

\addcontentsline{toc}{section}{Домашнее задание}
\section*{Домашнее задание}

\subsubsection*{2.17}

\textit{Задание.}
Пусть $X_1, \dotsc, X_n$ ---
выборка из равномерного распределения на отрезке $ \left[ 0, \theta \right] $.
Проверьте несмещённость,
состоятельность и найдите среднеквадратическое отклонение следующих оценок параметра $ \theta $:
\begin{enumerate}[label=\alph*)]
  \item $X_{ \left( 1 \right) } + X_{ \left( n \right) }$;
  \item $ \left( n + 1 \right) X_{ \left( 1 \right) }$.
\end{enumerate}

\textit{Решение.} $ \theta $ --- максимальное наблюдение.
\begin{enumerate}[label=\alph*)]
  \item Проверим,
  выполняется ли
  $X_{ \left( 1 \right) } + X_{ \left( n \right) } \overset{P}{ \to } \theta, \,
    n \to \infty $.

  По определению сходимости по вероятности
  $ \forall \varepsilon > 0 \,
    P \left( X_{ \left( 1 \right) } > \varepsilon \right) = \\
    = P \left( X_1 > \varepsilon, X_2 > \varepsilon, \dotsc, X_n > \varepsilon \right) =
    \left[ P \left( X_1 > \varepsilon \right) \right]^n.$
  Подставим значение вероятности из геометрического эксперимента
  $$ \left[ P \left( X_1 > \varepsilon \right) \right]^n =
    \left( \frac{ \theta - \varepsilon }{ \theta } \right)^n \to 0, \,
    n \to \infty.$$

  Значит, $X_{ \left( 1 \right) } \overset{P}{ \to } 0, \, n \to \infty $.

  Остаётся проверить,
  выполняется ли $X_{ \left( n \right) } \overset{P}{ \to } \theta, \, n \to \infty $.

  По определению сходимости по вероятности
  $$P \left\{ \left| X_{ \left( n \right) } - \theta \right| > \varepsilon \right\} \to 0, \,
    n \to \infty.$$

  Раскроем модуль
  \begin{equation*}
    \begin{split}
      P \left\{ \theta - X_{ \left( n \right) } > \varepsilon \right\} =
      P \left\{ X_{ \left( n \right) } - \theta < - \varepsilon \right\} =
      P \left\{ X_{ \left( n \right) } < \theta - \varepsilon \right\} = \\
      = P \left(
        X_1 < \theta - \varepsilon, X_2 < \theta - \varepsilon, \dotsc, X_n < \theta - \varepsilon
      \right) =
      \left[ P \left( X_1 < \theta - \varepsilon \right) \right]^n.
    \end{split}
  \end{equation*}
  Подставим значение вероятности из геометрического эксперимента
  $$ \left[ P \left( X_1 < \theta - \varepsilon \right) \right]^n =
    \left( \frac{ \theta - \varepsilon }{ \theta } \right)^n =
    \left( 1 - \frac{ \varepsilon }{ \theta } \right)^n \to 0, \,
    n \to \infty.$$

  Число в скобках строго меньше единицы, так как $0 \leq \theta \leq 1, \, \varepsilon > 0$.

  Отсюда следует, что оценка состоятельная.

  Проверим несмещённость оценки.
  Проверим, выполняется ли
  $$M \left( X_{ \left( 1 \right) } + X_{ \left( n \right) } \right) =
    \theta.$$

  Из задачи 1.21
  $$MX_{ \left( n \right) } =
    \frac{ \theta n}{n + 1}$$
  и
  $$MX_{ \left( 1 \right) } =
    \frac{ \theta }{n + 1}.$$
  Из свойства линейности математического ожидания
  $$M \left( X_{ \left( 1 \right) } + X_{ \left( n \right) } \right) =
    MX_{ \left( 1 \right) } + MX_{ \left( n \right) } =
    \frac{ \theta n}{n + 1} + \frac{ \theta }{n + 1} =
    \frac{ \theta \left( n + 1 \right) }{n + 1} =
    \theta.$$

  Отсюда следует, что оценка несмещённая.

  Формула среднеквадратического отклонения имеет вид $ \sigma = \sqrt{D \xi }$.

  Найдём дисперсию оценки
  \begin{equation*}
    \begin{split}
      D \left( X_{ \left( 1 \right) } + X_{ \left( n \right) } \right) =
      M \left( X_{ \left( 1 \right) } + X_{ \left( n \right) } \right)^2 -
      \left[ M \left(  X_{ \left( 1 \right) } + X_{ \left( n \right) } \right) \right]^2 = \\
      = M \left(
        X_{ \left( 1 \right) }^2 +
        X_{ \left( n \right) }^2 +
        2X_{ \left( 1 \right) } X_{ \left( n \right) }
      \right) -
      \left( MX_{ \left( 1 \right) } + MX_{ \left( n \right) } \right)^2 = \\
      = MX_{ \left( 1 \right) }^2 +
      MX_{ \left( n \right) }^2 +
      2M \left( X_{ \left( 1 \right) } X_{ \left( n \right) } \right) -
      \left( MX_{ \left( 1 \right) } \right)^2 -
      2MX_{ \left( 1 \right) } MX_{ \left( n \right) } - \\
      - \left( MX_{ \left( n \right) } \right)^2 =
      DX_{ \left( 1 \right) } +
      DX_{ \left( n \right) } +
      2cov \left( X_{ \left( 1 \right) }, X_{ \left( n \right) } \right).
    \end{split}
  \end{equation*}
  Возьмём необходимые значения из задачи 1.21, а именно
  $$DX_{ \left( 1 \right) } =
    \frac{ \theta^2 n}{ \left( n + 1 \right)^2 \left( n + 2 \right) } =
    DX_{ \left( n \right) }, \,
    cov \left( X_{ \left( 1 \right) }, X_{ \left( n \right) } \right) =
    \frac{ \theta^2}{ \left( n + 1 \right)^2 \left( n + 2 \right) }.$$
  Подставляя в найденной выражение, получаем
  \begin{equation*}
    \begin{split}
      DX_{ \left( 1 \right) } +
      DX_{ \left( n \right) } +
      2cov \left( X_{ \left( 1 \right) }, X_{ \left( n \right) } \right) = \\
      = \frac{ \theta^2 n}{ \left( n + 1 \right)^2 \left( n + 2 \right) } +
      \frac{ \theta^2 n}{ \left( n + 1 \right)^2 \left( n + 2 \right) } +
      \frac{2 \theta^2}{ \left( n + 1 \right)^2 \left( n + 2 \right) } = \\
      = \frac{2 \theta^2 n + 2 \theta }{ \left( n + 1 \right)^2 \left( n + 2 \right) } =
      \frac{2 \theta^2 \left( n + 1 \right) }{ \left( n + 1 \right)^2 \left( n + 2 \right) } =
      \frac{2 \theta^2}{ \left( n + 1 \right) \left( n + 2 \right) }.
    \end{split}
  \end{equation*}

  Извлекая корень, получаем
  $$ \sigma =
    \sqrt{ \frac{2 \theta^2}{ \left( n + 1 \right) \left( n + 2 \right) }} =
    \theta \sqrt{ \frac{2}{ \left( n + 1 \right) \left( n + 2 \right) }};$$
  \item проверим,
  выполняется ли
  $ \left( n + 1 \right) X_{ \left( 1 \right) } \overset{P}{ \to } \theta, \,
    n \to \infty $.

  По определению сходимости по вероятности
  $$ \forall \varepsilon > 0 \,
    P \left(
      \left| \left( n + 1 \right) X_{ \left( 1 \right) } - \theta \right| > \varepsilon
    \right) \to 0, \,
    n \to \infty.$$

  Перейдём к противоположному событию
  $$P \left(
    \left| \left( n + 1 \right) X_{ \left( 1 \right) } - \theta \right| > \varepsilon
  \right) =
  1 -
  P \left\{
    \left| \left( n + 1 \right) X_{ \left( 1 \right) } - \theta \right| \leq \varepsilon
  \right\}.$$
  Раскроем модуль
  \begin{equation*}
    \begin{split}
      1 - P \left\{
        \left| \left( n + 1 \right) X_{ \left( 1 \right) } - \theta \right| \leq \varepsilon
      \right\} = \\
      = 1 - P \left\{
        - \varepsilon \leq \left( n + 1 \right) X_{ \left( 1 \right) } - \theta \leq \varepsilon
      \right\} = \\
      = 1 - P \left\{
        \frac{- \varepsilon + \theta }{n + 1} \leq
        X_{ \left( 1 \right) } \leq
        \frac{ \varepsilon + \theta }{n + 1}
      \right\} = \\
      = 1 + P \left\{ X_{ \left( 1 \right) } \leq \frac{ \theta - \varepsilon }{n + 1} \right\} -
      P \left\{ X_{ \left( 1 \right) } < \frac{ \varepsilon + \theta }{n + 1} \right\} = \\
      = 1 + 1 - P \left(
        X_1 > \frac{ \theta - \varepsilon }{n + 1},
        \dotsc,
        X_n > \frac{ \theta - \varepsilon }{n + 1}
      \right) -
      1 + P \left( X_{ \left( 1 \right) } > \frac{ \varepsilon + \theta }{n + 1} \right) = \\
      = 1 - \left[ P \left( X_1 > \frac{ \theta - \varepsilon }{n + 1} \right) \right]^n +
      \left[ P \left( X_1 > \frac{ \theta + \varepsilon}{n + 1} \right) \right]^n.
    \end{split}
  \end{equation*}
  Подставим значения вероятностей из геометрического эксперимента
  \begin{equation*}
    \begin{split}
      1 - \left[ P \left( X_1 > \frac{ \theta - \varepsilon }{n + 1} \right) \right]^n +
      \left[ P \left( X_1 > \frac{ \theta + \varepsilon}{n + 1} \right) \right]^n = \\
      = 1 - \left( \frac{ \theta - \frac{ \theta - \varepsilon }{n + 1}}{ \theta } \right)^2 +
      \left( \frac{ \theta - \frac{ \varepsilon + \theta }{ n + 1}}{ \theta } \right)^n = \\
      = 1 - \left( 1 - \frac{ \theta - \varepsilon }{ \left( n + 1 \right) \theta } \right)^n +
      \left( 1 - \frac{ \varepsilon + \theta }{ \left( n + 1 \right) \theta } \right)^n \not \to
      0, \, n \to \infty.
    \end{split}
  \end{equation*}

  Отсюда следует, что оценка несостоятельная.

  Проверим несмещённость оценки.
  Проверим, выполняется ли
  $$M \left( n + 1 \right) X_{ \left( 1 \right) } =
    \theta.$$

  Выносим константу из-под знака математического ожидания
  $$ \left( n + 1 \right) MX_{ \left( 1 \right) } =
    \left( n + 1 \right) \cdot \frac{ \theta }{n + 1} =
    \theta.$$

  Отсюда следует, что оценка несмещённая.

  Найдём дисперсию оценки
  $$D \left( n + 1 \right) X_{ \left( 1 \right) } =
    \left( n + 1 \right)^2 DX_{ \left( 1 \right) } =
    \left( n + 1 \right)^2 \cdot \frac{ \theta^2 n}{ \left( n + 1 \right)^2 \left( n + 2 \right) } =
    \frac{ \theta^2 n}{n + 2},$$
  откуда среднеквадратическое отклонение равно
  $$ \sigma =
    \sqrt{D \left( n + 1 \right) X_{ \left( 1 \right) }} =
    \sqrt{ \frac{ \theta^2 n}{n + 2}} =
    \theta \sqrt{ \frac{n}{n + 2}}.$$
\end{enumerate}

\subsubsection*{2.18}

\textit{Задание.}
Пусть $X_1, \dotsc, X_n$ ---
выборка из показательного распределения с параметром $1 / \sqrt{ \alpha }$.
Выясните,
является ли статистика $ \hat{ \alpha_n} = \left( \overline{X} \right)^2$
несмещённой оценкой параметра $ \alpha $.
Является ли эта оценка состоятельной?

\textit{Решение.} Нужно проверить, выполняется ли $M \left( \overline{X} \right)^2 = \alpha $.

Запишем, что означает выборочное среднее
$$M \left( \overline{X} \right)^2 =
  M \left( \frac{1}{n} \sum \limits_{i = 1}^n X_i \right)^2 =
  M \left( \frac{1}{n^2} \left( \sum \limits_{i = 1}^n X_i \right)^2 \right) =
  \frac{1}{n^2} \cdot M \left( \sum \limits_{i = 1}^n X_i \right)^2.$$
Квадрат суммы запишем в виде двух сумм.
Получим
$$ \frac{1}{n^2} \cdot
  M \left( \sum \limits_{i = 1}^n X_i^2 + 2 \sum \limits_{k < i}^n X_k X_i \right) =
  \frac{1}{n^2} \cdot M \sum \limits_{i = 1}^n X_i +
  \frac{2}{n^2} M \sum \limits_{k < i}^n X_k X_i.$$
Случайные величины независимы и одинаково распределены, поэтому
$$ \frac{1}{n^2} \cdot M \sum \limits_{i = 1}^n X_i +
  \frac{2}{n^2} M \sum \limits_{k < i}^n X_k X_i =
  \frac{1}{n^2} \sum \limits_{i = 1}^n MX_i^2 + \frac{2}{n^2} \sum \limits_{k < i}^n MX_k MX_i.$$
Для показательного распределения
$$MX_i = \frac{1}{ \lambda } = \frac{1}{ \frac{1}{ \sqrt{ \alpha }}} = \sqrt{ \alpha }, \,
  MX_i^2 = \frac{2}{ \lambda^2} = \frac{2}{ \frac{1}{ \alpha }} = 2 \alpha.$$
Подставляем
\begin{equation*}
  \begin{split}
    \frac{1}{n^2} \sum \limits_{i = 1}^n MX_i^2 + \frac{2}{n^2} \sum \limits_{k < i}^n MX_k MX_i =
    \frac{1}{n^2} \cdot n \cdot 2 \alpha +
    \frac{1}{n^2} \cdot \left( n - 1 \right) n \left( MX_1 \right)^2 = \\
    = \frac{2 \alpha }{n} + \frac{n - 1}{n} \cdot \alpha =
    \frac{2 \alpha }{n} + \alpha - \frac{ \alpha }{n} =
    \frac{ \alpha }{n} + \alpha \to \alpha, \, n \to \infty,
  \end{split}
\end{equation*}
значит, оценка смещённая, но несмещённая ассимптотически.

Проверим, имеет ли место
$ \left( \overline{X} \right)^2 \overset{P}{ \to } \alpha, \,
  n \to \infty $.

По закону больших чисел
$$ \left( \overline{X} \right)^2 =
  \left( \frac{1}{n} \sum \limits_{i = 1}^n X_i \right)^2 \to
  \left( MX_1 \right)^2 =
  \left( \sqrt{ \alpha } \right)^2 =
  \alpha,$$
значит, оценка состоятельная.

\subsubsection*{2.21}

\textit{Задание.}
Пусть $X_1, \dotsc, X_n$ --- выборка из биномиального распределения с параметрами 2 и $p$.
Для какого параметра $ \theta = \theta \left( p \right) $ статистика
$ \hat{ \theta_n} = e^{ \overline{X}}$ будет состоятельной?
Является ли $ \hat{ \theta_n}$ сильно состоятельной оценкой того же параметра?
Является ли $ \hat{ \theta_n}$ несмещённой оценкой того же параметра?
Найдите среднеквадратическое отклонение этой оценки.

\textit{Решение.}
$$ \overline{X} =
  \frac{1}{n} \sum \limits_{i = 1}^n X_i.$$

По закону больших чисел
$$ \frac{1}{n} \sum \limits_{i = 1}^n X_i \overset{P}{ \to } MX_1, \,
  n \to \infty.$$

Случайные величины в выборке имеют биномиальное распределение с математическим ожиданием
$MX_1 = np = 2p$, значит, $ \overline{X} \overset{P}{ \to } 2p, \, n \to \infty $.

Применим непрерывную функцию $e^x$.
Получим $e^{ \overline{X}} \overset{P}{ \to } e^{2p}, \, n \to \infty $.

Проверим, является ли оценка несмещённой к параметру $e^{2p}$,
то есть выполняется ли $Me^{ \overline{X}} = e^{2p}$.

Вычисляем $Me^{ \overline{X}} = Me^{ \frac{1}{n} \sum \limits_{i = 1}^n X_i}$.
Случайные величины независимы и одинаково распределены,
поэтому $Me^{ \frac{1}{n} \sum \limits_{i = 1}^n X_i} = \left( Me^{ \frac{X_1}{n}} \right)^n$.
По определению характеристической функции $ \varphi_{X_1} \left( t \right) = Me^{itX_1}$, получим
$$ \left( Me^{ \frac{X_1}{n}} \right)^n =
  \left[ \varphi_{X_1} \left( \frac{1}{in} \right) \right]^n.$$
Характеристическая функция биномиального распределения
$$ \varphi_{X_1} \left( t \right) =
  Me^{itX_1} =
  \left[ \left( e^{it} - 1 \right) p + 1 \right]^n.$$
Подставим и получим
$$ \left[ \varphi_{X_1} \left( \frac{1}{in} \right) \right]^n =
  \left[ \left( e^{\frac{i}{in}} - 1 \right) p + 1 \right]^n =
  \left[ \left( e^{ \frac{1}{n}} - 1 \right) p + 1 \right]^n \neq
  e^{2p}.$$

Значит, оценка смещённая.

Проверим сильную состоятельность.
По усиленному закону больших чисел
$$ \overline{X} = \frac{1}{n} \sum \limits_{i = 1}^n X_i \overset{a.s.}{ \to } MX_1 = 2p, \,
  n \to \infty.$$
Отсюда следует, что $e^{ \overline{X}} \overset{a.s.}{ \to } e^{2p}, \, n \to \infty $.
Значит, оценка сильно состоятельная.

Найдём дисперсию оценки
$De^{ \overline{X}} =
  Me^{2 \overline{X}} - \left( Me^{ \overline{X}} \right)^2$.

Нашли, что $Me^{ \overline{X}} = \left[ \left( e^{ \frac{1}{n}} - 1 \right) p + 1 \right]^n$.

Вычисляем $Me^{2 \overline{X}} = Me^{ \frac{2}{n} \sum \limits_{i = 1}^n X_i}$.
Из независимости и одинаковой распределенности случайных величин следует,
что $Me^{ \frac{2}{n} \sum \limits_{i = 1}^n X_i} = \left( Me^{ \frac{2X_1}{n}} \right)^n$.
По определению характеристической функции $ \varphi_{X_1} \left( t \right) = Me^{itX_1}$, получаем
$$ \left( Me^{ \frac{2X_1}{n}} \right)^n =
  \left[ \varphi_{X_1} \left( \frac{2}{in} \right) \right]^n.$$
Подставляем характеристическую фунцию биномиального распределения
$$ \left[ \varphi_{X_1} \left( \frac{2}{in} \right) \right]^n =
  \left[ \left( e^{ \frac{2i}{in}} - 1 \right) p + 1 \right]^n =
  \left[ \left( e^{ \frac{2}{n}} - 1 \right) p + 1 \right]^n.$$
Отсюда находим дисперсию оценки
$$De^{ \overline{X}} =
  \left[ \left( e^{ \frac{2}{n}} - 1 \right) p + 1 \right]^n -
  \left[ \left( e^{ \frac{1}{n}} - 1 \right) p + 1 \right]^{2n}.$$
Извлекая корень, получим
$ \sigma =
  \sqrt{
    \left[ \left( e^{ \frac{2}{n}} - 1 \right) p + 1 \right]^n -
    \left[ \left( e^{ \frac{1}{n}} - 1 \right) p + 1 \right]^{2n}
  }.$

\subsubsection*{2.22}

\textit{Задание.} В партии из $n$ изделий оказалось $m$ бракованных.
Неизвестная вероятность $p$ появления бракованного изделия оценивается величиной $m / n$.
Проверьте состоятельность и несмещённость этой оценки.

\textit{Решение.} Выборка имеет распределение Бернулли, то есть
$$X_i =
  \begin{cases}
    1, \qquad p, \\
    0, \qquad 1 - p,
  \end{cases}$$
где событие $ \left\{ X_i = 1 \right\} $ означает, что $i$-тое изделие браковано.

Тогда
$$ \frac{m}{n} =
  \frac{X_1 + \dotsc + X_n}{n},$$
где $ \left( X_1 + \dotsc + X_n \right) $ --- количество бракованных изделий в партии.

Проверяем состоятельность, то есть проверяем, выполняется ли
$$ \frac{m}{n} \overset{p}{ \to } p, \,
  n \to \infty.$$
По закону больших чисел
$$ \frac{m}{n} =
  \frac{X_1 + \dotsc + X_n}{n} \overset{p}{ \to }
  MX_1 =
  p, \,
  n \to \infty,$$
так как это распределение Бернулли.

Это означает, что оценка состоятельная.

Проверим несмещённость оценки, то есть проверим, выполняется ли
$$M \frac{m}{n} =
  p.$$

Ищем математическое ожидание оценки
$$M \frac{m}{n} =
  M \frac{X_1 + \dotsc + X_n}{p} =
  \frac{1}{n} \cdot M \left( X_1 + \dotsc + X_n \right).$$
Пользуемся линейностью математического ожидания
$$ \frac{1}{n} \cdot M \left( X_1 + \dotsc + X_n \right) =
  \frac{1}{n} \cdot nMX_1 =
  p,$$
то есть оценка несмещённая.

\addcontentsline{toc}{chapter}{Занятие 5. Достаточные статистики}
\chapter*{Занятие 5. Достаточные статистики}

\addcontentsline{toc}{section}{Контрольные вопросы и задания}
\section*{Контрольные вопросы и задания}

\subsubsection*{Приведите определение условного распределения, определения достаточной статистики.}

$P \left( \xi \in \Delta \; \middle| \; \mathcal{F'} \right) =
  M \left[ \mathbbm{1}_{ \Delta } \left( \xi \right) \; \middle| \; \mathcal{F'} \right] $.

Статистика $T$ называется достаточной для параметра $ \theta $,
если условное распределение при известном $T$ не зависит от параметра $ \theta $.

\subsubsection*{Сформулируйте теорему про характеризацию достаточной статистики.}

Пусть $x_1, \dotsc, x_n$ ---
выборка из распределения с плотностью
$$p \left( x, \theta \right), \,
  \theta \in \Theta.$$

Статистика $T$ является достаточной тогда и только тогда,
когда функция правдоподобия $L \left( \vec{x}, \theta \right) $ допускает факторизацию,
то есть может быть представлена произведением двух функций следующего вида
$$L \left( \vec{x}, \theta \right) =
  h \left( T, \theta \right) \cdot g \left( \vec{x} \right).$$

\addcontentsline{toc}{section}{Аудиторные задачи}
\section*{Аудиторные задачи}

\subsubsection{5.3}

\textit{Задание.} Пусть $X_1, \dotsc, X_n$ --- выборка из распределения Бернулли с параметром $p$.
Выясните, является ли $ \overline{X}$ достаточной статистикой для параметра $p$.

\textit{Решение.} Записываем условное распределение.

Пусть $k_i = 0$ или $1$.

По определению условной вероятности
\begin{equation*}
  \begin{split}
    P \left\{ \left. X_1 = k_1, X_2 = k_2, \dotsc, X_n = k_n \right| \overline{X} = y \right\} = \\
    = \frac{P \left\{ X_1 = k_1, X_2 = k_2, \dotsc, X_n = k_n, \overline{X} = y \right\} }{P \left\{ \overline{X} = y \right\} }.
  \end{split}
\end{equation*}
Выборочное среднее $ \overline{X}$ выражаем через сумму
\begin{equation*}
  \begin{split}
    \frac{P \left\{ X_1 = k_1, X_2 = k_2, \dotsc, X_n = k_n, \overline{X} = y \right\} }{P \left\{ \overline{X} = y \right\} } = \\
    = \frac{P \left\{ X_1 = k_1, X_2 = k_2, \dotsc, X_n = k_n, \sum \limits_{i = 1}^n X_i = ny \right\} }{P \left\{ \sum \limits_{i = 1}^n X_i = ny \right\} }.
  \end{split}
\end{equation*}
Все $X_i$ --- независимы, следовательно, будет произведение вероятностей событий
\begin{equation*}
  \begin{split}
    \frac{P \left\{ X_1 = k_1, X_2 = k_2, \dotsc, X_n = k_n, \sum \limits_{i = 1}^n X_i = ny \right\} }{P \left\{ \sum \limits_{i = 1}^n X_i = ny \right\} } = \\
    = \frac{ \mathbbm{1} \left\{ \sum \limits_{i = 1}^n k_i = ny \right\} P \left\{ X_1 = k_1 \right\} \cdot \dotsc \cdot P \left\{ X_n = k_n \right\} }{P \left\{ \sum \limits_{i = 1}^n X_i = ny \right\} }.
  \end{split}
\end{equation*}
Сумма распределена по биномиальному распределению с параметрами $n, p$.
Подставляем значения вероятностей
\begin{equation*}
  \begin{split}
    \frac{ \mathbbm{1} \left\{ \sum \limits_{i = 1}^n k_i = ny \right\} P \left\{ X_1 = k_1 \right\} \cdot \dotsc \cdot P \left\{ X_n = k_n \right\} }{P \left\{ \sum \limits_{i = 1}^n X_i = ny \right\} } = \\
    = \frac{ \mathbbm{1} \left\{ \sum \limits_{i = 1}^n k_i = ny \right\} p^{k_1} \left( 1 - p \right)^{1 - k_1} \cdot \dotsc \cdot p^{k_n} \left( 1 - p \right)^{1 - k_n}}{C_n^{ny} p^{ny} \left( 1 - p \right)^{n + ny}}.
  \end{split}
\end{equation*}
Воспользуемся индикатором
\begin{equation*}
  \begin{split}
    \frac{ \mathbbm{1} \left\{ \sum \limits_{i = 1}^n k_i = ny \right\} \cdot p^{k_1} \left( 1 - p \right)^{1 - k_1} \cdot \dotsc \cdot p^{k_n} \left( 1 - p \right)^{1 - k_n}}{C_n^{ny} p^{ny} \left( 1 - p \right)^{n + ny}} = \\
    = \frac{ \mathbbm{1} \left\{ \sum \limits_{i = 1}^n k_i = ny \right\} \cdot p^{ny} \left( 1 - p \right)^{n - ny}}{C_n^{ny} p^{ny} \left( 1 - p \right)^{n - ny}}.
  \end{split}
\end{equation*}
Всё, что связано с $p$, пропадает
$$ \frac{ \mathbbm{1} \left\{ \sum \limits_{i = 1}^n k_i = ny \right\} \cdot p^{ny} \left( 1 - p \right)^{n - ny}}{C_n^{ny} p^{ny} \left( 1 - p \right)^{n - ny}} =
  \frac{ \mathbbm{1} \left\{ \sum \limits_{i = 1}^n k_i = ny \right\} }{C_n^{ny}}.$$

Зависимости от $p$ нет.
Отсюда следует, что $ \overline{X}$ --- достаточная статистика.

\subsubsection{5.4}

\textit{Задание.}
Пусть $X_1, \dotsc, X_n$ --- выборка из распределения Пуассона с параметром $ \lambda $.
Найдите условное распределение выборки при условии
$$X_1 + \dotsc + X_n = k.$$
Выясните, является ли $ \overline{X}$ достаточной статистикой для параметра $ \lambda $.

\textit{Решение.} Это дискретное распределение
\begin{equation*}
  \begin{split}
    P \left\{ X_1 = k_1, \dotsc, X_n = k_n \; \middle| \; \sum \limits_{i = 1}^n X_i = k \right\} = \\
    \frac{P \left\{ X_1 = k_1, \dotsc, X_n = k_n, \sum \limits_{i = 1}^n X_i = k \right\} }{P \left\{ \sum \limits_{i = 1}^n X_i = k \right\} }.
  \end{split}
\end{equation*}
Случайные величины $X_i$ --- независимы
\begin{equation*}
  \begin{split}
    \frac{P \left\{ X_1 = k_1, \dotsc, X_n = k_n, \sum \limits_{i = 1}^n X_i = k \right\} }{P \left\{ \sum \limits_{i = 1}^n X_i = k \right\} } = \\
    \frac{ \mathbbm{1} \left\{ \sum \limits_{i = 1}^n k_i = k \right\} P \left\{ X_1 = k_1 \right\} \cdot \dotsc \cdot P \left\{ X_n = k_n \right\} }{P \left\{ \sum \limits_{i = 1}^n X_i = k \right\} }.
  \end{split}
\end{equation*}
Имеем распределение Пуассона.
Сумма независимых случайных величин имеет распределение Пуассона с параметром $ \lambda n$.
Подставляем значения вероятностей
\begin{equation*}
  \begin{split}
    \frac{ \mathbbm{1} \left\{ \sum \limits_{i = 1}^n k_i = k \right\} P \left\{ X_1 = k_1 \right\} \cdot \dotsc \cdot P \left\{ X_n = k_n \right\} }{P \left\{ \sum \limits_{i = 1}^n X_i = k \right\} } = \\
    = \mathbbm{1} \left\{ \sum \limits_{i = 1}^n k_i = k \right\} \cdot
    \frac{ \frac{ \lambda^{k_1}}{k_1!} \cdot e^{- \lambda } \cdot \dotsc \cdot \frac{ \lambda^{k_n}}{k_n!} \cdot e^{- \lambda }}{ \frac{ \left( \lambda n \right)^k}{k!} \cdot e^{- \lambda }} =
    \frac{ \mathbbm{1} \left\{ \sum \limits_{i = 1}^n k_i = k \right\} k!}{k_1! \dotsc k_n! n^k}.
  \end{split}
\end{equation*}

Видим, что нет зависимости от $ \lambda $, следовательно, выборочное среднее $ \overline{X}$ ---
достаточная статистика для неизвестного $ \lambda $.

\addcontentsline{toc}{section}{Домашнее задание}
\section*{Домашнее задание}

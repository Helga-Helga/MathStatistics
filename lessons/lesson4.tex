\addcontentsline{toc}{chapter}{Занятие 4. Оценка максимального правдоподобия}
\chapter*{Занятие 4. Оценка максимального правдоподобия}

\addcontentsline{toc}{section}{Контрольные вопросы и задания}
\section*{Контрольные вопросы и задания}

\subsubsection*{Как построить оценку максимального правдоподобия?}

Плотность распределения выборки $\left( X_1, \dotsc, X_n \right) $ имеет вид
$$L \left( \vec{X}, \theta \right) =
  \prod \limits_{k = 1}^n p \left( X_k, \theta \right),$$
как плотность вектора с независимыми координатами.
$L \left( \vec{X}, \theta \right) $ --- функция правдоподобия.

Оценка максимаьлного правдоподобия $ \hat{ \theta }$ --- такое значение параметра $ \theta $,
при котором функция правдоподобия достикает своего максимального значения
$ \hat{ \theta } =
  \underset{ \theta }{argmax} L \left( \vec{X}, \theta \right) $.

\subsubsection*{Сформулируйте свойства оценки максимального правдоподобия.}

Оценка максимального правдоподобия, как правило, сильно состоятельная.

\addcontentsline{toc}{section}{Аудиторные задачи}
\section*{Аудиторные задачи}

\addcontentsline{toc}{section}{Домашнее задание}
\section*{Домашнее задание}

\subsubsection*{3.10}

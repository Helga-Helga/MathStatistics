\addcontentsline{toc}{chapter}{Занятие 1. Выборочные характеристики}
\chapter*{Занятие 1. Выборочные характеристики}

\addcontentsline{toc}{section}{Контрольные вопросы и задания}
\section*{Контрольные вопросы и задания}

\subsubsection*{Приведите определение выборки, вариационого ряда, статистики, порядковой статистики,
                эмпирической функции распределения.}

$x_1, \dotsc, x_n$ --- наблюдаемые значения ---
независимые одинаково распределённые случайные величины с неизвестной функцией распределения
$F \left( x \right) $.

Такой набор случайных величин называется выборкой из распределения $F$.

Вариационный ряд --- последовательность $x_{ \left( 1 \right) }, \dotsc, x_{ \left( n \right) }$,
полученная в результате расположения в порядке неубывания исходной последовательности
независимых одинаково распределённых случайных величин $x_1, \dotsc, x_n$.

Статистикой называют функцию $S$ от выборки $X = \left( x_1, x_2, \dotsc, x_n \right) $ такую,
что $S \left( X \right) = S \left( x_1, x_2, \dotsc, x_n \right) $.

Вариационный ряд и его члены являются порядковыми статистиками.

Эмпирической (выборочной) функцией распределения,
построенной по выборке $x_1, \dotsc, x_n$ называется функция
$$F_n \left( x \right) =
  \frac{1}{n} \sum \limits_{k = 1}^n \mathbbm{1}_{x_k \leq x}, \,
  x \in \mathbb{R}.$$

\subsubsection*{Какими свойствами обладает эмпирическая функция распределения?}

Есть множество полной вероятности,
на котором эмпирическая функция распределения аппроксимирует функцию распределения,
то есть почти наверное $F_n \Rightarrow F, \, n \to \infty $.

\subsubsection*{Запишите выражения для выборочного среднего, выборочной диспресии,
                выборочных моментов.}

$$ \frac{1}{n} \sum \limits_{k = 1}^n x_k$$
--- выборочное среднее.

Выборочная дисперсия
$$ \hat{ \sigma^2} =
  \frac{1}{n - 1} \sum \limits_{k = 1}^n \left( x_k - \overline{x} \right)^2.$$

Выборочные моменты в математической статистике ---
это оценка теоретических моментов распределения на основе выборки.

Выборочный момент порядка $k$ --- это случайная величина
$$a_n \left( k \right) =
   \frac{1}{n} \sum \limits_{i = 1}^n x_i^k.$$

\addcontentsline{toc}{section}{Аудиторные задачи}
\section*{Аудиторные задачи}

\addcontentsline{toc}{section}{Домашнее задание}
\section*{Домашнее задание}

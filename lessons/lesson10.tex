\addcontentsline{toc}{chapter}{Занятие 10. Интервальное оценивание параметров.
                              Доверительные интервалы}
\chapter*{Занятие 10. Интервальное оценивание параметров. Доверительные интервалы}

\addcontentsline{toc}{section}{Контрольные вопросы и задания}
\section*{Контрольные вопросы и задания}

\subsubsection*{Приведите определение доверительного интервала с уровнем значимости $ \alpha $,
                ассимпточисеского деверительного интервала с уровени значимости $ \alpha $.}

Имеем семейство распределений $F_{ \theta }$,
зависящее от параметра $ \theta \in \Theta \subset \mathbb{R}$.
Имеем выборку $X_1, \dotsc, X_n$ из распределения $F_{ \theta }$.

Есть две статистики: $T_1$ и $T_2$ такие, что $T_1 \leq T_2$ почти наверное,
то есть это некоторый случайный интервал $ \left[ T_1, T_2 \right] $.
Задаём число $ \alpha \in \left( 0, 1 \right) $, как правило $ \alpha \ll 1$.

Промежуток $ \left[ T_1, T_2 \right] $ ---
это доверительный интервал для $ \theta $ с уровнем доверия $ \alpha $,
если
$ \forall \theta \in \Theta: \,
  P_{ \theta } \left( T_1 \leq \theta \leq T_2 \right) \geq 1 - \alpha $.

Интервал $ \left[ T_1, T_2 \right] $
называется асимптотическим доверительным интервалом для параметра $ \theta $ уровня доверия
$ \alpha $,
если для любого
$$ \theta \in \Theta: \,
  \lim \limits_{n \to \infty } inf \, P_{ \theta } \left( T_1 \leq \theta \leq T_2 \right) \geq
  1 - \alpha.$$

\subsubsection*{Какая статистика называется центральной?}

Функция $G \left( \vec{X}, \theta \right) $ называется центральной статистикой,
если выполняется ряд требований:
\begin{enumerate}
  \item $G \left( \vec{X}, \cdot \right) $ --- функция второго аргумента ---
  строго монотонна и непрерывна (либо строго возрастает, либо строго убывает);
  \item $G \left( \vec{X}, \theta \right) $ имеет известное распределение,
  не зависящее от $ \theta $.
\end{enumerate}

\subsubsection*{Как с помощью центральной статистики построить доверительный интервал с заданным
                уровнем доверия?}

Пусть $G$ --- центральная статистика.
Распределение $G$ --- известно.

Следовательно, по $ \alpha $ можно указать такие числа $a_1 < a_2$, что имеет место следующее:
$P \left\{ a_1 \leq G \left( \vec{X}, \theta \right) \leq a_2 \right\} \geq
  1 - \alpha $.

Считаем, что $G \left( \vec{X}, \cdot \right) $ возрастает.
Тогда это равносильно тому,
что
$$P_{ \theta } \left\{
    G^{-1} \left( \vec{X}, a_1 \right) \leq \theta \leq G^{-1} \left( \vec{X}, a_2 \right)
  \right\}
  \geq 1 - \alpha.$$
Эти границы $G^{-1} \left( \vec{X}, a_1 \right), \, G_2^{-1} \left( \vec{X}, a_2 \right) $ ---
доверительный интервал.

\addcontentsline{toc}{section}{Аудиторные задачи}
\section*{Аудиторные задачи}

\addcontentsline{toc}{section}{Домашнее задание}
\section*{Домашнее задание}

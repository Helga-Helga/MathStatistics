\addcontentsline{toc}{chapter}{Занятие 7. Гауссовые системы}
\chapter*{Занятие 7. Гауссовые системы}

\addcontentsline{toc}{section}{Контрольные вопросы и задания}
\section*{Контрольные вопросы и задания}

\subsubsection*{Приведите определение гауссового случайного вектора,
                ковариационной матрицы гауссового случаайного вектора.}

$ \vec{ \xi } = \left( \xi_1, \dotsc, \xi_n \right) $ --- гауссовый, если
$$ \forall \alpha_1, \dotsc, \alpha_n: \,
  \sum \limits_{k = 1}^n \alpha_k \xi_k$$
--- гауссовская случайная величина.

$A = cov_{ \vec{ \xi } \vec{ \xi }}, \,
  A_{ij} =
  cov \left( \xi_i, \xi_j \right) =
  M \left( \xi_i \xi_j \right) - M \xi_i M \xi_j$.

\subsubsection*{Какими свойствами владеет ковариационная матрица?}

$A_{ii} = D \xi_i \geq 0$ --- на диагонале --- неотрицательные числа.

Матрица симметрична: $A_{ij} = A_{ji}$.

Матрица неотрицательно определена.

\subsubsection*{Как изменяются характеристики гауссового случайного
                вектора при действии на него линейного оператора?}
Пускай $ \vec{ \xi }$ случайный $n$-элементный вектор,
имеющий гауссовское распределение с параметрами $ \vec{a}$ и
$A, \,
  \vec{ \xi } \sim N \left( \vec{a}, A \right), \,
  \vec{b} \in \mathbb{R}^m, \,
  T \in \mathbb{R}^{m \times n}$.
Тогда $T \vec{ \xi } + \vec{b} \sim N \left( T \vec{a} + \vec{b}, TAT^* \right) $.

\subsubsection*{При каких условиях гауссовский случайный вектор имеет плотность распределения?}

Плотность можно записать только в случае, когда $det \, A \neq 0$.

\subsubsection*{Запишите плотность распределения гауссового случайного вектора.}

$$p_{ \vec{ \xi }} \left( \vec{x} \right) =
  \left( \frac{1}{ \sqrt{2 \pi }} \right)^n \cdot \frac{1}{ \sqrt{det \, A}} \cdot
  e^{- \frac{1}{2} \left[ \vec{x} - \vec{a}, A^{-1} \left( \vec{x} - \vec{a} \right) \right] }.$$

\subsubsection*{Сформулируйте теорему про нормальную корреляцию.}

Есть гауссоский вектор $ \vec{ \xi } \circ \vec{ \eta }$ с ненулевой ковариацией
$cov_{ \vec{ \xi } \circ \vec{ \eta }, \vec{ \xi } \circ \vec{ \eta }} \neq 0$.

Определитель ковариационной матрицы вектора $ \eta $ положителен
$$det \, cov_{ \vec{ \eta }, \vec{ \eta }} \geq
  0.$$

Тогда вектор $ \vec{ \xi }$ при условии $ \vec{ \eta }$ ---
гауссовский случайный вектор
$$ \left. \vec{ \xi} \right| \vec{ \eta } \sim
  N \left( \vec{m}, D \right).$$

Параметры $ \vec{m}$ и $D$ имеют следующий вид
$$ \vec{m} =
  M \vec{ \xi } +
  cov_{ \vec{ \xi }, \vec{ \eta }}
    cov_{ \vec{ \eta }, \vec{ \eta }^{-1}} \left( \eta - M \vec{ \eta } \right), \,
  D =
  cov_{ \vec{ \xi }, \vec{ \xi }} -
  cov_{ \vec{ \xi }, \vec{ \eta }} cov_{ \vec{ \eta }, \vec{ \eta }}^{-1}
    cov_{ \vec{ \eta }, \vec{ \xi }}.$$

\addcontentsline{toc}{section}{Аудиторные задачи}
\section*{Аудиторные задачи}

\subsubsection*{7.3}

\textit{Задание.}
Может ли матрица $A$ быть ковариационной матрицей гауссового случайного вектора, если:
\begin{enumerate}[label=\alph*)]
  \item $A =
    \begin{bmatrix}
      1 & 0 & 0 \\
      0 & 0 & 0
    \end{bmatrix};$
  \item $A =
    \begin{bmatrix}
      1 & 2 \\
      3 & 9
    \end{bmatrix};$
  \item $A =
    \begin{bmatrix}
      4 & 2 \\
      2 & -1
    \end{bmatrix};$
  \item $A =
    \begin{bmatrix}
      1 & 2 \\
      2 & 5
    \end{bmatrix};$
  \item $A =
    \begin{bmatrix}
      1 & 2 & 1 \\
      2 & 6 & -1 \\
      1 & -1 & 12
    \end{bmatrix};$
  \item $A =
    \begin{bmatrix}
      1 & 2 & 1 \\
      2 & 5 & 0 \\
      1 & 0 & 10
    \end{bmatrix}.$
\end{enumerate}
Если так, то предъявите такой вектор.

\textit{Решение.}
\begin{enumerate}[label=\alph*)]
  \item Нет (матрица не квадратная);
  \item нет (матрица не симметрична);
  \item нет (на диагонали отрицательные числа);
  \item матрица квадратная, симметричная, неотрицательно определённая, на диагонали ---
  неотрицательные числа;
  \item матрица квадратная, симметричная, на диагонали --- нетрицательные числа.
  Первый минор $M_1 = 1 \geq 0$, второй минор $M_2 = 6 - 4 = 2 \geq 0$, третий минор
  $$det \left(
      \begin{bmatrix}
        1 & 2 & 1 \\
        2 & 6 & -1 \\
        1 & -1 & 12
      \end{bmatrix}
    \right) =
    1 \cdot \left( 72 - 1 \right) - 2 \cdot \left( 24 + 1 \right) + 1 \cdot \left( -2 - 6 \right) =
    13 \geq
    0,$$
  значит, матрица неотрицательно определена,
  то есть может быть ковариационной матрицей гауссового случайного вектора;
  \item матрица квадратная, симметричная, на диагонали --- неотрицательные чиста.
  Первый минор $M_1 = 1 \geq 0$, второй минор $M_2 = 5 - 2 = 3 \geq 0$,
  третий минор
  $$M_3 =
    1 \cdot \left( 50 - 0 \right) - 2 \cdot \left( 20 - 0 \right) + 1 \cdot \left( 0 - 5 \right) =
    50 - 40 - 5 =
    5 \geq
    0,$$
  значит, матрица неотрицательно определена,
  то есть может быть ковариационной матрицей гауссового случайного вектора.
\end{enumerate}

\addcontentsline{toc}{section}{Домашнее задание}
\section*{Домашнее задание}

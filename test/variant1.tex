\addcontentsline{toc}{chapter}{Контрольная работа}
\chapter*{Контрольная работа}

\addcontentsline{toc}{section}{Вариант 1}
\section*{Вариант 1}

\subsubsection*{1}

\textit{Задание.} На таможню ежедневно приходит в среднем 10 грузовиков.
Среднее квадратическое отклонение количества грузовиков равно 3.
Используя центральную предельную теорему, найдите такое $x$, что вероятность того,
что за год через таможню пройдёт больше чем $x$ грузовиков, примерно равна $0.99$.

\textit{Решение.}
Обозначим через $ \xi_n$ количество грузовиков, которые прибыли на таможню в $n$-ый день года.
Тогда $S = \xi_1 + \dotsc + \xi_{365}$ является количеством грузовиков,
которые прибыли на таможню за год.
Чтобы найти такое $x$, что $P \left( S > x \right) \approx 0.99$,
воспользуемся аппроксимацией нормальным распределением:
$$P \left( S > x \right) =
  P \left( \frac{S - MS}{ \sqrt{DS}} > \frac{x - MS}{ \sqrt{DS}} \right) \approx
  \Phi_t \left( \frac{x - MS}{ \sqrt{DS}} \right),$$
где $ \Phi_t$ --- функция распределения стандартного нормального распределения.
Значит $x$ находим из условия
$$ \Phi_t \left( \frac{MS - x}{ \sqrt{DS}} \right) \approx
  0.01.$$
Воспользовавшись таблицей значений функции $ \Phi_t$, находим, что
$$ \frac{MS - x}{ \sqrt{DS}} \approx
  2.32.$$
По условию $MS = 365 \cdot M \xi_1 = 3650, \, \sqrt{DS} = \sqrt{365 \cdot D \xi_1} = 3 \sqrt{365}$
и значит $-x + 3650 \approx 2.32 \cdot 3 \sqrt{365} \approx 132.97$,
откуда $x = -132.97 + 3650 \approx 2517$.

\subsubsection*{2}

\textit{Задание.}
Для произвольного действительного числа $y$ вычислите дисперсию $DF_n^* \left( y \right) $,
где $F_n^*$ --- эмпирическая функция распределения, построенная по выборке из распределения $F$.

\textit{Решение.}
$$F_n^* \left( y \right) =
  \frac{1}{n} \sum \limits_{k = 1}^n \mathbbm{1} \left\{ X_k \leq y \right\}, \,
  y \in \mathbb{R}.$$

Ищем дисперсию
$$DF_n^* \left( y \right) =
  D \left( \frac{1}{n} \sum \limits_{k = 1}^n \mathbbm{1} \left\{ X_k \leq y \right\} \right) =
  \frac{1}{n^2} \cdot D \sum \limits_{k = 1}^n \mathbbm{1} \left\{ X_k \leq y \right\}.$$
Вносим дисперсию под знак суммы, пользуясь независимостью случайных величин, а затем используем то,
что они имеют одинаковое распределение
$$ \frac{1}{n^2} \cdot D \sum \limits_{k = 1}^n \mathbbm{1} \left\{ X_k \leq y \right\} =
  \frac{1}{n^2} \sum \limits_{k = 1}^n D \mathbbm{1} \left\{ X_k \leq y \right\} =
  \frac{1}{n^2} \cdot nD \mathbbm{1} \left\{ X_1 \leq y \right\}.$$
Сокращаем константы и пользуемся свойством дисперсии
$$ \frac{1}{n^2} \cdot nD \mathbbm{1} \left\{ X_1 \leq y \right\} =
  \frac{1}{n} \cdot D \mathbbm{1} \left\{ X_1 \leq y \right\} =
  \frac{1}{n} \left[ F \left( y \right) - F^2 \left( y \right) \right] =
  \frac{1}{n} \cdot F \left( y \right) \left[ 1 - F \left( y \right) \right],$$
так как
$M \mathbbm{1} \left\{ X_1 \leq y \right\} =
  1 \cdot P \left( X_1 \leq y \right) + 0 \cdot P \left( X_1 > y \right) =
  P \left( X_1 \leq y \right) =
  F \left( y \right) $,
и $M \left( \mathbbm{1} \left\{ X_1 \leq y \right\} \right)^2 = F \left( y \right) $.

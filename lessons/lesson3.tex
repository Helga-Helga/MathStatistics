\addcontentsline{toc}{chapter}{Занятие 3. Метод моментов построения оценок}
\chapter*{Занятие 3. Метод моментов построения оценок}

\addcontentsline{toc}{section}{Контрольные вопросы и задания}
\section*{Контрольные вопросы и задания}

\subsubsection*{Преведиты определение оценки: несмещённой, ассимптотически несмещённой,
                состоятельной, сильно состоятельной, оптимальной.}

Оценка $ \hat{ \theta}$  несмещённая,
если $ \forall \theta \in \Theta: \, M_{ \theta } \hat{ \theta } = \theta $.

Асимптотически несмещенная оценка --- такая оценка,
математическое ожидание которой совпадает с оцениваемым параметром при $n \to \infty $.

Оценка $ \hat{ \theta }$ называется состоятельной,
если стремится к истинному значению $ \theta $ по вероятности
$ \hat{ \theta } \overset{P}{ \rightarrow } \theta, \,
  n \to \infty $.

Оценка $ \hat{ \theta }$ называется сильно состоятельной,
если стремится к истинному значению $ \theta $ почти наверное
$ \hat{ \theta } \overset{a.s.}{ \rightarrow } \theta, \,
  n \to \infty $.

Несмещённая оценка $ \hat{ \theta } \in K$
называется оптимальной в классе квадратично интегрируемых оценок $K$,
если для всякой другой несмещённой оценки
$ \tilde{ \theta } \in \Theta \,
  \forall \theta \in \Theta: \,
  D_{ \theta } \hat{ \theta } \leq D_{ \theta } \tilde{ \theta }$
или же
$ \forall \theta \in \Theta, \,
  M_{ \theta } \left( \hat{ \theta } - \theta \right)^ \leq
  M_{ \theta } \left( \tilde{ \theta } - \theta \right)^2$.

\subsubsection*{Что называется среднеквадратическим отклонением оценки?}

$M_{ \theta } \left( \hat{ \theta } - \theta \right) $ --- среднеквадратическое оклонение.

\subsubsection*{Сформулируйте утверждение про поведение выборочных моментов.}

Выборочный начальный момент $M_k \, k$-го порядка стремится к начальному моменту $ \nu_k$
случайной величины $X$, то есть
$$ \lim \limits_{n \to \infty } P \left( \left| M_k - \nu_k \right| \geq \varepsilon \right) =
  0,$$
для любого сколь угодно малого $ \varepsilon > 0$,
если моменты $ \nu_{2k}$ и $ \nu_k$ случайной величины $X$ существуют и конечны.

\subsubsection*{Сформулируйте основную идею метода моментов построения оценки
                неизвестного параметра.}

$x_1, \dotsc, x_n$ ---
выборка из распределения $F_{ \theta }$ и
$ \theta: \,
  M_{ \theta } f \left( x_1 \right) = g \left( \theta \right) $.

Вычисляем математическое ожидание, считая, что $x_1$ имеет распределение с параметром $ \theta $,
иными словами,
$$M_{ \theta } f \left( x \right) = \int \limits_{ \mathbb{R}} xdF_{ \theta } \left( x \right) , \,
  f, g \in C \left( \mathbb{R} \right), \,
  g$$
--- строго монотонная.
Тогда в силу усиленного закона больших чисел
$$ \theta \approx
  g^{-1} \left( \frac{1}{n} \sum \limits_{k = 1}^n f \left( x_k \right) \right) \overset{a.s.}
  \theta, \,
  n \to \infty.$$

Если существуют непрерывные $f$ и $g, \, g$ ---
обратима и $g \left( \theta \right) = M_{ \theta } f \left( x \right) $,
то в качестве оценки можно выбрать
$$ \hat{ \theta } =
  g^{-1} \left( \int \limits_{ \mathbb{R}} fdF_n \right).$$

\addcontentsline{toc}{section}{Аудиторные задачи}
\section*{Аудиторные задачи}

\addcontentsline{toc}{section}{Домашнее задание}
\section*{Домашнее задание}

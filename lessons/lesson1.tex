\addcontentsline{toc}{chapter}{Занятие 1. Выборочные характеристики}
\chapter*{Занятие 1. Выборочные характеристики}

\addcontentsline{toc}{section}{Контрольные вопросы и задания}
\section*{Контрольные вопросы и задания}

\subsubsection*{Приведите определение выборки, вариационого ряда, статистики, порядковой статистики,
                эмпирической функции распределения.}

$x_1, \dotsc, x_n$ --- наблюдаемые значения ---
независимые одинаково распределённые случайные величины с неизвестной функцией распределения
$F \left( x \right) $.

Такой набор случайных величин называется выборкой из распределения $F$.

Вариационный ряд --- последовательность $x_{ \left( 1 \right) }, \dotsc, x_{ \left( n \right) }$,
полученная в результате расположения в порядке неубывания исходной последовательности
независимых одинаково распределённых случайных величин $x_1, \dotsc, x_n$.

Статистикой называют функцию $S$ от выборки $X = \left( x_1, x_2, \dotsc, x_n \right) $ такую,
что $S \left( X \right) = S \left( x_1, x_2, \dotsc, x_n \right) $.

Вариационный ряд и его члены являются порядковыми статистиками.

Эмпирической (выборочной) функцией распределения,
построенной по выборке $x_1, \dotsc, x_n$ называется функция
$$F_n \left( x \right) =
  \frac{1}{n} \sum \limits_{k = 1}^n \mathbbm{1}_{x_k \leq x}, \,
  x \in \mathbb{R}.$$

\subsubsection*{Какими свойствами обладает эмпирическая функция распределения?}

Есть множество полной вероятности,
на котором эмпирическая функция распределения аппроксимирует функцию распределения,
то есть почти наверное $F_n \Rightarrow F, \, n \to \infty $.

\subsubsection*{Запишите выражения для выборочного среднего, выборочной диспресии,
                выборочных моментов.}

$$ \frac{1}{n} \sum \limits_{k = 1}^n x_k$$
--- выборочное среднее.

Выборочная дисперсия
$$ \hat{ \sigma^2} =
  \frac{1}{n - 1} \sum \limits_{k = 1}^n \left( x_k - \overline{x} \right)^2.$$

Выборочные моменты в математической статистике ---
это оценка теоретических моментов распределения на основе выборки.

Выборочный момент порядка $k$ --- это случайная величина
$$a_n \left( k \right) =
   \frac{1}{n} \sum \limits_{i = 1}^n x_i^k.$$

\addcontentsline{toc}{section}{Аудиторные задачи}
\section*{Аудиторные задачи}

\subsubsection{1.4}

\textit{Задание.}
Пусть $X_1, \dotsc, X_n$ ---
выборка из равномерного распределения на отрезке $ \left[0, \theta \right] $
с неизвестным параметром $ \theta $.
Какие из приведённых ниже функций являются статистиками?
\begin{enumerate}[label=\alph*)]
  \item $ \overline{X}$;
  \item $5X_{ \left( n \right) }$;
  \item $ \theta / 2$;
  \item $X_1 / \theta $;
  \item $X_{ \left( 1 \right) } + X_1 + X_n$.
\end{enumerate}

\textit{Решение.}
\begin{enumerate}[label=\alph*)]
  \item Да;
  \item да;
  \item нет, так как не функция от выборки;
  \item функция не только от выборки (зависит от неизвестного параметра).
  Отсюда следует, что это не статистика;
  \item да.
\end{enumerate}

\subsubsection{1.5}

\textit{Задание.}
Пусть $X_1, \dotsc, X_n$ --- выборка из распределения Пуассона с параметром $ \lambda $.
Вычислите математическое ожидание и дисперсию статистики
$$ \overline{X} =
  \frac{1}{n} \sum \limits_{i = 1}^n X_i.$$
Выясните, имеет ли статистика $ \overline{X}$ распределение Пуассона.

\textit{Решение.} Все $X_i$ одинаково распределены.
Отсюда следует, что все математические ожидания одинаковы
$$M \overline{X} =
  \frac{1}{n} \sum \limits_{i = 1}^n MX_i =
  \frac{1}{n} \cdot nMX_1 =
  MX_1 =
  \lambda.$$

Для всякой выборки справедливо $M \overline{X} = MX_1$.

Из независимости $X_i$ получаем
$$D \overline{X} =
  \frac{1}{n^2} \sum \limits_{i = 1}^n DX_i.$$

Так как $X_i$ одинаково распределены, то все дисперсии одинаковы
$$D \overline{X} =
  \frac{DX_1}{n} =
  \frac{ \lambda }{n}.$$

Математическое ожидание и дисперсия для распределения Пуассона совпадают.
Отсюда следует, что эта случайная величина не имеет распределения Пуассона.

$ \overline{X}$ не обязательно буде принимать целые значения.

\subsubsection{1.6}

\textit{Задание.} Вычислите математическое ожидание статистик:
\begin{enumerate}[label=\alph*)]
  \item $S^2 = \overline{X^2} - \left( \overline{X} \right)^2$;
  \item $S_0^2 =
          1 / \left( n - 1 \right) \cdot
          \sum \limits_{i = 1}^n \left( X_i - \overline{X} \right)^2$.
\end{enumerate}

\textit{Решение.}
\begin{enumerate}[label=\alph*)]
  \item Распишем каждую из величин
  $$S^2 =
    \overline{X^2} - \left( \overline{X} \right)^2 =
    \frac{1}{n} \sum \limits_{i = 1}^n X_i^2 -
    \left( \frac{1}{n} \sum \limits_{i = 1}^n X_i \right)^2.$$
  Распишем квадрат
  $$ \frac{1}{n} \sum \limits_{i = 1}^n X_i^2 -
    \left( \frac{1}{n} \sum \limits_{i = 1}^n X_i \right)^2
    \frac{1}{n} \sum \limits_{i = 1}^n X_i^2 -
    \frac{1}{n^2} \left(
      \sum \limits_{i = 1}^n X_i^2 + 2 \sum \limits_{i, j = 1, i < j}^n X_i X_j
    \right).$$
  Берём слева и справа математическое ожидание.
  Из того, что случайные величины в выборке одинаково распределены
  $$MS^2 =
    MX_1^2 - \frac{1}{n^2} \left[ nMX_1 + 2C_n^2 \left( MX_1 \right)^2 \right].$$
  Подставляем $C_n^2$ и группируем
  $$MX_1^2 - \frac{1}{n^2} \left[ nMX_1 + 2C_n^2 \left( MX_1 \right)^2 \right] =
    \frac{n - 1}{n} \cdot MX_1^2 - \frac{n - 1}{n} \left( MX_1 \right)^2.$$
  Вынесем общий множитель за скобки
  $$ \frac{n - 1}{n} \cdot MX_1^2 - \frac{n - 1}{n} \left( MX_1 \right)^2 = 
    \frac{n - 1}{n} \left[ MX_1^2 - \left( MX_1 \right)^2 \right] =
    \frac{n - 1}{n} \cdot SX_1.$$
  Эта оценка смещена ассимптотически;
  \item выразим $S_0$ через $S$.
  Раскроем квадрат
  $$S_0^2 =
    \frac{1}{n - 1} \sum \limits_{i = 1}^n \left( X_i - \overline{X} \right)^2 =
    \frac{1}{n - 1}
    \sum \limits_{i = 1}^n \left( X_i^2 - 2X_i \overline{X} + \overline{X}^2 \right).$$
  Имеем сумму $n$ одинаковых слагаемых
  $$ S_0^2 =
    \frac{1}{n - 1}
    \left( n \overline{X^2} - 2 \left( \overline{X} \right)^2 n + n \overline{X}^2 \right) =
    \frac{n}{n - 1} \left[ \overline{X^2} - \left( \overline{X} \right)^2 \right] =
    \frac{n - 1}{n} \cdot S^2.$$
  Отсюда следует, что
  $$MS_0^2 =
    \frac{n}{n - 1} \cdot \frac{n - 1}{n} \cdot DX_1 =
    DX_1.$$
\end{enumerate}

\addcontentsline{toc}{section}{Домашнее задание}
\section*{Домашнее задание}

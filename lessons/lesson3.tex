\addcontentsline{toc}{chapter}{Занятие 3. Метод моментов построения оценок}
\chapter*{Занятие 3. Метод моментов построения оценок}

\addcontentsline{toc}{section}{Контрольные вопросы и задания}
\section*{Контрольные вопросы и задания}

\subsubsection*{Преведиты определение оценки: несмещённой, ассимптотически несмещённой,
                состоятельной, сильно состоятельной, оптимальной.}

Оценка $ \hat{ \theta}$  несмещённая,
если $ \forall \theta \in \Theta: \, M_{ \theta } \hat{ \theta } = \theta $.

Асимптотически несмещенная оценка --- такая оценка,
математическое ожидание которой совпадает с оцениваемым параметром при $n \to \infty $.

Оценка $ \hat{ \theta }$ называется состоятельной,
если стремится к истинному значению $ \theta $ по вероятности
$ \hat{ \theta } \overset{P}{ \rightarrow } \theta, \,
  n \to \infty $.

Оценка $ \hat{ \theta }$ называется сильно состоятельной,
если стремится к истинному значению $ \theta $ почти наверное
$ \hat{ \theta } \overset{a.s.}{ \rightarrow } \theta, \,
  n \to \infty $.

Несмещённая оценка $ \hat{ \theta } \in K$
называется оптимальной в классе квадратично интегрируемых оценок $K$,
если для всякой другой несмещённой оценки
$ \tilde{ \theta } \in \Theta \,
  \forall \theta \in \Theta: \,
  D_{ \theta } \hat{ \theta } \leq D_{ \theta } \tilde{ \theta }$
или же
$ \forall \theta \in \Theta, \,
  M_{ \theta } \left( \hat{ \theta } - \theta \right)^ \leq
  M_{ \theta } \left( \tilde{ \theta } - \theta \right)^2$.

\subsubsection*{Что называется среднеквадратическим отклонением оценки?}

$M_{ \theta } \left( \hat{ \theta } - \theta \right) $ --- среднеквадратическое оклонение.

\subsubsection*{Сформулируйте утверждение про поведение выборочных моментов.}

Выборочный начальный момент $M_k \, k$-го порядка стремится к начальному моменту $ \nu_k$
случайной величины $X$, то есть
$$ \lim \limits_{n \to \infty } P \left( \left| M_k - \nu_k \right| \geq \varepsilon \right) =
  0,$$
для любого сколь угодно малого $ \varepsilon > 0$,
если моменты $ \nu_{2k}$ и $ \nu_k$ случайной величины $X$ существуют и конечны.

\subsubsection*{Сформулируйте основную идею метода моментов построения оценки
                неизвестного параметра.}

$x_1, \dotsc, x_n$ ---
выборка из распределения $F_{ \theta }$ и
$ \theta: \,
  M_{ \theta } f \left( x_1 \right) = g \left( \theta \right) $.

Вычисляем математическое ожидание, считая, что $x_1$ имеет распределение с параметром $ \theta $,
иными словами,
$$M_{ \theta } f \left( x \right) = \int \limits_{ \mathbb{R}} xdF_{ \theta } \left( x \right) , \,
  f, g \in C \left( \mathbb{R} \right), \,
  g$$
--- строго монотонная.
Тогда в силу усиленного закона больших чисел
$$ \theta \approx
  g^{-1} \left( \frac{1}{n} \sum \limits_{k = 1}^n f \left( x_k \right) \right) \overset{a.s.}
  \theta, \,
  n \to \infty.$$

Если существуют непрерывные $f$ и $g, \, g$ ---
обратима и $g \left( \theta \right) = M_{ \theta } f \left( x \right) $,
то в качестве оценки можно выбрать
$$ \hat{ \theta } =
  g^{-1} \left( \int \limits_{ \mathbb{R}} fdF_n \right).$$

\addcontentsline{toc}{section}{Аудиторные задачи}
\section*{Аудиторные задачи}

\subsubsection*{3.3}

\textit{Задание.}
Пользуясь методом моментов, оцените параметр $ \theta $ равномерного распределения на отрезке:
\begin{enumerate}[label=\alph*)]
  \item $ \left[ 0, \theta \right] $;
  \item $ \left[ \theta - 1, \theta + 1 \right] $;
  \item $ \left[ 0, 2 \theta \right] $;
  \item $ \left[ - \theta, \theta \right] $.
\end{enumerate}

\textit{Решение.}
\begin{enumerate}[label=\alph*)]
  \item $X_i \sim U \left( \left[ 0, \theta \right] \right) $.
  Записываем теоретический момент.
  Для равномерного распределения --- это средина отрезка
  $$MX_1 =
    \frac{ \theta }{2}.$$

  Должны приравнять
  $$ \frac{ \theta^*}{2} =
    \overline{X},$$
  откуда $ \theta^* = 2 \overline{X} $;
  \item случайные величины имеют распределение
  $X_i \sim
    U \left( \left[ \theta - 1, \theta + 1 \right] \right) $.

  Вычисляем теоретический момент $MX_1 = \theta $.
  Должны записать, что $ \theta^* = \overline{X}$;
  \item случайные величины имеют распределение
  $X_i \sim
    U \left( \left[ 0, 2 \theta \right] \right) $.

  Вычисляем теоретический момент $MX_1 = \theta $, откуда $ \theta^* = \overline{X}$;
  \item случайные величины имеют распределение
  $X_i \sim
    U \left( \left[ - \theta, \theta \right] \right) $.

  Вычисляем теоретический момент $MX_1 = 0$ --- не подходит,
  потому что не является функцией от $ \theta $.
  Можем вычислить второй момент, который в данном случае совпадает с дисперсией $MX_1^2 = DX_1$.
  Для равномерного распределения
  $$DX_1 =
    \frac{4 \theta^2}{12} =
    \frac{ \theta }{3}.$$
  Должны записать уравнение
  $$ \frac{ \left( \theta^2 \right)^*}{3} =
    \overline{X^2},$$
  откуда $ \left( \theta^2 \right)^* = 3 \overline{X^2}$.
  Извлекая корень, получаем $ \theta^* = \sqrt{3 \overline{X^2}}$.
\end{enumerate}

\subsubsection*{3.4}

\textit{Задание.}
Пусть $X_1, \dotsc, X_n$ --- выборка из распределения Пуассона с параметром $ \lambda $.
Пользуясь методом моментов, постройте оценку параметра $ \lambda $ и убедитесь,
что эта оценка является несмещённой и состоятельной.

\textit{Решение.} Первый теоретический момент $MX_1 = \lambda $.
Должны приравнять
$$ \lambda^* =
  \frac{1}{n} \sum \limits_{i = 1}^n X_i.$$

По закону больших чисел $ \lambda^* \overset{P}{ \to } \lambda $.
Отсюда следует состоятельность.

$M \lambda^* = MX_1 = \lambda $.
Отсюда следует несмещённость.

\subsubsection*{3.5}

\textit{Задание.} Пользуясь методом моментов с пробной функцией $g \left( y \right) = y$,
оцените параметр сдвига $ \beta \in \mathbb{R}$ показательного распределения с плотностью
$$f_{ \beta } \left( y \right) =
  \begin{cases}
    e^{ \beta - y}, \qquad t \geq \beta, \\
    0, \qquad y < \beta.
  \end{cases}$$

\textit{Решение.} $Mg \left( X_1 \right) = \overline{g \left( X \right) }$ --- метод моментов.

Ищем теоретический момент, зная плотность распределения
$$MX_1 =
  \int \limits_{ \beta }^{ \infty } ye^{ \beta - y} dy.$$
Делаем замену $y - \beta = z$ и получаем
$$ \int \limits_{ \beta }^{ \infty } ye^{ \beta - y} dy =
  \int \limits_0^{ \infty } \left( \beta + z \right) e^{-z} dz =
  \beta + 1.$$

Отсюда должны решить уравение $ \beta^* + 1 = \overline{X}$.

Отсюда $ \beta^* = \overline{X} - 1$.
Эта оценка несмещённая и состоятельная.

\subsubsection*{3.6}

\textit{Задание.}
Пусть $X_1, \dotsc, X_n$ --- выборка из биномиального распределения с параметрами $m$ и $p$.
Пользуясь методом моментов, постройте оценку:
\begin{enumerate}[label=\alph*)]
  \item параметра $p$, если параметр $m$ известный;
  \item параметра $m$, если параметр $p$ известный;
  \item векторного параметра $ \left( m, p \right) $.
\end{enumerate}

\textit{Решение.}
\begin{enumerate}[label=\alph*)]
  \item $MX_1 = mp$.
  Тогда уравенение имеет вид $mp^* = \overline{X}$.
  Отсюда
  $$p^* =
    \frac{ \overline{X}}{m};$$
  \item параметр $m$ --- это целое число.

  $m^* p = \overline{X}$, откуда
  $$m^* =
    \left[ \frac{ \overline{X}}{p} \right] $$
  --- целая часть;
  \item одного теоретического момента мало.

  Нужно найти второй теоретический момент.

  Выражаем его через дисперсию и первый момент
  $$MX_1^2 =
    DX_1 + \left( MX_1 \right)^2 =
    mp \left( 1 - p \right) + \left( mp \right)^2 =
    mp \left( 1 - p + mp \right).$$
  Составляем уравения с двумя неизвестными
  $$ \begin{cases}
      m^* p^* = \overline{X}, \\
      m^* p^* \left( 1 - p^* + m^* p^* \right) = \overline{X^2}.
    \end{cases}$$

  Во второе уравнение можем подставить $ \overline{X}$ вместо произведения $m^* p^*$ и решить
  $ \overline{X} \left( 1 - p + \overline{X} \right) = \overline{X^2}$, откуда
  $$1 - p^* + \overline{X} =
    \frac{ \overline{X^2}}{ \overline{X}}.$$

  Отсюда находим
  $$p^* =
    1 + \overline{X} - \frac{ \overline{X^2}}{ \overline{X}},$$
  соответственно
  $$m^* =
    \left[ \frac{ \overline{X}}{p^*} \right] =
    \left[ \frac{ \overline{X}}{1 + \overline{X} - \frac{ \overline{X^2}}{ \overline{X}}} \right] =
    \left[
      \frac{ \left( \overline{X} \right)^2}{ \overline{X} + \overline{X}^2 - \overline{X^2}}
    \right].$$
\end{enumerate}

\subsubsection*{3.7}

\textit{Задание.} Пусть $X_1, \dotsc, X_n$ --- выборка из распределения Парето с плотностью
$$f_{ \beta, \theta } \left( y \right) =
  \begin{cases}
    \beta \theta y^{- \left( \beta + 1 \right) }, \qquad y \geq \theta, \\
    0, \qquad t < \theta,
  \end{cases}
  \beta > 0, \,
  \theta > 0.$$

Постройте оценки по методу моментов
\begin{enumerate}[label=\alph*)]
  \item параметра $ \beta > 1$, если параметр $ \theta > 0$ известный;
  \item параметра $ \theta > 0$, если параметр $ \beta > 1$ известный;
  \item векторного параметра $ \left( \beta, \theta \right) $, где $ \beta > 2, \, \theta > 0$.
\end{enumerate}

\textit{Решение.}
\begin{enumerate}[label=\alph*)]
  \item Вычисляем теоретический момент
  $$MX_1 =
    \beta \int \limits_{ \theta }^{ \infty } \theta^{ \beta } y^{- \beta } dy =
    \beta \theta^{ \beta } \cdot \frac{y^{- \beta + 1}}{- \beta + 1} =
    - \beta \theta^{ \beta } \cdot \frac{ \theta }{1 - \beta } =
    \frac{ \beta }{ \beta - 1} \cdot \theta.$$

  Найдём второй момент
  $$MX_1^2 =
    \beta \int \limits_{ \theta }^{ \infty } \theta^{ \beta } y^{- \beta + 1} dy =
    \beta \cdot \frac{1}{ \beta - 2} \cdot \theta^2.$$

  Приравниваем теоретический и выборочный момент, учитывая, что $ \beta $ --- параметр
  $$ \frac{ \beta^*}{ \beta^* - 1} \cdot \theta =
    \overline{X},$$
  откуда
  $$ \frac{ \beta^*}{ \beta^* - 1} =
    \frac{ \overline{X}}{ \theta }.$$
  Решим пропорцию
  $ \beta^* \theta =
    \left( \beta - 1 \right) \overline{X} =
    \beta^* \overline{X} - \overline{X}$.

  Получаем
  $$ \beta^* =
    \frac{ \overline{X}}{ \overline{X} - \theta };$$
  \item считаем, что $ \beta $ --- известно, ищем $ \theta $.

  Приравниваем моменты
  $$ \frac{ \beta }{ \beta - 1} \cdot \theta =
    \overline{X},$$
  откуда
  $$ \theta^* =
    \frac{ \overline{X} \left( \beta - 1 \right) }{ \beta };$$
  \item нужно 2 уравнения
  $$ \begin{cases}
      \frac{ \beta^*}{ \beta^* - 1} \cdot \theta^* = \overline{X}, \\
      \frac{ \beta^*}{ \beta^* - 2} \cdot \left( \theta^* \right)^2 = \overline{X^2}.
    \end{cases}$$

  Выражаем $ \theta^*$ из первого уравнения и подставляем во второе
  $$ \frac{ \beta^*}{ \beta^* - 2} \cdot
    \left( \frac{ \beta^* - 1}{ \beta^*} \right)^2 \cdot
    \left( \overline{X} \right)^2 =
    \overline{X^2}.$$
  Умножим на знаменатель
  $ \left( \overline{X} \right)^2 \left( \beta^* - 1 \right)^2 =
    \overline{X^2} \beta^* \left( \beta^* - 2 \right) $.

  Оставляем в левой части только квадрат выборочного момента
  $$ \left( \overline{X} \right)^2 =
    \overline{X^2} \cdot
    \frac{ \left( \beta^* \right)^2 - 2 \beta^*}{ \left( \beta^* - 1 \right)^2}.$$

  Делаем замену $ \left( \beta^* - 1 \right)^2 = t$.

  Получаем
  $$ \left( \overline{X} \right)^2 =
    \frac{t - 1}{t} \cdot \overline{X^2} =
    \left( 1 - \frac{1}{t} \right) \cdot \overline{X^2} =
    \overline{X^2} - \frac{ \overline{X^2}}{t}.$$

  Отсюда
  $$t =
    \frac{ \overline{X^2}}{ \overline{X^2} - \left( \overline{X} \right)^2}.$$

  Отсюда найдём $ \beta^* = \sqrt{t} + 1$, соответственно находим $ \theta^*$.
\end{enumerate}

\addcontentsline{toc}{section}{Домашнее задание}
\section*{Домашнее задание}

\subsubsection*{3.10}

\textit{Задание.}
Пользуясь методом моментов,
оцените значение $ \alpha $
по выборке из показательного распределения с параметром $1 / \sqrt{ \alpha }$.

\textit{Решение.}
$$X_i \sim
  Exp \left( \frac{1}{ \sqrt{ \alpha }} \right).$$
Записываем теоретический момент.
Для показательного распределения --- это обратная величина к параметру распределения
$$MX_1 =
  \frac{1}{ \frac{1}{ \sqrt{ \alpha }}} =
  \sqrt{ \alpha }.$$

Должны приравнять $ \sqrt{ \alpha^*} = \overline{X}$,
откуда $ \alpha^* = \left( \overline{X} \right)^2$.

\subsubsection*{3.11}

\textit{Задание.}
Пусть $X_1, \dotsc, X_n$ --- выборка из показательного распределения с параметром $ \alpha > 0$.
Пользуясь методом моментов с пробной функцией $g \left( y \right) = y$,
оцените параметр $ \theta \left( \alpha \right) = P \left( X_1 > 1 \right) $.

\textit{Решение.} $Mg \left( X_1 \right) = \overline{g \left( X \right) }$ --- метод моментов.

Ищем теоретический момент, зная плотность распределения
$$MX_1 =
  \frac{1}{ \alpha }.$$
Из метода моментов следует, что
$$ \frac{1}{ \alpha^*} =
  \overline{X}.$$
Выражаем оценку
$$ \alpha^* =
  \frac{1}{ \overline{X}}.$$

Перейдём в параметре к противоположному событию
$$ \theta \left( \alpha \right) =
  P \left( X_1 > 1 \right) =
  1 - P \left( X_1 \leq 1 \right) =
  1 - F_{X_1} \left( 1 \right).$$
Найдём функцию показательного распределения как интеграл от плотности
$$F_{X_1} \left( y \right) =
  \int \limits_0^y \alpha e^{ - \alpha x} dx =
  \alpha \cdot \left( - \frac{1}{ \alpha } \right) \cdot
  \int \limits_0^{ \alpha } e^{- \alpha x} d \left( - \alpha x \right) =
  \left. -e^{- \alpha x} \right|_0^y =
  1 - e^{- \alpha y}.$$
Тогда параметр примет вид
$ \theta \left( \alpha \right) =
  1 - \left( 1 - e^{- \alpha } \right) =
  1 - 1 + e^{- \alpha } =
  e^{- \alpha } =
  e^{- \frac{1}{ \overline{X}}}.$

\subsubsection*{3.12}

\textit{Задание.}
Пользуясь методом моментов с пробной функцией $g \left( y \right) = y$,
оцените параметр $p$ распределения Бернулли.
Возможно ли с помощью метода мометов с некоторой
пробной функией $g \left( y \right) $ получить оценку параметра $p$, отличную от $ \overline{X}$?

\textit{Решение.}
$$X_i =
  \begin{cases}
    0, \qquad 1 - p, \\
    1, \qquad p.
  \end{cases}$$

Теоретический момент $MX_1 = p, \, p^* = \overline{X}$.

Найдём теоретический момент с пробной функцией
$$Mg \left( X_1 \right) =
  g \left( 0 \right) \left( 1 - p \right) + g \left( 0 \right) p =
  g \left( 0 \right) + p \left[ g \left( 1 \right) - g \left( 0 \right) \right] =
  \overline{g \left( X \right) }.$$
Отсюда
$$p^* =
  \frac{g \left( \overline{X} \right) - g \left( 0 \right) }{g \left( 1 \right) - g \left( 0 \right) }=
  \frac{g \left( X_1 \right) + \dotsc + g \left( X_n \right) - g \left( 0 \right) }{n \left[ g \left( 1 \right) - g \left( 0 \right) \right]}.$$
Количество единиц равно $ \overline{X} n$, соответственно $n - \overline{X} n$ --- количество нулей.
$$\frac{g \left( X_1 \right) + \dotsc + g \left( X_n \right) - g \left( 0 \right) }{n \left[ g \left( 1 \right) - g \left( 0 \right) \right]} =
  \frac{g \left( 1 \right) \overline{X} n + g \left( 0 \right) \left( n - \overline{X} n \right) - g \left( 0 \right) n}{n \left[ g \left( 1 \right) - g \left( 0 \right) \right] }.$$
Раскроем скобки и приведём подобные
$$ \frac{g \left( 1 \right) \overline{X} n + g \left( 0 \right) \left( n - \overline{X} n \right) - g \left( 0 \right) n}{n \left[ g \left( 1 \right) - g \left( 0 \right) \right] } =
  \frac{ng \left( 1 \right) \overline{X} - gn \overline{X}}{n \left[ g \left( 1 \right) - g \left( 0 \right) \right] } =
  \overline{X},$$
откуда следует, что невозможно получить другую оценку параметра $p$.

\subsubsection*{3.13}

\textit{Задание.}
Пользуясь методом моментов,
оцените параметр $ \lambda > 1$ по выборке из распределения Пуассона с параметром $ln \lambda $.

\textit{Решение.} Первый теоретический момент $MX_1 = ln \lambda $.
Должны приравнять $ln \lambda = \overline{X}$.
Отсюда следует, что $ \lambda^* = e^{ \overline{X}}$.

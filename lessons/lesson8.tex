\addcontentsline{toc}{chapter}{Занятие 8. Основные распределения, связанные с нормальным законом}
\chapter*{Занятие 8. Основные распределения, связанные с нормальным законом}

\addcontentsline{toc}{section}{Контрольные вопросы и задания}
\section*{Контрольные вопросы и задания}

\subsubsection*{Приведите определение распределения Пирсона $ \left( \chi^2 \right) $ с $n$
                степенями свободы, распределения Стьюдента ($t$-распределения)
                с $n$ степенями свободы, распределения Фишера ($F$-распределения)
                с $ \left(m, n \right) $ степенями свободы.}

Считаем, что $ \vec{ \xi }$ --- стандартный гауссовский вектор.

Случайная величина $ \xi_1^2 + \dotsc + \xi_n^2$ имеет распределение $ \chi_n^2$.

Есть $n + 1$ независимая стандартная гауссовская случайная величина $ \xi_0, \dotsc, \xi_n$.
Отношение первой (нулевой) случайной величины к корню суммы квадратов остальных, делённой на $n$,
имеет распределение Стьюдента с $n$ степенями свободы
$$ \frac{ \xi_0}{ \sqrt{ \frac{1}{n} \sum \limits_{k = 1}^n \xi_k^2}} \sim
  t_n.$$

Отношение независимых случайных величин $ \chi_{k_1}^2$ и $ \chi_{k_2}^2$
называется распределением Фишера
$$F_{k_1, k_2} =
  \frac{ \chi_{k_1}^2}{ \chi_{k_2}^2}.$$

\addcontentsline{toc}{section}{Аудиторные задачи}
\section*{Аудиторные задачи}

\subsubsection*{8.3}

\textit{Задание.}
Пусть $X_1, \dotsc, x_n$ --- выборка из распределения $N \left( a, \sigma^2 \right) $.
Докажите, что величина
$$ \frac{ \left( \overline{X} - a \right) \sqrt{n}}{ \sigma }$$
имеет распределение $N \left( 0, 1 \right) $.

\textit{Решение.}
$ \varphi_{ \overline{X}} \left( t \right) =
  Me^{it \overline{X}} =
  Me^{it \cdot \frac{1}{n} \sum \limits_{i = 1}^n X_i} =
  Me^{it \cdot \frac{1}{n} \left( X_1 + \dotsc + X_n \right) }$.
Воспользуемся независимостью
$$Me^{it \cdot \frac{1}{n} \left( X_1 + \dotsc + X_n \right) } =
  \varphi_{X_1} \left( \frac{t}{n} \right) \cdot \dotsc \cdot
  \varphi_{X_n} \left( \frac{t}{n} \right).$$
Воспользуемся одинаковой распределённостью
$$ \varphi_{X_1} \left( \frac{t}{n} \right) \cdot \dotsc \cdot
  \varphi_{X_n} \left( \frac{t}{n} \right) =
  exp \left\{ \left( \frac{iat}{n} - \frac{t^2 \sigma^2}{2n^2} \right) n \right\} =
  exp \left\{ ita - \frac{t^2 \sigma^2}{2n} \right\},$$
значит,
$$ \overline{X} \sim
  N \left( a, \frac{ \sigma^2}{n} \right).$$
После этого центрированием следует, что
$$ \overline{X} - a \sim
  N \left( 0, \frac{ \sigma^2}{n} \right).$$

Получаем, что
$$ \frac{ \left( \overline{X} - a \right) \sqrt{n}}{ \sigma } \sim
  N \left( 0, 1 \right).$$

\subsubsection*{8.4}

\textit{Задание.}
Пусть $ \chi_n^2$ --- случайная величина,
распределённая по закону $ \chi^2$ с $n$ степенями свободы.
Вычислите её математическое ожидание $M \chi_n^2$ и дисперсию $D \chi_n^2$.

\textit{Решение.}
Раз эта случайная величина имеет распределение $ \chi^2$ с $n$ степенями свободы, изобразим её как
$$ \chi_n^2 =
  \sum \limits_{ \xi_i^2},$$
где $ \xi_1, \dotsc, \xi_n$ --- независимые, имеющие распределение $N \left( 0, 1 \right) $.
Тогда
$$M \chi_n^2 =
  M \sum \limits_{i = 1}^n \xi_i^2 =
  nM \xi_1^2 =
  n \left[ M \xi_1^2 + \left( D \xi_1 \right)^2 \right] =
  n.$$

Теперь ищем дисперсию
$$D \chi_n^2 =
  D \sum \limits_{i = 1}^n \xi_i^2.$$
Случайные величины независимы
$$D \sum \limits_{i = 1}^n \xi_i^2 =
  nD \xi_1^2 =
  n \left[ M \xi_1^4 - \left( M \xi_1^2 \right)^2 \right].$$
Случайная величина $ \xi \sim N \left( 0, 1 \right) $.
Если $ \xi \sim N \left( 0, \sigma^2 \right) $,
то
$$M \xi^{2n} =
  \left( 2n - 1 \right)!! \left( \sigma^2 \right)^n,$$
значит,
$n \left[ M \xi_1^4 - \left( M \xi_1^2 \right)^2 \right] =
  n \left( 3 - 1 \right) =
  2n$.

\subsubsection*{8.5}

\textit{Задание.}
Пусть $ \chi_n^2$ --- случайная величина,
распределённая по закону $ \chi^2$ с $n$ степенями свободы.
Докажите, что
$$ \frac{1}{n} \cdot \chi_n^2 \overset{P}{ \rightarrow}
  1$$
при $n \to \infty $.

\textit{Решение.}
$$ \frac{ \xi_1^2 + \dotsc + \xi_n^2}{n} \overset{P}{ \rightarrow }
  1,$$
где $ \xi_1, \dotsc, \xi_n$ ---
независимые случайные величины со стандартным нормальным распределением $N \left( 0, 1 \right) $.

Такая сумма похожа на закон больших чисел.
Раз $ \xi_i$ --- независимы, то их квадраты тоже независимы между собой.

Тогда
$$ \frac{ \sum \limits_{i = 1}^n  \xi_i}{n} \overset{P}{ \rightarrow }
  M \xi_i^2 =
  1$$
по закону больших чисел.

\subsubsection*{8.6}

\textit{Задание.}
Пусть $ \xi_1, \dotsc, \xi_n$ --- независимые случайные величины,
причём $ \xi_i$ имеет распределение $ \chi^2$ с $m_i$ степенями свободы.
Докажите,
что сумма $ \xi = \xi_1 + \dotsc + \xi_n$ этих случайных величин имеет распределение $ \chi^2$ с
$$m =
  m_1 + \dotsc + m_n$$
степенями свободы.

\textit{Решение.} $ \xi_i \sim \chi_{m_i}^2$ --- знаем.
Нужно доказать, что тогда
$$ \xi =
  \sum \limits_{i = 1}^n \xi_i \sim
  \chi_m^2,$$
где $m = m_1 + \dotsc + m_n$.

Сумма квадратов $m_1$-й случайной величины
$$ \xi_1 =
  \sum \limits_{i = 1}^{m_1} \theta_{1i}^2,$$
где $ \theta_{1i} \sim N \left( 0, 1 \right)$, и они независимы между собой, соответственно
$$ \xi_1 =
  \sum \limits_{i = 1}^{m_2} \theta_{2i}^2,$$
где $ \theta_{2i} \sim N \left( 0, 1 \right) $, и они независимы между собой и так далее,
$$ \xi_n =
  \sum \limits_{i = 1}^{m_n} \theta_{ni}^2,$$
где $ \theta_{2i} \sim N \left( 0, 1 \right) $, и они независимы между собой.

Всего будет $m_1 + m_2 + \dotsc + m_n$ слагаемых.
Случайные величины
$$ \theta_{1i}, \dotsc, \theta_{ni}$$
--- независимы между собой.
$$ \xi =
  \sum \limits_{i = 1}^n  \xi_i =
  \sum \limits_{i = 1}^n \sum \limits_{j = 1}^{m_i} \theta_{ij}^2.$$

\addcontentsline{toc}{section}{Домашнее задание}
\section*{Домашнее задание}

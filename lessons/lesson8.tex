\addcontentsline{toc}{chapter}{Занятие 8. Основные распределения, связанные с нормальным законом}
\chapter*{Занятие 8. Основные распределения, связанные с нормальным законом}

\addcontentsline{toc}{section}{Контрольные вопросы и задания}
\section*{Контрольные вопросы и задания}

\subsubsection*{Приведите определение распределения Пирсона $ \left( \chi^2 \right) $ с $n$
                степенями свободы, распределения Стьюдента ($t$-распределения)
                с $n$ степенями свободы, распределения Фишера ($F$-распределения)
                с $ \left(m, n \right) $ степенями свободы.}

Считаем, что $ \vec{ \xi }$ --- стандартный гауссовский вектор.

Случайная величина $ \xi_1^2 + \dotsc + \xi_n^2$ имеет распределение $ \chi_n^2$.

Есть $n + 1$ независимая стандартная гауссовская случайная величина $ \xi_0, \dotsc, \xi_n$.
Отношение первой (нулевой) случайной величины к корню суммы квадратов остальных, делённой на $n$,
имеет распределение Стьюдента с $n$ степенями свободы
$$ \frac{ \xi_0}{ \sqrt{ \frac{1}{n} \sum \limits_{k = 1}^n \xi_k^2}} \sim
  t_n.$$

Отношение независимых случайных величин $ \chi_{k_1}^2$ и $ \chi_{k_2}^2$
называется распределением Фишера
$$F_{k_1, k_2} =
  \frac{ \chi_{k_1}^2}{ \chi_{k_2}^2}.$$

\addcontentsline{toc}{section}{Аудиторные задачи}
\section*{Аудиторные задачи}

\subsubsection*{8.3}

\textit{Задание.}
Пусть $X_1, \dotsc, x_n$ --- выборка из распределения $N \left( a, \sigma^2 \right) $.
Докажите, что величина
$$ \frac{ \left( \overline{X} - a \right) \sqrt{n}}{ \sigma }$$
имеет распределение $N \left( 0, 1 \right) $.

\textit{Решение.}
$ \varphi_{ \overline{X}} \left( t \right) =
  Me^{it \overline{X}} =
  Me^{it \cdot \frac{1}{n} \sum \limits_{i = 1}^n X_i} =
  Me^{it \cdot \frac{1}{n} \left( X_1 + \dotsc + X_n \right) }$.
Воспользуемся независимостью
$$Me^{it \cdot \frac{1}{n} \left( X_1 + \dotsc + X_n \right) } =
  \varphi_{X_1} \left( \frac{t}{n} \right) \cdot \dotsc \cdot
  \varphi_{X_n} \left( \frac{t}{n} \right).$$
Воспользуемся одинаковой распределённостью
$$ \varphi_{X_1} \left( \frac{t}{n} \right) \cdot \dotsc \cdot
  \varphi_{X_n} \left( \frac{t}{n} \right) =
  exp \left\{ \left( \frac{iat}{n} - \frac{t^2 \sigma^2}{2n^2} \right) n \right\} =
  exp \left\{ ita - \frac{t^2 \sigma^2}{2n} \right\},$$
значит,
$$ \overline{X} \sim
  N \left( a, \frac{ \sigma^2}{n} \right).$$
После этого центрированием следует, что
$$ \overline{X} - a \sim
  N \left( 0, \frac{ \sigma^2}{n} \right).$$

Получаем, что
$$ \frac{ \left( \overline{X} - a \right) \sqrt{n}}{ \sigma } \sim
  N \left( 0, 1 \right).$$

\subsubsection*{8.4}

\textit{Задание.}
Пусть $ \chi_n^2$ --- случайная величина,
распределённая по закону $ \chi^2$ с $n$ степенями свободы.
Вычислите её математическое ожидание $M \chi_n^2$ и дисперсию $D \chi_n^2$.

\textit{Решение.}
Раз эта случайная величина имеет распределение $ \chi^2$ с $n$ степенями свободы, изобразим её как
$$ \chi_n^2 =
  \sum \limits_{ \xi_i^2},$$
где $ \xi_1, \dotsc, \xi_n$ --- независимые, имеющие распределение $N \left( 0, 1 \right) $.
Тогда
$$M \chi_n^2 =
  M \sum \limits_{i = 1}^n \xi_i^2 =
  nM \xi_1^2 =
  n \left[ M \xi_1^2 + \left( D \xi_1 \right)^2 \right] =
  n.$$

Теперь ищем дисперсию
$$D \chi_n^2 =
  D \sum \limits_{i = 1}^n \xi_i^2.$$
Случайные величины независимы
$$D \sum \limits_{i = 1}^n \xi_i^2 =
  nD \xi_1^2 =
  n \left[ M \xi_1^4 - \left( M \xi_1^2 \right)^2 \right].$$
Случайная величина $ \xi \sim N \left( 0, 1 \right) $.
Если $ \xi \sim N \left( 0, \sigma^2 \right) $,
то
$$M \xi^{2n} =
  \left( 2n - 1 \right)!! \left( \sigma^2 \right)^n,$$
значит,
$n \left[ M \xi_1^4 - \left( M \xi_1^2 \right)^2 \right] =
  n \left( 3 - 1 \right) =
  2n$.

\subsubsection*{8.5}

\textit{Задание.}
Пусть $ \chi_n^2$ --- случайная величина,
распределённая по закону $ \chi^2$ с $n$ степенями свободы.
Докажите, что
$$ \frac{1}{n} \cdot \chi_n^2 \overset{P}{ \rightarrow}
  1$$
при $n \to \infty $.

\textit{Решение.}
$$ \frac{ \xi_1^2 + \dotsc + \xi_n^2}{n} \overset{P}{ \rightarrow }
  1,$$
где $ \xi_1, \dotsc, \xi_n$ ---
независимые случайные величины со стандартным нормальным распределением $N \left( 0, 1 \right) $.

Такая сумма похожа на закон больших чисел.
Раз $ \xi_i$ --- независимы, то их квадраты тоже независимы между собой.

Тогда
$$ \frac{ \sum \limits_{i = 1}^n  \xi_i}{n} \overset{P}{ \rightarrow }
  M \xi_i^2 =
  1$$
по закону больших чисел.

\subsubsection*{8.6}

\textit{Задание.}
Пусть $ \xi_1, \dotsc, \xi_n$ --- независимые случайные величины,
причём $ \xi_i$ имеет распределение $ \chi^2$ с $m_i$ степенями свободы.
Докажите,
что сумма $ \xi = \xi_1 + \dotsc + \xi_n$ этих случайных величин имеет распределение $ \chi^2$ с
$$m =
  m_1 + \dotsc + m_n$$
степенями свободы.

\textit{Решение.} $ \xi_i \sim \chi_{m_i}^2$ --- знаем.
Нужно доказать, что тогда
$$ \xi =
  \sum \limits_{i = 1}^n \xi_i \sim
  \chi_m^2,$$
где $m = m_1 + \dotsc + m_n$.

Сумма квадратов $m_1$-й случайной величины
$$ \xi_1 =
  \sum \limits_{i = 1}^{m_1} \theta_{1i}^2,$$
где $ \theta_{1i} \sim N \left( 0, 1 \right)$, и они независимы между собой, соответственно
$$ \xi_1 =
  \sum \limits_{i = 1}^{m_2} \theta_{2i}^2,$$
где $ \theta_{2i} \sim N \left( 0, 1 \right) $, и они независимы между собой и так далее,
$$ \xi_n =
  \sum \limits_{i = 1}^{m_n} \theta_{ni}^2,$$
где $ \theta_{2i} \sim N \left( 0, 1 \right) $, и они независимы между собой.

Всего будет $m_1 + m_2 + \dotsc + m_n$ слагаемых.
Случайные величины
$$ \theta_{1i}, \dotsc, \theta_{ni}$$
--- независимы между собой.
$$ \xi =
  \sum \limits_{i = 1}^n  \xi_i =
  \sum \limits_{i = 1}^n \sum \limits_{j = 1}^{m_i} \theta_{ij}^2.$$

\subsubsection*{8.9}

\textit{Задание.}
Пусть $X_1, \dotsc, X_n$ --- выборка из показательного распределения с параметром $ \lambda $.
Докажите,
что статистика $T_n = 2n \lambda \overline{X}$ имеет распределение $ \chi^2$ с $2n$
стеепенями свободы.

\textit{Решение.}
$$T_n =
  2n \lambda \overline{X} =
  2 \lambda \sum \limits_{i = 1}^n X_i,$$
где $X_i \sim Exp \left( \lambda \right) $,
тогда статистика $T_n \overset{d}{ \sim } \xi_1^2 + \dotsc + \xi_{2n}^2$, где $ \xi_i^2$ ---
независимые одинаково распределённые случайные величины,
каждая из которых имеет стандартное нормальное распределение $N \left( 0, 1 \right) $.

Имеем $n$ пар
$ \left( \xi_1^2 + \xi_2^2 \right) + \left( \xi_3^2 + \xi_4^2 \right) + \dotsc +
  \left( \xi_{2n - 1}^2 + \xi_{2n}^2 \right)$,
имеющих распределение
$$Exp \left( \frac{1}{2} \right).$$

Будет сумма $n$ одинаково распределённых показательных случайных величин с параметром $0.5$.

Для распределения Эрланга
$$p_{X_1 + X_2} \left( x \right) =
  \int \limits_{ \mathbb{R}} p_{X_1} \left( v \right) p_{X_2} \left( x - v \right) dv.$$
Распределение показательное, следовательно, $v \leq x$.
Получаем
$$ \int \limits_{ \mathbb{R}} p_{X_1} \left( v \right) p_{X_2} \left( x - v \right) dv =
  \int \limits_0^x \lambda e^{- \lambda v} \lambda e^{- \lambda \left( x - v \right) } dv =
  \lambda^2 \int \limits_0^x e^{- \lambda v + \lambda v - \lambda x} dv.$$
Упрощаем степень экспоненты
$$ \lambda^2 \int \limits_0^x e^{- \lambda v + \lambda v - \lambda x} dv =
  \lambda^2 \int \limits_0^x e^{- \lambda x} dv =
  \lambda^2 e^{- \lambda x} x.$$

Плотность суммы $n$ слагаемых равна (по предположению индукции)
$$p_{ \sum \limits_{i = 1}^n X_i} \left( x \right) =
  \frac{ \lambda^n}{ \left( n - 1 \right)!} \cdot x^{n - 1} e^{- \lambda x}.$$

Делаем шаг индукции
$$p_{ \sum \limits_{i = 1}^{n + 1}} \left( x \right) =
  \int \limits_{ \mathbb{R}}
    p_{X_1 + \dotsc + X_n} \left( v \right) p_{X_{n + 1}} \left( x - v \right)
  dv.$$
Подставляем явный вид плотностей
$$ \int \limits_{ \mathbb{R}}
    p_{X_1 + \dotsc + X_n} \left( v \right) p_{X_{n + 1}} \left( x - v \right)
  dv =
  \int \limits_0^x
    \frac{ \lambda^n}{ \left( n - 1 \right)!} \cdot v^{n - 1} e^{- \lambda v} \lambda \cdot
    e^{- \lambda \left( x - v \right) }
  dv.$$
Выносим константы и берём интеграл
$$ \int \limits_0^x
    \frac{ \lambda^n}{ \left( n - 1 \right)!} \cdot v^{n - 1} e^{- \lambda v} \lambda \cdot
    e^{- \lambda \left( x - v \right) }
  dv =
  \frac{ \lambda^{n + 1}}{n!} \cdot e^{- \lambda x} x^n.$$

Функция распределения
$$F_{T_n} \left( x \right) =
  P \left( T_n \leq x \right) =
  P \left( \sum \limits_{X_i} \leq \frac{x}{2 \lambda } \right) =
  F_{ \sum \limits_{i = 1}^n X_i} \left( \frac{x}{2 \lambda } \right).$$

Плотность распределения получается дифференцированием функции распределения
$$p_{T_n} \left( x \right) =
  \frac{1}{2 \lambda } \cdot p_{ \sum \limits_{i = 1}^n X_i} \left( \frac{x}{2 \lambda } \right).$$

Подставим
$$ \frac{x}{2 \lambda }$$
в соответствующее выражение
$$p_{T_n} \left( x \right) =
  \frac{1}{2 \lambda } \cdot \frac{ \lambda^n}{ \left( n - 1 \right)!} \cdot
  \left( \frac{x}{2 \lambda } \right)^{n - 1} \cdot e^{- \frac{x}{2}} =
  \frac{x^{n - 1}}{2^n \left( n - 1 \right)!} \cdot e^{- \frac{x}{2}}.$$

Если бы знали плотность $ \xi_1^2$ или $ \xi_2^2$, было бы удобно
$$p_{ \xi_1^2 + \xi_2^2} \left( x \right) =
  \int \limits_{ \mathbb{R}} p_{ \xi_1^2} \left( v \right) p_{ \xi_2^2} \left( x - v \right) dv.$$

По определению
$F_{ \xi_1^2 + \xi_2^2} \left( x \right) =
  P \left( \xi_1^2 + \xi_2^2 \leq x \right) =
  M \mathbbm{1} \left\{ \xi_1^2 + \xi_2^2 \leq x \right\}$.
По правилу вычисления математического ожидания, если есть две случайные величины, будет 2 интеграла
$$M \mathbbm{1} \left\{ \xi_1^2 + \xi_2^2 \leq x \right\} =
  \iint \limits{ \mathbb{R}^2}
    \mathbbm{1} \left\{ v^2 + u^2 \leq x \right\} \cdot p_{ \left( \xi_1, \xi_2 \right) }
  dudv.$$
Они независимы, значит, совместная плотность --- это произведение плотностей, которые мы знаем.
Подставляем в явном виде.
Каждая имеет стандартное нормальное распределение
$$ \iint \limits{ \mathbb{R}^2}
    \mathbbm{1} \left\{ v^2 + u^2 \leq x \right\} \cdot p_{ \left( \xi_1, \xi_2 \right) }
  dudv =
  \iint \limits_{ \mathbb{R}^2}
    \mathbbm{1} \left\{ v^2 + u^2 \leq x \right\} \cdot \frac{1}{2 \pi } \cdot
    e^{- \frac{v^2 + u^2}{2}}
  dudv.$$
Замена: $u = \rho \cos \phi, \, v = \rho \sin \phi$.
Получаем
$$ \iint \limits_{ \mathbb{R}^2}
    \mathbbm{1} \left\{ v^2 + u^2 \leq x \right\} \cdot \frac{1}{2 \pi } \cdot
    e^{- \frac{v^2 + u^2}{2}}
  dudv =
  \int \limits_0^{ \sqrt{x}}
    \int \limits_0^{2 \pi } \frac{1}{2 \pi } \cdot e^{- \frac{ \rho^2}{2}} \rho d \rho
  d \phi.$$
По $ \phi $ можем проинтегрировать сразу
$$ \int \limits_0^{ \sqrt{x}}
    \int \limits_0^{2 \pi } \frac{1}{2 \pi } \cdot e^{- \frac{ \rho^2}{2}} \rho d \rho
  d \phi =
  \frac{2 \pi }{2 \pi } \cdot
  \int \limits_0^{ \sqrt{x}} e^{- \frac{ \rho^2}{2}} d \left( \frac{ \rho^2}{2} \right) =
  \left. -e^{- \frac{ \rho^2}{2}} \right|_0^{ \sqrt{x}} =
  1 - e^{- \frac{x}{2}}$$
--- функция показательного распределения с параметром $0.5$.

$$p_{ \xi_1^2 + \dotsc + \xi_n^2} \left( x \right) =
  \left( \frac{1}{2} \right)^n \cdot \frac{1}{ \left( n - 1 \right)!} \cdot
  x^{n - 1} e^{- \frac{x}{2}}.$$

Такая же плотность была у случайной величины $T_n$.

\addcontentsline{toc}{section}{Домашнее задание}
\section*{Домашнее задание}

\subsubsection*{8.10}

\textit{Задание.}
Пусть $ \chi_n^2$ --- случайная величина,
распределённая по закону $ \chi^2$ с $n$ степенями свободы.
Докажите, что
$$ \frac{ \chi_n^2 - n}{ \sqrt{2n}} \overset{d}{ \rightarrow}
  N \left( 0, 1 \right) $$
при $n \to \infty $.

\textit{Решение.}
Это следует из центральной предельной теоремы при
$$M \xi_1^2 = 1, \,
  D \xi_1^2 = 2.$$


\subsubsection*{8.11}

\textit{Задание.} Пусть
$$t_n =
  \frac{ \xi }{ \sqrt{ \frac{1}{n} \chi_n^2}}$$
--- случайная величина, распределённая по закону Стьюдента с $n$ степенями свободы.
Докажите, что $t_n \overset{P}{ \rightarrow } \xi $ при $n \to \infty $.

\textit{Решение.}
$$t_n =
  \frac{ \xi }{ \sqrt{ \frac{1}{n} \chi_n^2}} =
  \frac{ \xi }{ \sqrt{ \frac{1}{n} \sum \limits_{i = 1}^n \xi_i^2}}.$$

По закону больших чисел
$$ \frac{1}{n} \sum \limits_{i = 1}^n  \xi_i^2 \overset{P}{ \rightarrow }
  M \xi_1 =
  1, \,
  n \to \infty,$$
тогда
$$t_n \overset{P}{ \rightarrow } \frac{ \xi }{ \sqrt{1}} = \xi, \,
  n \to \infty.$$

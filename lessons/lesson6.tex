\addcontentsline{toc}{chapter}{Занятие 6. Теорема Колмогорова про улучшение оценок.
                              Неравенство Рао-Крамера}
\chapter*{Занятие 6. Теорема Колмогорова про улучшение оценок. Неравенство Рао-Крамера}

\addcontentsline{toc}{section}{Контрольные вопросы и задания}
\section*{Контрольные вопросы и задания}

\subsubsection*{Приведите определение достаточной статистики.}

Статистика $T$ называется достаточной для параметра $ \theta $,
если условное распределение при известном $T$ не зависит от параметра $ \theta $.

\subsubsection*{Сформулируйте теорему про характеризацию достаточной статистики.}

Пусть $x_1, \dotsc, x_n$ ---
выборка из распределения с плотностью
$$p \left( x, \theta \right), \,
  \theta \in \Theta.$$

Статистика $T$ является достаточной тогда и только тогда,
когда функция правдоподобия $L \left( \vec{x}, \theta \right) $ допускает факторизацию,
то есть может быть представлена произведением двух функций следующего вида
$$L \left( \vec{x}, \theta \right) =
  h \left( T, \theta \right) \cdot g \left( \vec{x} \right).$$

\subsubsection*{Сформулируйте теорему Колмогорова про улучшение
                оценки с помощью достаточной статистики.}

Оптимальная оценка единственная (в том случае, когда она существует).

\subsubsection*{Что называется количеством информации Фишера?}

$$D_{ \theta } U \left( \vec{x}, \theta \right) =
  nD_{ \theta } \left(
    \frac{ \partial }{ \partial \theta } ln \, p \left( x, \theta \right)
  \right) =
  nI_0,$$
где $nI_0$ --- количество информации Фишера.

\subsubsection*{Запишите неравенство Рао-Крамера.}

Пусть $ \hat{ \theta }$ --- несмещённая оценка для параметра $ \theta $.
Тогда
$$ \forall \theta \in \Theta: \,
  D_{ \theta } \hat{ \theta } \geq \frac{1}{nI_0}.$$
Как угодно лучшую оценку взять нельзя.

\subsubsection*{Какая оценка называется эффективной?}

Если $ \hat{ \theta }$ такова, что $ \hat{ \theta }$ --- несмещённая и
$$D_{ \theta } \hat{ \theta } =
  \frac{1}{nI_0}, \,
  \theta \in \Theta,$$
то $ \hat{ \theta }$ называется эффективной оценкой.

\subsubsection*{Приведите определение экспоненциальной семьи распределений.}

\addcontentsline{toc}{section}{Аудиторные задачи}
\section*{Аудиторные задачи}

\addcontentsline{toc}{section}{Домашнее задание}
\section*{Домашнее задание}

\addcontentsline{toc}{chapter}{Занятие 8. Основные распределения, связанные с нормальным законом}
\chapter*{Занятие 8. Основные распределения, связанные с нормальным законом}

\addcontentsline{toc}{section}{Контрольные вопросы и задания}
\section*{Контрольные вопросы и задания}

\subsubsection*{Приведите определение распределения Пирсона $ \left( \chi^2 \right) $ с $n$
                степенями свободы, распределения Стьюдента ($t$-распределения)
                с $n$ степенями свободы, распределения Фишера ($F$-распределения)
                с $ \left(m, n \right) $ степенями свободы.}

Считаем, что $ \vec{ \xi }$ --- стандартный гауссовский вектор.

Случайная величина $ \xi_1^2 + \dotsc + \xi_n^2$ имеет распределение $ \chi_n^2$.

Есть $n + 1$ независимая стандартная гауссовская случайная величина $ \xi_0, \dotsc, \xi_n$.
Отношение первой (нулевой) случайной величины к корню суммы квадратов остальных, делённой на $n$,
имеет распределение Стьюдента с $n$ степенями свободы
$$ \frac{ \xi_0}{ \sqrt{ \frac{1}{n} \sum \limits_{k = 1}^n \xi_k^2}} \sim
  t_n.$$

Отношение независимых случайных величин $ \chi_{k_1}^2$ и $ \chi_{k_2}^2$
называется распределением Фишера
$$F_{k_1, k_2} =
  \frac{ \chi_{k_1}^2}{ \chi_{k_2}^2}.$$

\addcontentsline{toc}{section}{Аудиторные задачи}
\section*{Аудиторные задачи}

\addcontentsline{toc}{section}{Домашнее задание}
\section*{Домашнее задание}

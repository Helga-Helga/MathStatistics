\addcontentsline{toc}{chapter}{Занятие 3. Метод моментов построения оценок}
\chapter*{Занятие 3. Метод моментов построения оценок}

\addcontentsline{toc}{section}{Контрольные вопросы и задания}
\section*{Контрольные вопросы и задания}

\subsubsection*{Преведиты определение оценки: несмещённой, ассимптотически несмещённой,
                состоятельной, сильно состоятельной, оптимальной.}

Оценка $ \hat{ \theta}$  несмещённая,
если $ \forall \theta \in \Theta: \, M_{ \theta } \hat{ \theta } = \theta $.

Асимптотически несмещенная оценка --- такая оценка,
математическое ожидание которой совпадает с оцениваемым параметром при $n \to \infty $.

Оценка $ \hat{ \theta }$ называется состоятельной,
если стремится к истинному значению $ \theta $ по вероятности
$ \hat{ \theta } \overset{P}{ \rightarrow } \theta, \,
  n \to \infty $.

Оценка $ \hat{ \theta }$ называется сильно состоятельной,
если стремится к истинному значению $ \theta $ почти наверное
$ \hat{ \theta } \overset{a.s.}{ \rightarrow } \theta, \,
  n \to \infty $.

Несмещённая оценка $ \hat{ \theta } \in K$
называется оптимальной в классе квадратично интегрируемых оценок $K$,
если для всякой другой несмещённой оценки
$ \tilde{ \theta } \in \Theta \,
  \forall \theta \in \Theta: \,
  D_{ \theta } \hat{ \theta } \leq D_{ \theta } \tilde{ \theta }$
или же
$ \forall \theta \in \Theta, \,
  M_{ \theta } \left( \hat{ \theta } - \theta \right)^ \leq
  M_{ \theta } \left( \tilde{ \theta } - \theta \right)^2$.

\subsubsection*{Что называется среднеквадратическим отклонением оценки?}

$M_{ \theta } \left( \hat{ \theta } - \theta \right) $ --- среднеквадратическое оклонение.

\subsubsection*{Сформулируйте утверждение про поведение выборочных моментов.}

Выборочный начальный момент $M_k \, k$-го порядка стремится к начальному моменту $ \nu_k$
случайной величины $X$, то есть
$$ \lim \limits_{n \to \infty } P \left( \left| M_k - \nu_k \right| \geq \varepsilon \right) =
  0,$$
для любого сколь угодно малого $ \varepsilon > 0$,
если моменты $ \nu_{2k}$ и $ \nu_k$ случайной величины $X$ существуют и конечны.

\subsubsection*{Сформулируйте основную идею метода моментов построения оценки
                неизвестного параметра.}

$x_1, \dotsc, x_n$ ---
выборка из распределения $F_{ \theta }$ и
$ \theta: \,
  M_{ \theta } f \left( x_1 \right) = g \left( \theta \right) $.

Вычисляем математическое ожидание, считая, что $x_1$ имеет распределение с параметром $ \theta $,
иными словами,
$$M_{ \theta } f \left( x \right) = \int \limits_{ \mathbb{R}} xdF_{ \theta } \left( x \right) , \,
  f, g \in C \left( \mathbb{R} \right), \,
  g$$
--- строго монотонная.
Тогда в силу усиленного закона больших чисел
$$ \theta \approx
  g^{-1} \left( \frac{1}{n} \sum \limits_{k = 1}^n f \left( x_k \right) \right) \overset{a.s.}
  \theta, \,
  n \to \infty.$$

Если существуют непрерывные $f$ и $g, \, g$ ---
обратима и $g \left( \theta \right) = M_{ \theta } f \left( x \right) $,
то в качестве оценки можно выбрать
$$ \hat{ \theta } =
  g^{-1} \left( \int \limits_{ \mathbb{R}} fdF_n \right).$$

\addcontentsline{toc}{section}{Аудиторные задачи}
\section*{Аудиторные задачи}

\subsubsection*{3.3}

\textit{Задание.}
Пользуясь методом моментов, оцените параметр $ \theta $ равномерного распределения на отрезке:
\begin{enumerate}[label=\alph*)]
  \item $ \left[ 0, \theta \right] $;
  \item $ \left[ \theta - 1, \theta + 1 \right] $;
  \item $ \left[ 0, 2 \theta \right] $;
  \item $ \left[ - \theta, \theta \right] $.
\end{enumerate}

\textit{Решение.}
\begin{enumerate}[label=\alph*)]
  \item $X_i \sim U \left( \left[ 0, \theta \right] \right) $.
  Записываем теоретический момент.
  Для равномерного распределения --- это средина отрезка
  $$MX_1 =
    \frac{ \theta }{2}.$$

  Должны приравнять
  $$ \frac{ \theta^*}{2} =
    \overline{X},$$
  откуда $ \theta^* = 2 \overline{X} $;
  \item случайные величины имеют распределение
  $X_i \sim
    U \left( \left[ \theta - 1, \theta + 1 \right] \right) $.

  Вычисляем теоретический момент $MX_1 = \theta $.
  Должны записать, что $ \theta^* = \overline{X}$;
  \item случайные величины имеют распределение
  $X_i \sim
    U \left( \left[ 0, 2 \theta \right] \right) $.

  Вычисляем теоретический момент $MX_1 = \theta $, откуда $ \theta^* = \overline{X}$;
  \item случайные величины имеют распределение
  $X_i \sim
    U \left( \left[ - \theta, \theta \right] \right) $.

  Вычисляем теоретический момент $MX_1 = 0$ --- не подходит,
  потому что не является функцией от $ \theta $.
  Можем вычислить второй момент, который в данном случае совпадает с дисперсией $MX_1^2 = DX_1$.
  Для равномерного распределения
  $$DX_1 =
    \frac{4 \theta^2}{12} =
    \frac{ \theta }{3}.$$
  Должны записать уравнение
  $$ \frac{ \left( \theta^2 \right)^*}{3} =
    \overline{X^2},$$
  откуда $ \left( \theta^2 \right)^* = 3 \overline{X^2}$.
  Извлекая корень, получаем $ \theta^* = \sqrt{3 \overline{X^2}}$.
\end{enumerate}

\subsubsection*{3.4}

\textit{Задание.}
Пусть $X_1, \dotsc, X_n$ --- выборка из распределения Пуассона с параметром $ \lambda $.
Пользуясь методом моментов, постройте оценку параметра $ \lambda $ и убедитесь,
что эта оценка является несмещённой и состоятельной.

\textit{Решение.} Первый теоретический момент $MX_1 = \lambda $.
Должны приравнять
$$ \lambda^* =
  \frac{1}{n} \sum \limits_{i = 1}^n X_i.$$

По закону больших чисел $ \lambda^* \overset{P}{ \to } \lambda $.
Отсюда следует состоятельность.

$M \lambda^* = MX_1 = \lambda $.
Отсюда следует несмещённость.

\subsubsection*{3.5}

\textit{Задание.} Пользуясь методом моментов с пробной функцией $g \left( y \right) = y$,
оцените параметр сдвига $ \beta \in \mathbb{R}$ показательного распределения с плотностью
$$f_{ \beta } \left( y \right) =
  \begin{cases}
    e^{ \beta - y}, \qquad t \geq \beta, \\
    0, \qquad y < \beta.
  \end{cases}$$

\textit{Решение.} $Mg \left( X_1 \right) = \overline{g \left( X \right) }$ --- метод моментов.

Ищем теоретический момент, зная плотность распределения
$$MX_1 =
  \int \limits_{ \beta }^{ \infty } ye^{ \beta - y} dy.$$
Делаем замену $y - \beta = z$ и получаем
$$ \int \limits_{ \beta }^{ \infty } ye^{ \beta - y} dy =
  \int \limits_0^{ \infty } \left( \beta + z \right) e^{-z} dz =
  \beta + 1.$$

Отсюда должны решить уравение $ \beta^* + 1 = \overline{X}$.

Отсюда $ \beta^* = \overline{X} - 1$.
Эта оценка несмещённая и состоятельная.

\addcontentsline{toc}{section}{Домашнее задание}
\section*{Домашнее задание}

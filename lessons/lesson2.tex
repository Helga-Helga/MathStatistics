\addcontentsline{toc}{chapter}{Занятие 2. Свойства оценок}
\chapter*{Занятие 2. Свойства оценок}

\addcontentsline{toc}{section}{Контрольные вопросы и задания}
\section*{Контрольные вопросы и задания}

\subsubsection*{Что называют оценкой неизвестного параметра?}

Статистику, значение которой заменяет неизвестный параметр, называют оценкой этого параметра.

\subsubsection*{Преведиты определение оценки: несмещённой, ассимптотически несмещённой,
                состоятельной, сильно состоятельной, оптимальной.}

Оценка $ \hat{ \theta}$  несмещённая,
если $ \forall \theta \in \Theta: \, M_{ \theta } \hat{ \theta } = \theta $.

Асимптотически несмещенная оценка --- такая оценка,
математическое ожидание которой совпадает с оцениваемым параметром при $n \to \infty $.

Оценка $ \hat{ \theta }$ называется состоятельной,
если стремится к истинному значению $ \theta $ по вероятности
$ \hat{ \theta } \overset{P}{ \rightarrow } \theta, \,
  n \to \infty $.

Оценка $ \hat{ \theta }$ называется сильно состоятельной,
если стремится к истинному значению $ \theta $ почти наверное
$ \hat{ \theta } \overset{a.s.}{ \rightarrow } \theta, \,
  n \to \infty $.

Несмещённая оценка $ \hat{ \theta } \in K$
называется оптимальной в классе квадратично интегрируемых оценок $K$,
если для всякой другой несмещённой оценки
$ \tilde{ \theta } \in \Theta \,
  \forall \theta \in \Theta: \,
  D_{ \theta } \hat{ \theta } \leq D_{ \theta } \tilde{ \theta }$
или же
$ \forall \theta \in \Theta, \,
  M_{ \theta } \left( \hat{ \theta } - \theta \right)^ \leq
  M_{ \theta } \left( \tilde{ \theta } - \theta \right)^2$.

\subsubsection*{Что называется среднеквадратическим отклонением оценки?}

$M_{ \theta } \left( \hat{ \theta } - \theta \right) $ --- среднеквадратическое оклонение.

\subsubsection*{Сформулируйте утверждение про поведение выборочных моментов.}

\subsubsection*{Какая оценка является несмещённой и содержательной для математического ожидания
                распределения выборки?}

\subsubsection*{Какая статистика является несмещённой оценкой для дисперсии распределения выборки?}

$$ \frac{1}{n - 1} \sum \limits_{k = 1}^n \left( x_k - \overline{x} \right)^2$$
--- несмещённая оценка для $ \sigma^2 = Dx_1$.

\addcontentsline{toc}{section}{Аудиторные задачи}
\section*{Аудиторные задачи}

\addcontentsline{toc}{section}{Домашнее задание}
\section*{Домашнее задание}

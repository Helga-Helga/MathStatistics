\addcontentsline{toc}{chapter}{Занятие 5. Достаточные статистики}
\chapter*{Занятие 5. Достаточные статистики}

\addcontentsline{toc}{section}{Контрольные вопросы и задания}
\section*{Контрольные вопросы и задания}

\subsubsection*{Приведите определение условного распределения, определения достаточной статистики.}

$P \left( \xi \in \Delta \; \middle| \; \mathcal{F'} \right) =
  M \left[ \mathbbm{1}_{ \Delta } \left( \xi \right) \; \middle| \; \mathcal{F'} \right] $.

Статистика $T$ называется достаточной для параметра $ \theta $,
если условное распределение при известном $T$ не зависит от параметра $ \theta $.

\subsubsection*{Сформулируйте теорему про характеризацию достаточной статистики.}

Пусть $x_1, \dotsc, x_n$ ---
выборка из распределения с плотностью
$$p \left( x, \theta \right), \,
  \theta \in \Theta.$$

Статистика $T$ является достаточной тогда и только тогда,
когда функция правдоподобия $L \left( \vec{x}, \theta \right) $ допускает факторизацию,
то есть может быть представлена произведением двух функций следующего вида
$$L \left( \vec{x}, \theta \right) =
  h \left( T, \theta \right) \cdot g \left( \vec{x} \right).$$

\addcontentsline{toc}{section}{Аудиторные задачи}
\section*{Аудиторные задачи}

\addcontentsline{toc}{section}{Домашнее задание}
\section*{Домашнее задание}

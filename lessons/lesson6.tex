\addcontentsline{toc}{chapter}{Занятие 6. Теорема Колмогорова про улучшение оценок.
                              Неравенство Рао-Крамера}
\chapter*{Занятие 6. Теорема Колмогорова про улучшение оценок. Неравенство Рао-Крамера}

\addcontentsline{toc}{section}{Контрольные вопросы и задания}
\section*{Контрольные вопросы и задания}

\subsubsection*{Приведите определение достаточной статистики.}

Статистика $T$ называется достаточной для параметра $ \theta $,
если условное распределение при известном $T$ не зависит от параметра $ \theta $.

\subsubsection*{Сформулируйте теорему про характеризацию достаточной статистики.}

Пусть $x_1, \dotsc, x_n$ ---
выборка из распределения с плотностью
$$p \left( x, \theta \right), \,
  \theta \in \Theta.$$

Статистика $T$ является достаточной тогда и только тогда,
когда функция правдоподобия $L \left( \vec{x}, \theta \right) $ допускает факторизацию,
то есть может быть представлена произведением двух функций следующего вида
$$L \left( \vec{x}, \theta \right) =
  h \left( T, \theta \right) \cdot g \left( \vec{x} \right).$$

\subsubsection*{Сформулируйте теорему Колмогорова про улучшение
                оценки с помощью достаточной статистики.}

Оптимальная оценка единственная (в том случае, когда она существует).

\subsubsection*{Что называется количеством информации Фишера?}

$$D_{ \theta } U \left( \vec{x}, \theta \right) =
  nD_{ \theta } \left(
    \frac{ \partial }{ \partial \theta } ln \, p \left( x, \theta \right)
  \right) =
  nI_0,$$
где $nI_0$ --- количество информации Фишера.

\subsubsection*{Запишите неравенство Рао-Крамера.}

Пусть $ \hat{ \theta }$ --- несмещённая оценка для параметра $ \theta $.
Тогда
$$ \forall \theta \in \Theta: \,
  D_{ \theta } \hat{ \theta } \geq \frac{1}{nI_0}.$$
Как угодно лучшую оценку взять нельзя.

\subsubsection*{Какая оценка называется эффективной?}

Если $ \hat{ \theta }$ такова, что $ \hat{ \theta }$ --- несмещённая и
$$D_{ \theta } \hat{ \theta } =
  \frac{1}{nI_0}, \,
  \theta \in \Theta,$$
то $ \hat{ \theta }$ называется эффективной оценкой.

\subsubsection*{Приведите определение экспоненциальной семьи распределений.}

\addcontentsline{toc}{section}{Аудиторные задачи}
\section*{Аудиторные задачи}

\subsubsection*{6.3}

\textit{Задание.}
Пусть $X_1, \dotsc, X_n$ ---
выборка из равномерного распределения на отрезке $ \left[ 0, \theta \right].$
Найдите несмещённую оценку неизвестного параметра
$ \tau \left( \theta, y \right) =
  P \left( X_1 > y \right) $
и улучшите её усреднением по достаточной для параметра $ \theta $ статистике.

\textit{Решение.} $X_{ \left( n \right) }$ --- достаточная статистика для $ \theta $.

Нужно найти $ \hat{ \tau }$ из условия, что она была бы несмещённой,
что
$$M \hat{ \tau } =
  \tau =
  P \left( X_1 > y \right).$$
Берём в качестве $ \hat{ \tau } = \mathbbm{1} \left\{ X_1 > y \right\} $ ---
это несмещённая оценка для $ \tau $.
Теперь улучшаем эту оценку.
$ \tau^* =
  M \left( \hat{ \tau } \; \middle| \; X_{ \left( n \right) } \right) =
  M \left\{ \mathbbm{1} \left\{ X_1 < y \right\} \; \middle| \; X_{ \left( n \right) } \right\}.$
Индикатор может принимать значения 0 и 1,
значит
\begin{equation*}
  \begin{split}
    M \left\{
      \mathbbm{1} \left\{ X_1 < y \right\} \; \middle| \; X_{ \left( n \right) }
    \right\} = \\
    = 0 \cdot
    P \left(
      \mathbbm{1} \left\{ X_1 > y \right\} = 0 \; \middle| \; X_{ \left( n \right) }
    \right) +
    1 \cdot
    P \left( \mathbbm{1} \left\{ X_1 > y \right\} = 1 \; \middle| \; X_{ \left( n \right) } \right).
  \end{split}
\end{equation*}
Первое слагаемое пропадает
\begin{equation*}
  \begin{split}
    0 \cdot
    P \left(
      \mathbbm{1} \left\{ X_1 > y \right\} = 0 \; \middle| \; X_{ \left( n \right) }
    \right) +
    1 \cdot
    P \left(
      \mathbbm{1} \left\{ X_1 > y \right\} = 1 \; \middle| \; X_{ \left( n \right) }
    \right) = \\
    = P \left( X_1 > y \; \middle| X_{ \left( n \right) } \right) =
    f \left( X_{ \left( n \right) } \right).
  \end{split}
\end{equation*}
Ищем $f \left( z \right) = P \left( X_1 > y \; \middle| \; X_{ \left( n \right) } = z \right).$
Выборка имеет равномерное распределение, то есть распределение, которое имеет плотность.
Вероятность попадания в точку равна нулю
$$ P \left( X_1 > y \; \middle| \; X_{ \left( n \right) } = z \right) =
  \lim \limits_{h \to 0}
    P \left\{ X_1 > y \; \middle| \; X_{ \left( n \right) } \in \left[ z, z + h \right] \right\}.$$
Воспользуемся определением условной вероятности
$$ \lim \limits_{h \to 0}
    P \left\{ X_1 > y \; \middle| \; X_{ \left( n \right) } \in \left[ z, z + h \right] \right\} =
  \lim \limits_{h \to 0}
    \frac{P \left\{ X_1 > y, X_{ \left( n \right) } \in \left[ z, z + h \right] \right\} }{P \left\{ X_{ \left( n \right) } \in \left[ z, z + h \right] \right\} }.$$
Это нужно рассматривать при $y < z$.
Разбиваем на разность двух вероятностей в числителе и знаменателе
\begin{equation*}
  \begin{split}
    \lim \limits_{h \to 0}
      \frac{P \left\{ X_1 > y, X_{ \left( n \right) } \in \left[ z, z + h \right] \right\} }{P \left\{ X_{ \left( n \right) } \in \left[ z, z + h \right] \right\} } = \\
    = \lim \limits_{h \to 0} \left[
      \frac{P \left\{ X_1 > y, X_{ \left( n \right) } \leq z + h \right\} }{P \left\{ X_{ \left( n \right) } \leq z + h \right\} - P \left\{ X_{ \left( n \right) } \leq z \right\} } -
      \frac{P \left\{ X_1 > y, X_{ \left( n \right) } \leq z \right\} }{P \left\{ X_{ \left( n \right) } \leq z + h \right\} - P \left\{ X_{ \left( n \right) } \leq z \right\} }
    \right].
  \end{split}
\end{equation*}
Переходим к функции распределения выборки
\begin{equation*}
  \begin{split}
    \lim \limits_{h \to 0} \left[
      \frac{P \left\{ X_1 > y, X_{ \left( n \right) } \leq z + h \right\} }{P \left\{ X_{ \left( n \right) } \leq z + h \right\} - P \left\{ X_{ \left( n \right) } \leq z \right\} } -
      \frac{P \left\{ X_1 > y, X_{ \left( n \right) } \leq z \right\} }{P \left\{ X_{ \left( n \right) } \leq z + h \right\} - P \left\{ X_{ \left( n \right) } \leq z \right\} }
    \right] = \\
    = \lim \limits_{h \to 0} \left\{
      \frac{ \left[ F \left( z + h \right) - F \left( y \right) \right] \left[ F \left( z + h \right) \right]^{n - 1}}{ \left[ F \left( z + h \right) \right]^n - \left[ F \left( z \right) \right]^n} -
      \frac{ \left[ F \left( z \right) - F \left( y \right) \right] \left[ F \left( z \right) \right]^{n - 1}}{ \left[ F \left( z + h \right) \right]^n - \left[ F \left( z \right) \right]^n}
    \right\}.
  \end{split}
\end{equation*}

Знаменатель и числитель стремятся к нулю.
Дифференцируем по $h$ числитель и знаменатель.
Это правило Лопиталя
\begin{equation*}
  \begin{split}
    \lim \limits_{h \to 0} \left\{
      \frac{ \left[ F \left( z + h \right) - F \left( y \right) \right] \left[ F \left( z + h \right) \right]^{n - 1}}{ \left[ F \left( z + h \right) \right]^n - \left[ F \left( z \right) \right]^n} -
      \frac{ \left[ F \left( z \right) - F \left( y \right) \right] \left[ F \left( z \right) \right]^{n - 1}}{ \left[ F \left( z + h \right) \right]^n - \left[ F \left( z \right) \right]^n}
    \right\} = \\
    = \lim \limits_{h \to 0} \left\{
      \frac{p \left( z + h \right) \left[ F \left( z + h \right) \right]^{n - 1}}{n \left[ F \left( z + h \right) \right]^{n - 1} p \left( z + h \right) } +
      \frac{ \left( n + 1 \right) \left[ F \left( z + h \right) \right]^{n - 2} p \left( z + h \right) \left[ F \left( z + h \right) - F \left( y \right) \right] }{n \left[ F \left( z + h \right) \right]^{n - 1} p \left( z + h \right) }
    \right\} = \\
    = \lim \limits_{h \to 0}
      \frac{F \left( z + h \right) + \left( n - 1 \right) \left[ F \left( z + h \right) - F \left( y \right) \right] }{nF \left( z + h \right) }.
  \end{split}
\end{equation*}
Переходим к пределу
$$ \lim \limits_{h \to 0}
    \frac{F \left( z + h \right) + \left( n - 1 \right) \left[ F \left( z + h \right) - F \left( y \right) \right] }{nF \left( z + h \right) } =
  \frac{F \left( z \right) + \left( n + 1 \right) \left[ F \left( z \right) - F \left( y \right) \right] }{nF \left( z \right) }.$$
Приводим подобные
$$ \frac{F \left( z \right) + \left( n + 1 \right) \left[ F \left( z \right) - F \left( y \right) \right] }{nF \left( z \right) } =
  \frac{nF \left( z \right) - \left( n - 1 \right) F \left( y \right) }{nF \left( z \right) }.$$
Почленно поделим числитель на знаменатель
$$ \frac{nF \left( z \right) - \left( n - 1 \right) F \left( y \right) }{nF \left( z \right) } =
  1 - \frac{n - 1}{n} \cdot \frac{F \left( y \right) }{F \left( z \right) }.$$
Подставляем значения функции равномерного распределения на отрезке $ \left[ 0, \theta \right] $.
Получаем
$$1 - \frac{n - 1}{n} \cdot \frac{F \left( y \right) }{F \left( z \right) } =
  1 - \frac{n - 1}{n} \cdot \frac{y}{z}.$$

Окончательно
$$f \left( X_{ \left( n \right) } \right) =
  1 - \frac{n - 1}{n} \cdot \frac{y}{X_{ \left( n \right) }}.$$

\addcontentsline{toc}{section}{Домашнее задание}
\section*{Домашнее задание}

\addcontentsline{toc}{chapter}{Контрольная работа}
\chapter*{Контрольная работа}

\addcontentsline{toc}{section}{Вариант 1}
\section*{Вариант 1}

\subsubsection*{1}

\textit{Задание.} На таможню ежедневно приходит в среднем 10 грузовиков.
Среднее квадратическое отклонение количества грузовиков равно 3.
Используя центральную предельную теорему, найдите такое $x$, что вероятность того,
что за год через таможню пройдёт больше чем $x$ грузовиков, примерно равна $0.99$.

\textit{Решение.}
Обозначим через $ \xi_n$ количество грузовиков, которые прибыли на таможню в $n$-ый день года.
Тогда $S = \xi_1 + \dotsc + \xi_{365}$ является количеством грузовиков,
которые прибыли на таможню за год.
Чтобы найти такое $x$, что $P \left( S > x \right) \approx 0.99$,
воспользуемся аппроксимацией нормальным распределением:
$$P \left( S > x \right) =
  P \left( \frac{S - MS}{ \sqrt{DS}} > \frac{x - MS}{ \sqrt{DS}} \right) \approx
  \Phi_t \left( \frac{x - MS}{ \sqrt{DS}} \right),$$
где $ \Phi_t$ --- функция распределения стандартного нормального распределения.
Значит $x$ находим из условия
$$ \Phi_t \left( \frac{MS - x}{ \sqrt{DS}} \right) \approx
  0.01.$$
Воспользовавшись таблицей значений функции $ \Phi_t$, находим, что
$$ \frac{MS - x}{ \sqrt{DS}} \approx
  2.32.$$
По условию $MS = 365 \cdot M \xi_1 = 3650, \, \sqrt{DS} = \sqrt{365 \cdot D \xi_1} = 3 \sqrt{365}$
и значит $-x + 3650 \approx 2.32 \cdot 3 \sqrt{365} \approx 132.97$,
откуда $x = -132.97 + 3650 \approx 2517$.

\subsubsection*{2}

\textit{Задание.}
Для произвольного действительного числа $y$ вычислите дисперсию $DF_n^* \left( y \right) $,
где $F_n^*$ --- эмпирическая функция распределения, построенная по выборке из распределения $F$.

\textit{Решение.}
$$F_n^* \left( y \right) =
  \frac{1}{n} \sum \limits_{k = 1}^n \mathbbm{1} \left\{ X_k \leq y \right\}, \,
  y \in \mathbb{R}.$$

Ищем дисперсию
$$DF_n^* \left( y \right) =
  D \left( \frac{1}{n} \sum \limits_{k = 1}^n \mathbbm{1} \left\{ X_k \leq y \right\} \right) =
  \frac{1}{n^2} \cdot D \sum \limits_{k = 1}^n \mathbbm{1} \left\{ X_k \leq y \right\}.$$
Вносим дисперсию под знак суммы, пользуясь независимостью случайных величин, а затем используем то,
что они имеют одинаковое распределение
$$ \frac{1}{n^2} \cdot D \sum \limits_{k = 1}^n \mathbbm{1} \left\{ X_k \leq y \right\} =
  \frac{1}{n^2} \sum \limits_{k = 1}^n D \mathbbm{1} \left\{ X_k \leq y \right\} =
  \frac{1}{n^2} \cdot nD \mathbbm{1} \left\{ X_1 \leq y \right\}.$$
Сокращаем константы и пользуемся свойством дисперсии
$$ \frac{1}{n^2} \cdot nD \mathbbm{1} \left\{ X_1 \leq y \right\} =
  \frac{1}{n} \cdot D \mathbbm{1} \left\{ X_1 \leq y \right\} =
  \frac{1}{n} \left[ F \left( y \right) - F^2 \left( y \right) \right] =
  \frac{1}{n} \cdot F \left( y \right) \left[ 1 - F \left( y \right) \right],$$
так как
$M \mathbbm{1} \left\{ X_1 \leq y \right\} =
  1 \cdot P \left( X_1 \leq y \right) + 0 \cdot P \left( X_1 > y \right) =
  P \left( X_1 \leq y \right) =
  F \left( y \right) $,
и $M \left( \mathbbm{1} \left\{ X_1 \leq y \right\} \right)^2 = F \left( y \right) $.

\subsubsection*{3}

\textit{Задание.}
Пусть $X_1, \dotsc, X_n$ --- выборка из нормального распределения $N \left( a, 1 \right) $.
Проверьте несмещённость и состоятельность оценки
$$ \hat{a_n} =
  \frac{ \sum \limits_{i = n + 1}^{2n} X_i}{n}$$
параметра $a$.
Найдите среднеквадратическое отклонение данной оценки от истинного значения параметра $a$.

\textit{Решение.} Проверим несмещённость оценки.
Проверим, выполняется ли $M \hat{a_n} = a$.

Нужно найти математическое ожидание
$$M \frac{ \sum \limits_{i = n + 1}^{2n} X_i}{n} =
  \frac{1}{n} \cdot M \sum \limits_{i = n + 1}^{2n} X_i =
  \frac{1}{n} \sum \limits_{i = n + 1}^{2n} MX_i =
  \frac{1}{n} \left[ 2n - \left( n + 1 \right) + 1 \right] MX_1.$$
Раскрываем скобки
$$ \frac{1}{n} \left[ 2n - \left( n + 1 \right) + 1 \right] MX_1 =
  \frac{1}{n} \left( 2n - n - 1 + 1 \right) MX_1 =
  \frac{1}{n} \cdot na =
  a.$$

Отсюда следует, что оценка несмещённая.

Чтобы дать ответ на второй вопрос задачи,
нужно найти предел по вероятности статистики $ \hat{a_n}$ при $n \to \infty $.
Поскольку случайные величины $X_1, \dotsc, X_n$
являются независимыми одинаково распределёнными случайными величинами с конечным первым моментом,
то согласно закону больших чисел
$$ \frac{1}{n} \sum \limits_{i = n + 1}^{2n} X_i \overset{P}{ \rightarrow } MX_1, \,
  n \to \infty,$$
а тогда при $n \to \infty: \, \hat{a_n} \overset{P}{ \rightarrow } MX_1 = a$.

Значит статистика $ \hat{a_n}$ является состоятельной оценкой параметра $a$.

Формула среднеквадратического отклонения имеет вид $ \sigma = \sqrt{D \xi }$.

Найдём дисперсию оценки
$$D \frac{ \sum \limits_{i = n + 1}^{2n} X_i}{n} =
  \frac{1}{n^2} \cdot D \sum \limits_{i = n + 1}^{2n} X_i =
  \frac{1}{n^2} \cdot nDX_1 =
  \frac{1}{n} \cdot DX_1 =
  \frac{1}{n}.$$

Извлекая корень, получаем
$$ \sigma =
  \sqrt{ \frac{1}{n}} =
  \frac{1}{ \sqrt{n}}.$$

\subsubsection*{3}

\textit{Задание.} Пусть $X_1, \dotsc, X_n$ --- выборка из распределения Бернулли.
Является ли статистика $p_n^* = \left( \overline{X} \right)^2$ несмещённой оценкой параметра $p$?
Состоятельной?

\textit{Решение.}
Для того, чтобы выяснить, является ли статистиа $p_n^*$ несмещённой оценкой параметра $p$,
необходимо проверить, выполняется ли условие $Mp_n^* = p$.
Находим математическое ожидание статистики $p_n^*:$
$$Mp_n^* =
  M \left( \overline{X} \right)^2 =
  M \left( \frac{1}{n} \sum \limits_{i = 1}^n X_i \right)^2 =
  \frac{1}{n^2} \cdot M \left( \sum \limits_{i = 1}^n X_i \right)^2.$$
По свойствам дисперсии
$$ \frac{1}{n^2} \cdot M \left( \sum \limits_{i = 1}^n X_i \right)^2 =
  \frac{1}{n^2} \cdot D \sum \limits_{i = 1}^n X_i +
  \frac{1}{n^2} \cdot \left( M \sum \limits_{i = 1}^n X_i \right)^2.$$
Воспользуемся тем, что случайные величины независимы и одинаково распределены
$$ \frac{1}{n^2} \cdot D \sum \limits_{i = 1}^n X_i +
  \frac{1}{n^2} \cdot \left( M \sum \limits_{i = 1}^n X_i \right)^2 =
  \frac{1}{n^2} \cdot nDX_1 + \frac{1}{n^2} \cdot \left( nMX_1 \right)^2.$$
Сокращаем константы
$$ \frac{1}{n^2} \cdot nDX_1 + \frac{1}{n^2} \cdot \left( nMX_1 \right)^2 =
  \frac{1}{n} \cdot DX_1 + \left( MX_1 \right)^2 =
  \frac{1}{n} \cdot \sqrt{p} \left( 1 - \sqrt{p} \right) + p.$$

Таким образом
$$Mp_n^* =
  \frac{1}{n} \cdot \sqrt{p} \left( 1 - \sqrt{p} \right) + p \neq
  p,$$
а значит статистика $p_n^*$ не является несмещённой оценкой неизвесного параметра $p$.

Чтобы дать ответ на второй вопрос задачи,
нужно найти предел по вероятности статистики $p_n^*$ при $n \to \infty $.
Поскольку случайные величины $X_1, \dotsc, X_n$
являются независимыми одинаково распределёнными случайными величинаим с конечным первым моментом,
то согласно закону больших чисел
$$ \frac{1}{n} \sum \limits_{i = 1}^n X_i \overset{P}{ \rightarrow } MX_1, \,
  n \to \infty,$$
а тогда при $n \to \infty: \, p_n^* \overset{P}{ \rightarrow } \left( MX_1 \right)^2 = p$.
Значит статистика $p_n^*$ является состоятельной оценкой параметра $p$.


\subsubsection*{5}

\textit{Задание.}
Найдите оценку максимального правдоподобия параметра
$$ \theta >
  0,$$
если распределение выборки имеет плотность
$$p \left( y \right) =
  \frac{2y}{ \theta } \cdot \mathbbm{1} \left\{ y \in \left[ 0, \theta \right] \right\}.$$

\textit{Решение.} Согласно методу максимального правдоподобия,
в качестве оценки неизвестного параметра $ \theta $ выбирается статистика $ \theta^*$,
которая даёт наибольшее значение функции правдоподобия, которая для данной выборки имеет вид
$$L \left( X_1, \dotsc, X_n, \theta \right) =
  \prod \limits_{i = 1}^n
    \frac{2X_i}{ \theta } \cdot \mathbbm{1} \left\{ X_i \in \left[ 0, \theta \right] \right\} =
  \frac{2^n \prod \limits_{i = 1}^n X_i}{ \theta^{n}} \cdot
  \mathbbm{1}
    \left\{ X_{ \left( 1 \right) } \geq 0, \, X_{ \left( n \right) } \leq \theta \right\}.$$

Заметим, что функция правдоподобия не дифференцируема по $ \theta $.

Из  приведённого на рисунке \ref{fig:5} графика функции $L$ как функции переменной $ \theta $
делаем вывод, что максимального значения она достигает в точке $X_{ \left( n \right) }$.
Значит оценкой максимального правдоподобия является статистика $ \theta^* = X_{ \left( n \right) }$.

\begin{figure}[h!]
  \centering
  \includegraphics[width=.4\textwidth]{./pictures/v1_5.png}
  \caption{Функция правдоподобия}
  \label{fig:5}
\end{figure}

\subsubsection*{6}

\textit{Задание.}
Пусть $X_1, \dotsc, X_n$ ---
выборка из равномерного распределения на отрезке $ \left[ a, b \right], \, a < b$.
Пользуясь методом моментов, постройте оценку векторного параметра $ \left( a, b \right) $.

\textit{Решение.}
Метод моментов в качестве оценки неизвестного параметра $ \left( a, b \right) $
предлагает брать решение $ \left( a^*, b^* \right) $ системы уравнений
$$ \begin{cases}
    m_1 \left( a^*, b^* \right) = \overline{X}, \\
    m_2 \left( a^*, b^* \right) = \overline{X^2},
  \end{cases}$$
где $m_1 \left( a, b \right) = MX_1, \, m_2 \left( a, b \right) = MX_1^2$ ---
моменты заданного распределения.

Находим первые два момента элементов выборки:
$$MX_1 = \frac{a + b}{2}, \,
  MX_1^2 =
  DX_1 + \left( MX_1 \right)^2 =
  \frac{ \left( b - a \right)^2}{12} + \frac{ \left( a + b \right)^2}{4}.$$
Раскроем скобки и приведём к общему знаменателю
$$ \frac{ \left( b - a \right)^2}{12} + \frac{ \left( a + b \right)^2}{4} =
  \frac{b^2 - 2ab + a^2}{12} + \frac{3a^2 + 6ab + 3b^2}{12} =
  \frac{4b^2 + 4ab + 4a^2}{12}.$$
Сокращаем на 4 и получаем второй момент
$$ \frac{4b^2 + 4ab + 4a^2}{12} =
  \frac{a^2 + ab + b^2}{3}.$$

Решая систему уравнений
$$ \begin{cases}
    a + b = 2 \overline{X}, \\
    a^2 + ab + b^2 = 3 \overline{X^2},
  \end{cases}$$
находим, что
$$ \begin{cases}
    a + b = 2 \overline{X}, \\
    ab = 4 \left( \overline{X} \right)^2 - 3 \overline{X^2},
  \end{cases}$$
откуда $2 \overline{X} b - b^2 = 4 \left( \overline{X} \right)^2 - 3 \overline{X^2}$ или
$b^2 - 2 \overline{X} b + 4 \left( \overline{X} \right)^2 - 3 \overline{X^2} = 0$.
Выделяем полный квадрат
$ \left( b - \overline{X} \right)^2 + 3 \left( \overline{X} \right)^2 - 3 \overline{X}^2 =
  0$.
Берём корень
$$b - \overline{X} =
  \sqrt{3 \overline{X}^2 - 3 \left( \overline{X} \right)^2},$$
тогда $b^* = \overline{X} + \sqrt{3 \overline{X}^2 - 3 \left( \overline{X} \right)^2}$.

Соответственно
$a^* =
  2 \overline{X} - \overline{X} - \sqrt{3 \overline{X}^2 - 3 \left( \overline{X} \right)^2} =
  \overline{X} - \sqrt{3 \overline{X}^2 - 3 \left( \overline{X} \right)^2}$.


\subsubsection*{7}

\textit{Задание.}
Пусть $X_1, \dotsc, X_n$ --- выборка из смещённого показательного распределения с плотностью
$$f_{ \beta } \left( y \right) =
  \begin{cases}
    e^{ \beta - y}, \qquad y \geq \beta, \\
    0, \qquad y < \beta.
  \end{cases}$$
Предъявите достаточную статистику для оценки параметра $ \beta $
и вычислите её математическое ожидание.

\textit{Решение.} Запишем функцию правдоподобия данной выборки:
$$L \left( X_1, \dotsc, X_n, \beta \right) =
  \prod \limits_{i = 1}^n f_{ \beta } \left( X_i \right) =
  \prod \limits_{i = 1}^n e^{ \beta - X_i} \cdot \mathbbm{1} \left\{ X_i \leq \beta \right\}.$$
Раскроем произведение
$$ \prod \limits_{i = 1}^n e^{ \beta - X_i} \cdot \mathbbm{1} \left\{ X_i \leq \beta \right\} =
  e^{ \sum \limits_{i = 1}^n \left( \beta - X_i \right) } \cdot
  \mathbbm{1} \left\{ X_{ \left( 1 \right) } \geq \beta \right\} =
  e^{n \left( \beta - \overline{X} \right) } \cdot
  \mathbbm{1} \left\{ X_{ \left( 1 \right) } \geq \beta \right\}.$$
Разобьём экспоненту на две
$$e^{n \left( \beta - \overline{X} \right) } \cdot
  \mathbbm{1} \left\{ X_{ \left( 1 \right) } \geq \beta \right\} =
  e^{n \beta } e^{-n \overline{X}} \cdot
  \mathbbm{1} \left\{ X_{ \left( 1 \right) } \geq \beta \right\}.$$

Видим,
что функция правдоподобия подаётся в виде произведения функции
$h \left( X_1, \dotsc, X_n \right) = e^{-n \overline{X}}$,
которая не зависит от неизвестного параметра,
и функции
$g \left( X_1, \dotsc, X_n, \beta \right) =
  e^{n \beta } \cdot \mathbbm{1} \left\{ X_{ \left( 1 \right) } \geq \beta \right\} $,
которая зависит от неизвестного параметра и от функции выборки
$T \left( X_1, \dotsc, X_n \right) =
  X_{ \left( 1 \right) }$.

Тогда, по теореме про характеризацию достаточной статистики,
функция $T \left( X_1, \dotsc, X_n \right) = X_{ \left( 1 \right) }$
является достаточной статистикой параметра $ \beta $.

Найдём функцию распределения порядковой статистики
$$F_{X_{ \left( 1 \right) }} \left( y \right) =
  P \left( X_{ \left( 1 \right) } \leq y \right) =
  P \left\{ \min \left( X_1, \dotsc, X_n \right) \leq y \right\} =$$
$=P$(хотя бы один элемент выборки не превышает $y$) $=$
\begin{equation*}
  \begin{split}
    = \sum \limits_{k = 1}^n
      P \left( exactly \, k \, elements \, of \, the \, set \, dot'n \, exceed \, y \right) = \\
    = \sum \limits_{k = 1}^n
      C_n^k F^k \left( y \right) \left[ 1 - F \left( y \right) \right]^{n - k} = \\
    = \sum \limits_{k = 0}^n
      C_n^k F^k \left( y \right) \left[ 1 - F \left( y \right) \right]^{n - k} -
    C_n^0 F^0 \left( y \right) \left[ 1 - F \left( y \right) \right]^{n - 0}.
  \end{split}
\end{equation*}
Применяем формулу для бинома Ньютона
\begin{equation*}
  \begin{split}
    \sum \limits_{k = 0}^n
      C_n^k F^k \left( y \right) \left[ 1 - F \left( y \right) \right]^{n - k} -
    C_n^0 F^0 \left( y \right) \left[ 1 - F \left( y \right) \right]^{n - 0} = \\
    = \left[ F \left( y \right) + 1 - F \left( y \right) \right]^n -
    \left[ 1 - F \left( y \right) \right]^n =
    1 - \left[ 1 - \int \limits_{- \infty }^y f \left( x \right) dx \right]^n = \\
    = 1 -
    \left(
      1 -
      \int \limits_{- \infty }^y e^{ \beta - x} \cdot \mathbbm{1} \left\{ x \geq \beta \right\} dx
    \right)^n =
    1 - \left( 1 - \int \limits_{ \beta }^y e^{ \beta - x} dx \right)^n = \\
    = 1 - \left( 1 - e^{ \beta } \int \limits_{ \beta }^y e^{-x} dx \right)^n =
    1 - \left( 1 + \left. e^{ \beta } e^{-x} \right|_{ \beta }^y \right)^n =
    1 - \left( 1 + e^{ \beta - y} - e^{ \beta } e^{- \beta } \right)^n = \\
    = 1 - \left( 1 + e^{ \beta - y} - 1 \right)^n =
    1 - \left( e^{ \beta - y} \right)^n =
    1 - e^{n \beta - ny}.
  \end{split}
\end{equation*}

Продифференцируем
$$f_{X_{ \left( 1 \right) }} \left( y \right) =
  \frac{ \partial F_{X_{ \left( 1 \right) }} \left( y \right) }{ \partial y} =
  \frac{ \partial }{ \partial y} \left( 1 - e^{n \beta } e^{-ny} \right) =
  -e^{n \beta } \cdot \frac{ \partial }{ \partial y} e^{-ny}.$$
Берём производную от сложной функции
$$-e^{n \beta } \cdot \frac{ \partial }{ \partial y} e^{-ny} =
  -e^{n \beta } \left( -n \right) e^{-ny} =
  ne^{ n \beta - ny}, \,
  y \in \left[ \beta, + \infty \right).$$

Найдём математическое ожидание  по определению
$$MX_{ \left( 1 \right) } =
  \int \limits_{ \beta }^{+ \infty } yf_{X_{ \left( 1 \right) }} \left( y \right) dy =
  \int \limits_{ \beta }^{+ \infty } yne^{n \beta - ny} dy =
  ne^{n \beta } \int \limits_{ \beta }^{+ \infty } ye^{-ny} dy.$$
Берём интеграл по частям
$$u = y, \,
  dv = e^{-ny} dy, \,
  du = dy, \,
  c = - \frac{1}{n} e^{-ny}.$$
Получаем
$$ne^{n \beta } \int \limits_{ \beta }^{+ \infty } ye^{-ny} dy =
  ne^{n \beta } \left(
    \left. -y \cdot \frac{1}{n} \cdot e^{-ny} \right|_{ \beta }^{+ \infty } +
    \int \limits_{ \beta }^{+ \infty } \frac{1}{n} \cdot e^{-ny} dy
  \right).$$
Подставляем пределы интегрирования и берём интеграл
\begin{equation*}
  \begin{split}
    ne^{n \beta } \left(
      \left. -y \cdot \frac{1}{n} \cdot e^{-ny} \right|_{ \beta }^{+ \infty } +
      \int \limits_{ \beta }^{+ \infty } \frac{1}{n} \cdot e^{-ny} dy
    \right) = \\
    = ne^{n \beta } \left(
      \beta \cdot \frac{1}{n} \cdot e^{-n \beta } -
      \left. \frac{1}{n^2} \cdot e^{-ny} \right|_{ \beta }^{+ \infty }
    \right) =
    \beta + \frac{1}{n} \cdot e^{n \beta } e^{-n \beta } =
    \beta + \frac{1}{n}.
  \end{split}
\end{equation*}

\subsubsection*{8}

\textit{Задание.}
Пусть $X_1, \dotsc, X_n$ ---
выборка из биномиального распределения с параметрами $m$ и $p$ при известном $m$.
Найдите смещение и дисперсию оценки $ \hat{p} = X_1$ неизвестного параметра $p$.
Улучшите эту оценку усреднением по достаточной статистике.
Найдите смещение и дисперсию улучшенной оценки.
Выясните, является ли улучшенная оценка эффективной.

\textit{Решение.}
Ищем смещение $M \hat{p} = MX_1 = mp$, значит,
смещение
$$b_n \left( p \right) =
  M \hat{p} - p =
  mp - p =
  p \left( m - 1 \right).$$

Дисперсия равна $D \hat{p} = DX_1 = mp \left( 1 - p \right) $.

Запишем функцию правдоподобия данной выборки:
$$L \left( X_1, \dotsc, X_n, p \right) =
  \prod \limits_{i = 1}^n X_m^{X_i} p^{X_i} \left( 1 - p \right)^{m - X_i} =
  \prod \limits_{i = 1}^n C_m^{X_i} \cdot p^{ \sum \limits_{i = 1}^n X_i} \cdot
  \left( 1 - p \right)^{ \sum \limits_{i = 1}^n \left( m - X_i \right) }.$$
Запишем через выборочное среднее
$$ \prod \limits_{i = 1}^n C_m^{X_i} \cdot p^{ \sum \limits_{i = 1}^n X_i} \cdot
  \left( 1 - p \right)^{ \sum \limits_{i = 1}^n \left( m - X_i \right) } =
  \prod \limits_{i = 1}^n C_m^{X_i} \cdot p^{n \overline{X}} \cdot
  \left( 1 - p \right)^{n \left( m - \overline{X} \right) }.$$

Видим, что функция правдоподобия подаётся в виде произведения функции
$$h \left( X_1, \dotsc, X_n \right) =
  \prod \limits_{i = 1}^n C_m^{X_i},$$
которая не зависит от неизвестного параметра,
и функции
$$g \left( X_1, \dotsc, X_n, p \right) =
  p^{n \overline{X}} \left( 1 - p \right)^{n \left( m - \overline{X} \right) },$$
которая зависит от неизвестного параметра и от функции выборки
$$T \left( X_1, \dotsc, X_n \right) =
  \overline{X}.$$
Тогда, по теореме про характеризацию достаточной статистики,
функция $T \left( X_1, \dotsc, X_n \right) = \overline{X}$
является достаточной статистикой для параметра $p$.

Теперь, чтобы улучшить оценку $ \hat{p} = X_1$ параметра $p$,
усредним её по достаточной статистике
$ \overline{X}: \,
  p^* =
  M \left( \hat{p} \; \middle| \; \overline{X} \right) =
  M \left( X_1 \; \middle| \; \overline{X} \right) =
  \overline{X}$,
так как
$M \left( X_1 \; \middle| \; \overline{X} \right) =
  \dotsc =
  M \left( X_n \; \middle| \; \overline{X} \right) $.

Ищем смещение
$$Mp^* =
  M \overline{X} =
  M \left( \frac{1}{n} \sum \limits_{i = 1}^n X_i \right) =
  \frac{1}{n} \cdot nMX_1 =
  MX_1 =
  mp,$$
значит смещение
$b_n \left( p \right) =
  Mp^* - p =
  M \overline{X} - p =
  mp - p =
  p \left( m - 1 \right) $.

Вычисляем дисперсию статистики
$$Dp^* =
  D \overline{X} =
  D \left( \frac{1}{n} \sum \limits_{i = 1}^n X_i \right) =
  \frac{1}{n} \cdot DX_1 =
  \frac{1}{n} \cdot mp \left( 1 - p \right).$$

Так как оценка смещённая, то она не является эффективной.

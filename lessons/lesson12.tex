\addcontentsline{toc}{chapter}{Занятие 12. Проверка параметрических гипотез.
                              Критерий Неймана-Пирсона}
\chapter*{Занятие 12. Проверка параметрических гипотез. Критерий Неймана-Пирсона}

\addcontentsline{toc}{section}{Контрольные вопросы и задания}
\section*{Контрольные вопросы и задания}

\subsubsection*{Что называется статистической гипотезой?}

Статистическая гипотеза --- предположение о виде распределения и свойствах случайной величины,
которое можно подтвердить или опровергнуть применением статистических методов к данным выборки.

\subsubsection*{Какую гипотезу называют основной, альтернативной, простой, сложной?}

Нулевая гипотеза --- гипотеза, подлежащая проверке.
Альтернативная гипотеза --- каждая допустимая гипотеза, отличная от нулевой.
Нулевую гипотезу обозначают $H_0$, альтернативную ---
$H_1$ (от Hypothesis --- <<гипотеза>> (англ.)).

Простой гипотезой называют предположение, состоящее в том,
что неизвестная функция $F \left( t \right) $
отвечает некоторому совершенно конкретному вероятностному распределению.

Сложной гипотезой называют предположение о том,
что неизвестная функция $F \left( t \right) $ принадлежит некоторому множеству распределений,
состоящему из более чем одного элемента.

\subsubsection*{Что такое статистический критерий?}

Статистический критерий --- строгое математическое правило,
по которому принимается или отвергается та или иная статистическая
гипотеза с известным уровнем значимости.

\subsubsection*{Что такое уровень занчимости критерия для проверки статистической гипотезы?}

Можно отвергнуть гипотезу $F_1$, когда она будет верна.
В случае простой гипотезы $F_1$ вероятность ошибки равна  $P_{F_1} \left( \vec{X} \in D \right) $.
Эту вероятность называют уровнем значимости статистического критерия.

\subsubsection*{Какое множество называют критическим для проверки статистической гипотезы?}

Критическая область --- это совокупность значений статистики, которые <<говорят>>,
что нулевую гипотезу следует отвергнуть.

\subsubsection*{В чём состоит ошибка первого рода, второго рода?}

$1 - \alpha = P \left( H_1 \; \middle| \; H_0 \right) $ --- ошибка первого рода.
Означает, что отклонили нулевую гипотезу в то время, как на самом деле она истинна.

$ \beta = P \left( H_0 \; \middle| \; H_1 \right) $ --- ошибка второго рода.
Это означает принять нулевую гипотезу, которая на самом деле ложна.

\subsubsection*{Что называют мощностью критерия?}

$1 - \beta $ --- мощность критерия.

\subsubsection*{Сформулируйте критерий согласованности Колмогорова,
                критерий согласованности $ \chi^2$ Пирсона, лемму Неймана-Пирсона}

\textit{Критерий Колмогорова-Смирнова.}
Если $F$ --- непрерывное распределение,
то
$ \sqrt{n} \cdot
  \sup \limits_{ \mathbb{R}} \left[ F_n \left( x \right) - F \left( x \right) \right] \approx
  D_{ \theta }$
--- известное распределение.

\textit{Критерий Пирсона (критерий согласия $ \chi^2$).} Есть выборка $X_1, \dotsc, X_n$.
Имеет ли оа распределение $F$?
Попробуем устроить дискретную процедуру.
Разбиваем интервал возможных значений выборки на $N$ полуинтервалов (чисел, частей) ---
рис. \ref{fig:12}.

\begin{figure}[h!]
  \centering
  \includegraphics[width=.4\textwidth]{./pictures/12.png}
  \caption{Возможные значения выборки}
  \label{fig:12}
\end{figure}

Обозначим через $p_i = P_F \left( X_1 \in \Delta_i \right) > 0$.

Количество элементов выборки, которые попали в $ \Delta_i$, обозначим через
$$ \nu_i =
  \sum \limits_{k = 1}^n \mathbbm{1}_{ \Delta_i} \left( X_k \right).$$
Рассмотрим сумму
$$ \sum \limits_{i = 1}^n \frac{ \left( \nu_i - np_i \right)^2}{np_i} \approx
  \chi_{n - 1}^2$$
при $n \to \infty $ (после применения центральной предельной теоремы).

Все $ \nu_i$ в сумме дают $n$, следовательно, есть связь,
то есть $ \left( n - 1 \right) $-на степень свободы.

Так происходит, если угадали $p_i$, иначе --- выражение быстро растёт.

\textit{Лемма Неймана-Пирсона.}
$D_{C_{ \alpha }}$ --- оптимально, то есть для произвольной $D$,
такой что $P_{F_1} \left( D^C \right) = \alpha $, оказывается,
что $P_{F_2} \left( D \right) = P_{F_2} \left( D_{C_{ \alpha }} \right) $, то есть утверждается,
что множество уровня является оптимальным.

\addcontentsline{toc}{section}{Аудиторные задачи}
\section*{Аудиторные задачи}

\addcontentsline{toc}{section}{Домашнее задание}
\section*{Домашнее задание}

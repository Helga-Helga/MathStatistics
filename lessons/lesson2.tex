\addcontentsline{toc}{chapter}{Занятие 2. Свойства оценок}
\chapter*{Занятие 2. Свойства оценок}

\addcontentsline{toc}{section}{Контрольные вопросы и задания}
\section*{Контрольные вопросы и задания}

\subsubsection*{Что называют оценкой неизвестного параметра?}

Статистику, значение которой заменяет неизвестный параметр, называют оценкой этого параметра.

\subsubsection*{Преведиты определение оценки: несмещённой, ассимптотически несмещённой,
                состоятельной, сильно состоятельной, оптимальной.}

Оценка $ \hat{ \theta}$  несмещённая,
если $ \forall \theta \in \Theta: \, M_{ \theta } \hat{ \theta } = \theta $.

Асимптотически несмещенная оценка --- такая оценка,
математическое ожидание которой совпадает с оцениваемым параметром при $n \to \infty $.

Оценка $ \hat{ \theta }$ называется состоятельной,
если стремится к истинному значению $ \theta $ по вероятности
$ \hat{ \theta } \overset{P}{ \rightarrow } \theta, \,
  n \to \infty $.

Оценка $ \hat{ \theta }$ называется сильно состоятельной,
если стремится к истинному значению $ \theta $ почти наверное
$ \hat{ \theta } \overset{a.s.}{ \rightarrow } \theta, \,
  n \to \infty $.

Несмещённая оценка $ \hat{ \theta } \in K$
называется оптимальной в классе квадратично интегрируемых оценок $K$,
если для всякой другой несмещённой оценки
$ \tilde{ \theta } \in \Theta \,
  \forall \theta \in \Theta: \,
  D_{ \theta } \hat{ \theta } \leq D_{ \theta } \tilde{ \theta }$
или же
$ \forall \theta \in \Theta, \,
  M_{ \theta } \left( \hat{ \theta } - \theta \right)^ \leq
  M_{ \theta } \left( \tilde{ \theta } - \theta \right)^2$.

\subsubsection*{Что называется среднеквадратическим отклонением оценки?}

$M_{ \theta } \left( \hat{ \theta } - \theta \right) $ --- среднеквадратическое оклонение.

\subsubsection*{Сформулируйте утверждение про поведение выборочных моментов.}

\subsubsection*{Какая оценка является несмещённой и содержательной для математического ожидания
                распределения выборки?}

\subsubsection*{Какая статистика является несмещённой оценкой для дисперсии распределения выборки?}

$$ \frac{1}{n - 1} \sum \limits_{k = 1}^n \left( x_k - \overline{x} \right)^2$$
--- несмещённая оценка для $ \sigma^2 = Dx_1$.

\addcontentsline{toc}{section}{Аудиторные задачи}
\section*{Аудиторные задачи}

\subsubsection*{2.4}

\textit{Задание.}
Для выборки равномерного распределения на отрезке $ \left[ 0, 1 \right] $
проверьте состоятельность и несмещённость оценки $X_{ \left( 1 \right) }$ параметра $ \theta $.

\textit{Решение.} $ \theta $ --- минимальное наблюдение.
Проверяем,
выполняется ли $X_{ \left( 1 \right) )} \overset{P}{ \rightarrow } \theta, \, n \to \infty $.

По определению сходимости по вероятности
$$ \forall \varepsilon > 0 \,
  P \left( \left| X_{ \left( 1 \right) } - \theta \right| > \varepsilon \right) \to 0, \,
  n \to \infty.$$

Раскроем модуль
$$P \left\{ X_{ \left( 1 \right) } > \varepsilon + \theta \right\} =
  P \left( X_1 > \varepsilon + \theta, \dotsc, X_n > \varepsilon + \theta \right) =
  \left[ P \left( X_1 > \varepsilon + \theta \right) \right]^n.$$
Подставим значение вероятности из геометрического эксперимента
$$ \left[ P \left( X_1 > \varepsilon + \theta \right) \right]^n
  \left( \frac{1 - \theta - \varepsilon }{1 - \theta } \right)^n =
  \left( 1 - \frac{ \varepsilon }{1 - \theta } \right)^n \to
  0, \,
  n \to \infty.$$

Число в скобках строго меньше единицы, так как $0 \leq \theta \leq 1$.

Отсюда следует, что оценка состоятельная.

Проверяем несмещённость оценки.
Проверяем, выполняется ли
$$MX_{ \left( 1 \right) } =
  \theta.$$

Нужно найти плотность
$$MX_{ \left( 1 \right) } =
  \int \limits_{ \mathbb{R}} f_{X_{ \left( 1 \right) }} \left( y \right) ydy.$$

Начинаем с функции распределения
$F_{X_{ \left( 1 \right) }} \left( y \right) =
  P \left( X_{ \left( 1 \right) } \leq y \right) $.
Переходим к противоположному событию
$$P \left( X_{ \left( 1 \right) } \leq y \right) =
  1 - P \left( X_{ \left( 1 \right) } > y \right) =
  1 - \left[ P \left( X_1 > y \right) \right]^n.$$
Переходим к противоположному событию
$1 - \left[ P \left( X_1 > y \right) \right]^n =
  1 - \left[ 1 - F \left( y \right) \right]^n$.

Продифференцируем
$$ \frac{dF_{X_{ \left( 1 \right) }} \left( y \right) }{dy} =
  n \left[ 1 - F \left( y \right) \right]^{n - 1} f \left( y \right).$$
На отрезке $ \left[ \theta, 1 \right] $ имеет равномерное распределение
$$n \left[ 1 - F \left( y \right) \right]^{n - 1} f \left( y \right) =
  n \left[ 1 - \frac{y - \theta }{1 - \theta } \right]^{n - 1} \cdot
  \mathbbm{1} \left\{ y \in \left[ \theta, 1 \right] \right\} \cdot \frac{1}{1 - \theta }.$$
Приведём к общему знаменателю
$$n \left[ 1 - \frac{y - \theta }{1 - \theta } \right]^{n - 1} \cdot
  \mathbbm{1} \left\{ y \in \left[ \theta, 1 \right] \right\} \cdot \frac{1}{1 - \theta } =
  \frac{n}{ \left( 1 - \theta \right)^2} \cdot \left( 1 - y \right)^{n - 1} \cdot
  \mathbbm{1} \left\{ y \in \left[ \theta, 1 \right] \right\}.$$

Нашли плотность $X_{ \left( 1 \right) }$ и теперь можем вычислить интеграл
$$MX_{ \left( 1 \right) } =
  \int \limits_{ \theta }^1
    y \cdot \frac{n}{ \left( 1 - \theta \right)^2} \cdot \left( 1 - y \right)^{n - 1}
  dy.$$
Замена:
$$1 - y = z, \,
  dy = -dz, \,
  y = 1 - z, \,
  y = 1 \Rightarrow \implies z = 0, \,
  y = \theta \implies z = 1 - \theta.$$
Подставляя замену, получаем
$$ \int \limits_{ \theta }^1
    y \cdot \frac{n}{ \left( 1 - \theta \right)^2} \cdot \left( 1 - y \right)^{n - 1}
  dy =
  n \cdot \frac{1}{ \left( 1 - \theta \right)^n}
  \int \limits_0^{1 - \theta } \left( 1 - z \right) z^{n - 1} dz.$$
Вычислим интеграл
\begin{equation*}
  \begin{split}
    n \cdot \frac{1}{ \left( 1 - \theta \right)^n}
    \int \limits_0^{1 - \theta } \left( 1 - z \right) z^{n - 1} dz
    \frac{n}{ \left( 1 - \theta \right)^n}
    \left[
      \frac{ \left( 1 - \theta \right)^n}{n} - \frac{ \left( 1 - \theta \right)^{n+1}}{n + 1}
    \right] = \\
    = n \left( \frac{1}{n} - \frac{1 - \theta }{n + 1} \right) =
    1 - \frac{n}{n + 1} \left( 1 - \theta \right).
  \end{split}
\end{equation*}
Раскроем скобки
$$1 - \frac{n}{n + 1} \left( 1 - \theta \right) =
  1 - \frac{n}{n + 1} - \theta \cdot \frac{n}{n + 1} \neq
  \theta.$$
Отсюда следует, что оценка смещённая, но ассимптотически несмещённая, потому что
$$1 - \frac{n}{n + 1} \to 0, \,
  n \to \infty $$
и
$$ \frac{n}{n + 1} \to 1, \,
  n \to \infty.$$

\subsubsection*{2.5}

\textit{Задание.}
Пусть $X_1, \dotsc, X_n$ --- выборка из распределения Пуассона с параметром $ \lambda > 0$.
Выясните, является ли статистика
$$ \frac{1}{n} \sum \limits_{i = 1}^n X_i^2:$$
\begin{enumerate}[label=\alph*)]
  \item несмещённой оценкой для $ \lambda^2$;
  \item состоятельной оценкой для $ \lambda^2$.
\end{enumerate}

\textit{Решение.}
\begin{enumerate}[label=\alph*)]
  \item Нужно проверить, выполняется ли
  $$M \frac{1}{n} \sum \limits_{i = 1}^n X_i^2 =
    \lambda^2.$$
  Преобразуем левую часть
  $$M \frac{1}{n} \sum \limits_{i = 1}^n X_i^2 =
    MX_1^2 =
    DX_1 + \left( MX_1 \right)^2 =
    \lambda + \lambda^2 \neq
    \lambda^2.$$

  Значит, оценка смещённая;
  \item проверяем, имеет ли место
  $$ \frac{1}{n} \sum \limits_{i = 1}^n X_i^2  \overset{P}{ \rightarrow } \lambda^2, \,
    n \to \infty.$$

  По закону больших чисел
  $$ \frac{1}{n} \sum \limits_{i = 1}^n X_i^2 \to
    MX_1^2 =
    \lambda^2 + \lambda \neq
    \lambda^2,$$
  значит, оценка не состоятельная.
\end{enumerate}

\subsubsection*{2.6}

\textit{Задание.}
Пусть $X_1, \dotsc, X_n$ --- выборка из показательного распределения с параметром $ \alpha > 0$.
Докажите, что статистика $1 / \overline{X}$ является состоятельной оценкой для $ \alpha $.

\textit{Решение.} Нужно показать, что
$$ \frac{1}{ \overline{X}} \overset{P}{ \rightarrow } \alpha, \,
   n \to \infty.$$

Выборочное среднее
$$ \overline{X} =
  \frac{1}{n} \sum \limits_{i = 1}^n X_i.$$

По закону больших чисел
$$ \overline{X} \overset{P}{ \rightarrow }
  MX_1 =
  \frac{1}{ \alpha }.$$
Отсюда следует, что
$$ \frac{1}{ \overline{X}} \overset{P}{ \rightarrow }
  \frac{1}{MX_1} =
  \alpha.$$

\subsubsection*{2.7}

\textit{Задание.}
Пусть $X_1, \dotsc, X_n$ --- выборка из нормального распределения $N \left( a, \sigma^2 \right) $.
Докажите, что статистика
$$S_n =
  \frac{1}{2 \left( n - 1 \right) } \sum \limits_{i = 1}^{n - 1} \left( X_{i + 1} - X_i \right)^2$$
является несмещённой и состоятельной оценкой для $ \sigma^1$.

\textit{Решение.} Нужно проверить условие $MS_n = \sigma^2$.

Разность двух соседних элементов выборки имеет распределение
$$X_{i + 1} - X_i \sim
  N \left( 0, 2 \sigma^2 \right).$$

Найдём математическое ожидание статистики
$$MS_n =
  M \frac{1}{2 \left( n - 1 \right) } \cdot
  \sum \limits_{i = 1}^{n - 1} \left( X_{i + 1} - X_i \right)^2 =
  \frac{1}{2 \left( n - 1 \right) } \sum \limits_{i = 1}^n M \left( X_{i + 1} - X_i \right)^2.$$
Случайные величины одинаково распределены
$$ \frac{1}{2 \left( n - 1 \right) } \sum \limits_{i = 1}^n M \left( X_{i + 1} - X_i \right)^2 =
  \frac{n}{2 \left( n - 1 \right) } \cdot M \left( X_2 - X_1 \right)^2.$$
В данном случае второй момент равен  дисперсии
$$ \frac{n}{2 \left( n - 1 \right) } \cdot M \left( X_2 - X_1 \right)^2 =
  \frac{1}{2} \cdot 2 \sigma^2 =
  \sigma^2.$$
Отсюда следует, что оценка несмещённая.

Проверим состоятельность, то есть $S_n \overset{P}{ \rightarrow } \sigma^2, \, n \to \infty $.

Разобъём $S_n$ на две суммы
$$S_n =
  \frac{1}{2 \left( n - 1 \right) } \cdot
  \left[
    \sum \limits_{even \, i} \left( X_{i + 1} - X_i \right)^2 +
    \sum \limits_{odd \, i} \left( X_{i + 1} - X_i \right)^2
  \right].$$
В каждой из сумм слагаемые независимы
\begin{equation*}
  \begin{split}
    \frac{1}{2 \left( n - 1 \right) } \cdot
    \left[
      \sum \limits_{even \, i} \left( X_{i + 1} - X_i \right)^2 +
      \sum \limits_{odd \, i} \left( X_{i + 1} - X_i \right)^2
    \right] = \\
    = \frac{1}{2 \left( n - 1 \right) } \cdot
    \left[
      \frac{m}{m} \sum \limits_{even \, i} \left( X_{i + 1} - X_i \right)^2 +
      \frac{n - 1 - m}{n - 1 - m} \sum \limits_{odd \, i} \left( X_{i + 1} - X_i \right)^2
    \right].
  \end{split}
\end{equation*}
По закону больших чисел
\begin{equation*}
  \begin{split}
    \frac{1}{2 \left( n - 1 \right) } \cdot
    \left[
      \frac{m}{m} \sum \limits_{even \, i} \left( X_{i + 1} - X_i \right)^2 +
      \frac{n - 1 - m}{n - 1 - m} \sum \limits_{odd \, i} \left( X_{i + 1} - X_i \right)^2
    \right] \overset{P}{ \rightarrow } \\
    \overset{P}{ \rightarrow } \frac{1}{2} \cdot M \left( X_2 - X_1 \right)^2 =
    \frac{1}{2} \cdot D \left( X_2 - X_1 \right) =
    \sigma^2, \,
    n \to \infty.
  \end{split}
\end{equation*}
Отсюда следует, что оценка состоятельная.

\addcontentsline{toc}{section}{Домашнее задание}
\section*{Домашнее задание}
